%D \module
%D   [      file=s-dtk-01,
%D        version=1999.08.15,
%D          title=\CONTEXT\ Style File,
%D       subtitle=\DTK\ basis stijl,
%D         author=Hans Hagen \& Taco Hoekwater,  
%D           date=\currentdate,
%D      copyright=PRAGMA]
%C
%C This module is part of the \CONTEXT\ macro||package and is
%C therefore copyrighted by \PRAGMA. See mreadme.pdf for 
%C details. 

%D Preliminary. Still dutch and english intermixed. 

\unprotect

%D Temp hack. 

\def\doverbatimgoodbreak{}

%D Fonts. 

\definebodyfontenvironment 
  [8pt]  
  [\c!interlinie=9pt, 
   \c!groot=9pt,  
   \c!klein=7pt]

\definebodyfontenvironment 
  [9pt]  
  [\c!interlinie=11pt,
   \c!groot=10pt, 
   \c!klein=8pt]

\definebodyfontenvironment 
  [10pt]   
  [\c!interlinie=12pt,
   \c!groot=12pt, 
   \c!klein=8pt]

\definebodyfontenvironment
  [12pt]   
  [\c!interlinie=14pt,
   \c!groot=14.4pt,
   \c!klein=10pt]

\definebodyfontenvironment
  [14.4pt] 
  [\c!interlinie=18pt,
   \c!groot=14.4pt,
   \c!klein=12pt]

\setupbodyfont
  [10pt,cmx,ams]

%D Color.

\setupcolors
  [\c!conversie=\v!altijd] 

%D Default language.

\mainlanguage[en] \language[en]

%D Paper size and layout. The Dante style depends on the 
%D driver to sort out the margins. Therefore we are not 
%D really typesetting in A5, but on something larger. 

\definepapersize
  [DanteA5]
  [breedte=486.54pt]  

\setuppapersize
  [DanteA5][A4] 

\setuplayout % w=342pt h=484pt 
  [\c!rugwit=1in,
   \c!kopwit=1in,
   \c!plaats=\v!enkelzijdig,
   \c!breedte=342pt,  
   \c!hoogte=548pt, 
   \c!hoofd=17pt,   
   \c!hoofdafstand=17pt,
   \c!voetafstand=17pt,
   \c!voet=17pt]  

%D Headers and footers.
 
\setupheader [\c!voor=\vfill,\c!na=]
\setupfooter [\c!voor=\vfill,\c!na=]

%D Numbering. 

\setuppagenumbering
  [\c!plaats=,
   \c!variant=\v!dubbelzijdig]

%D Spacing. 

\stelwitruimtein [\v!halveregel]
\stelblankoin    [\v!halveregel]

%D Itemize. 

\stelopsommingin [\v!elk] [\c!afstand=0pt,\c!marge=.5em]
\stelopsommingin [1] [\c!breedte=1.0em,\c!symbool=circle]
\stelopsommingin [2] [\c!breedte=0.9em,\c!symbool=triangle]
\stelopsommingin [3] [\c!breedte=0.8em,\c!symbool=diamond]
\stelopsommingin [4] [\c!breedte=0.7em,\c!symbool=dash]

\stelopsommingin [1] [\v!opelkaar]

%D Verbatim. 

\setuptyping [\c!optie=,\c!blanko=halveregel]

%D Footnotes. 

\def\VoetNootLijn%
  {\hrule width 5pc height .4pt} 

\setupfootnotes
  [\c!korps=8pt,
   \c!lijn=\VoetNootLijn,
   \c!voor=\vskip24pt,
   \c!nummercommando=]

\setupfootnotedefinition
  [\c!plaats=\v!aansluitend,
   \c!breedte=\v!passend,
   \c!kopletter=\v!normaal,
   \c!afstand=.5em]

% Abbreviations and logos. 

\usemodule[abr-03]


%D Some real macros: 

\def\startAbstract%
  {\dostartbuffer[abstract][startAbstract][stopAbstract]}

%D And some dummies (for \MAPS\ style compatibility): 

\def\startKeywords#1\stopKeywords
  {}

\def\Keywords#1%
  {}

%D We will pick up some user settings. 

\def\DTKTypering[#1]%
  {\getparameters
     [DTK]
     [Jaar=1998,
      Volume=19,
      Nummer=3,
      Pagina=99,
      Titel=Publish or Perish,
      Subtitel=,
      Auteur=D.T.K. Auteur,
      Adres=PRAGMA Advanced Document Engineering \\
            Ridderstraat 27 \\ 8061GH Hasselt NL,
      Email=pragma@wxs.nl,
      #1]}

\DTKTypering[]

%D Some of those are shown in headers and footers.

\def\DTKFooter%
  {Die \TeX nische Kom\"odie \DTKVolume/\DTKJaar
   \space---\space
   Proceedings of the \DTKJaar\ Euro\TeX\ Meeting}

\setupheader[\c!linkerletter=\ss\sl,\c!rechterletter=\ss]
\setupfooter[\c!letter={\switchtobodyfont[9pt]\ss\sl}]

\setupheadertexts [\DTKTitel]  [\pagenumber]
\setupfootertexts [\DTKFooter] []

%D It all starts here: 

\def\dostartBijdrage[#1]%
  {\pagina
   \DTKTypering[#1]
   \setupheader[\c!status=\v!leeg]
   \stelpaginanummerin[\c!nummer=\DTKPagina]
   \bgroup
     \stelwitruimtein[\v!geen]
     \bgroup
       \switchtobodyfont[14.4pt,ss]
       \steluitlijnenin[\v!rechts]
       \let\\=\par 
       \DTKTitel\par
       \doifsomething{\DTKSubtitel}
         {\bgroup
          \vskip3pt
          \switchtobodyfont[12pt,ss]
          \DTKSubtitel\par
          \egroup}
     \egroup
     \vskip12pt
     \bgroup
       \switchtobodyfont[12pt,ss]
       \def\\{\unskip\space\ignorespaces} 
       \DTKAuteur 
       \par
     \egroup
     \vskip12pt
     \bgroup
       \switchtobodyfont[9pt]
       \stelsmallerin[voor=,na=]
       \startsmaller[.05\hsize]
         \stelwitruimtein[\v!halveregel]
         \haalbuffer[abstract]
       \stopsmaller
     \egroup
     \vskip20pt
    \egroup}

\def\startBijdrage%
  {\starttekst
   \dosingleempty\dostartBijdrage}

\def\stopBijdrage%
  {\stoptekst}

%D Sectioning.

\stelkopin
  [\v!paragraaf]
  [\c!letter={\ss\bfa},
   \c!uitlijnen=\v!rechts,
   \c!voor={\blanko[\v!regel,\v!halveregel]},
   \c!na={\blanko[\v!halveregel]}]

\stelkopin
  [\v!sub\v!paragraaf]
  [\c!letter=\ss,
   \c!uitlijnen=\v!rechts,
   \c!voor={\blanko[\v!halveregel]},
   \c!na={\blanko[\v!halveregel]}]

\stelkopin
  [\v!sub\v!sub\v!paragraaf]
  [\c!letter=\ss,
   \c!variant=\v!tekst,
   \c!voor=,
   \c!na=]

%D Done. 

\protect \endinput
