%D \module
%D   [       file=supp-mat,
%D        version=1998.09.10,
%D          title=\CONTEXT\ Support Macros,
%D       subtitle=Math,
%D         author=Hans Hagen,
%D           date=\currentdate,
%D      copyright={PRAGMA / Hans Hagen \& Ton Otten}]
%C
%C This module is part of the \CONTEXT\ macro||package and is
%C therefore copyrighted by \PRAGMA. See mreadme.pdf for 
%C details. 

%D For practical reasons, I decided to move some math things to 
%D a support module. There is nothing spectacular here. 

\writestatus{loading}{Context Support Macros / Math}

\unprotect

%D \macros
%D   {mathematics, math, nomathematics, startmathmode}
%D 
%D The \type{$} can be both an begin and end math character.
%D This can lead to confusing and errorprone situations when
%D macros insert \type{$}. When for instance we have defined: 
%D 
%D \starttypen
%D \def\MyPlus{$\,+\,$}
%D \stoptypen
%D 
%D the sequence \type{$x^2 \MyPlus y^2 = r^2$} will expand to:
%D 
%D \starttypen
%D $x^2 $\,+\,$ y^2 = r^2$
%D \stoptypen
%D 
%D Here the \type{\,} are given outside math mode and \TEX\ will
%D definitely complain about this. A more save definition would 
%D have been: 
%D 
%D \starttypen
%D \def\MyPlus{\mathematics{\,+\,}}
%D \stoptypen
%D 
%D Which is implemented as:

\def\mathematics#1%
  {\ifmmode#1\else$#1$\fi}

\def\startmathmode
  {\ifmmode
     \let\stopmathmode\relax
   \else
     $\def\stopmathmode{$}%  \let\stopmathmode=$
   \fi}

\def\nomathematics#1%
  {\ifmmode\hbox{#1}\else#1\fi}

\let\math\mathematics

%D \macros
%D   {dimension, nodimension}
%D
%D The next few macros are used for typesetting dimensions in
%D such a way that spacing is acceptable. I won't spend much 
%D words on these macros, because they will be overloaded in 
%D the units module.

\newsignal\dimensionsignal 

\def\dimensiontypeface  {\tf}
\def\dimensionhalfspace {\,}

\unexpanded\def\dimension#1%
  {\def\dodimensionsignal{\kern\dimensionsignal}%
   \ifdim\lastskip=\zeropoint\relax
     \ifdim\lastkern=\zeropoint\relax
       \ifmmode
         \mathematics{\dimensionhalfspace\dimensionhalfspace\dimensiontypeface#1}%
       \else
         \mathematics{\dimensiontypeface#1}%
       \fi
     \else\ifdim\lastkern=\dimensionsignal
       \mathematics{\dimensionhalfspace\dimensiontypeface#1}%
     \else
       \unkern\mathematics{\dimensionhalfspace\dimensionhalfspace\dimensiontypeface#1}%
     \fi\fi
   \else
     \unskip\mathematics{\dimensionhalfspace\dimensionhalfspace\dimensiontypeface#1}%
   \fi
   \dodimensionsignal}

\unexpanded\def\nodimension#1%
  {\unskip#1\global\let\dodimensionsignal\relax}

%D \macros
%D   {super, suber}
%D
%D \TEX\ uses \type{^} and \type{_} for entering super- and 
%D subscript mode. We want however a bit more control than 
%D normally provided, and therefore provide \type {\super} 
%D and \type{\suber} (\type {\sub} is already taken). 

\global\let\normalsuper=^ 
\global\let\normalsuber=_ 

\newcount\supersubmode

\newevery\everysupersub \EverySuperSub

\appendtoks \advance\supersubmode 1\relax \to \everysupersub

% \def\dodosuper#1{\normalsuper{\the\everysupersub#1}}
% \def\dodosuber#1{\normalsuber{\the\everysupersub#1}}
% 
% \def\dosuper{\ifx\next\bgroup\expandafter\dodosuper\else\normalsuper\fi}
% \def\dosuber{\ifx\next\bgroup\expandafter\dodosuber\else\normalsuber\fi}
% 
% \def\super{\futurelet\next\dosuper}
% \def\suber{\futurelet\next\dosuber}

\def\super#1{\normalsuper{\the\everysupersub#1}}
\def\suber#1{\normalsuber{\the\everysupersub#1}}

%D \macros
%D   {enablesupsub}
%D 
%D We can let \type {^} and \type {_} act like \type {\super} 
%D and \type {\suber} by saying \type {\enablesupsub}. 

\bgroup
\catcode`\^=\@@active
\catcode`\_=\@@active
\gdef\enablesupsub
  {\catcode`\^=\@@active
   \def^{\ifmmode\expandafter\super\else\expandafter\normalsuper\fi}%
   \catcode`\_=\@@active
   \def_{\ifmmode\expandafter\suber\else\expandafter\normalsuber\fi}}
\egroup

%D \macro
%D   {restoremathstyle}
%D
%D We can pick up the current math style by calling \type 
%D {\restoremathstyle}.

\def\restoremathstyle
  {\ifmmode
     \ifcase\supsubmode
       \textstyle
     \or
       \scriptstyle
     \else
       \scriptscriptstyle
     \fi
   \fi}

\protect \endinput
