%D \module
%D   [       file=core-mat,
%D        version=1998.12.07,
%D          title=\CONTEXT\ Core Macros,
%D       subtitle=Math Fundamentals,
%D         author=Hans Hagen,
%D           date=\currentdate,
%D      copyright={PRAGMA / Hans Hagen \& Ton Otten}]
%C
%C This module is part of the \CONTEXT\ macro||package and is
%C therefore copyrighted by \PRAGMA. See licen-en.pdf for 
%C details. 

\writestatus{loading}{Context Core Macros / Math Fundamentals}

\unprotect

%D \macros 
%D   {big..}
%D
%D Because they are related to the bodyfontsize, we redefine 
%D some \PLAIN\ macros. 

\def\@@dobig#1#2%
  {{\hbox{$\left#2\vbox\!!to#1\bodyfontsize{}\right.\n@space$}}}

\def\big {\@@dobig{0.85}}
\def\Big {\@@dobig{1.15}}
\def\bigg{\@@dobig{1.45}}
\def\Bigg{\@@dobig{1.75}}

%D \macros 
%D   {bordermatrix}
%D
%D We already redefined \type {\bordermatrix} in \type 
%D {font-ini}.

%D \macros
%D   {super, sub}
%D
%D \TEX\ uses \type{^} and \type{_} for entering super- and 
%D subscript mode. We want however a bit more control than 
%D normally provided, and therefore provide \type {\super} 
%D and \type{sub}. 

\global\let\normalsuper=^ 
\global\let\normalsub  =_ 

\newcount\supersubmode

\newevery\everysupersub \EverySuperSub

\appendtoks \advance\supersubmode by 1 \to \everysupersub

% \def\dodosuper#1{\normalsuper{\the\everysupersub#1}}
% \def\dodosub  #1{\normalsub  {\the\everysupersub#1}}
% 
% \def\dosuper{\ifx\next\bgroup\expandafter\dodosuper\else\normalsuper\fi}
% \def\dosub  {\ifx\next\bgroup\expandafter\dodosub  \else\normalsub  \fi}
% 
% \def\super  {\futurelet\next\dosuper}
% \def\sub    {\futurelet\next\dosub  }

\def\super#1{\normalsuper{\the\everysupersub#1}}
\def\sub  #1{\normalsub  {\the\everysupersub#1}}

%D \macros
%D   {enablesupersub}
%D 
%D We can let \type {^} and \type {_} act like \type {\super} 
%D and \type {\sub} by saying \type {\enablesupersub}. 

\bgroup
\catcode`\^=\@@active
\catcode`\_=\@@active
\gdef\enablesupersub%
  {\catcode`\^=\@@active
   \def^{\ifmmode\expandafter\super\else\expandafter\normalsuper\fi}%
   \catcode`\_=\@@active
   \def_{\ifmmode\expandafter\sub  \else\expandafter\normalsub  \fi}}
\egroup

%D \macro
%D   {restoremathstyle}
%D
%D We can pick up the current math style by calling \type 
%D {\restoremathstyle}.

\def\restoremathstyle%
  {\ifmmode
     \ifcase\supersubmode
       \textstyle
     \or
       \scriptstyle
     \else
       \scriptscriptstyle
     \fi
   \fi}

%D \macros
%D   {mathstyle}
%D
%D If one want to be sure that something is typeset in the 
%D appropriate style, \type {\mathstyle} can be used:
%D
%D \starttypen
%D \mathstyle{something}
%D \stoptypen

\def\mathstyle#1%
  {\mathchoice
     {\displaystyle     #1}%
     {\textstyle        #1}%
     {\scriptstyle      #1}%
     {\scriptscriptstyle#1}}

%D \macros
%D   {frac}
%D
%D This is another one Tobias asked for. It replaces the 
%D primitive \type {\over}. We also take the opportunity to 
%D handle math style restoring, which makes sure units and 
%D chemicals come out ok. 

\def\frac#1#2%
  {\relax
   \ifmmode
      {{\mathstyle{#1}}\over{\mathstyle{#2}}}%
   \else
      $\frac{#1}{#2}$%
   \fi}

\def\frac#1#2%
  {\relax\mathematics{{\mathstyle{#1}}\over{\mathstyle{#2}}}}

%D The next macro, \type {\ch}, is \PPCHTEX\ aware. In
%D formulas one can therefore best use \type {\ch} instead of
%D \type {\chemical}, especially in fractions. 

\ifx\mathstyle\undefined 
  \let\mathstyle\relax
\fi

\def\ch#1%
  {\ifx\@@chemicalletter\undefined
     \mathstyle{\rm#1}%
   \else
     \dosetsubscripts
     \mathstyle{\@@chemicalletter{#1}}%
     \doresetsubscripts
   \fi}

\protect

\endinput 
