%D \module
%D   [       file=font-uni,
%D        version=1999.10.10,
%D          title=\CONTEXT\ Font Macros,
%D       subtitle=\UNICODE\ Initialization,
%D         author=Hans Hagen,
%D           date=\currentdate,
%D      copyright={PRAGMA / Hans Hagen \& Ton Otten}]
%C
%C This module is part of the \CONTEXT\ macro||package and is
%C therefore copyrighted by \PRAGMA. See mreadme.pdf for 
%C details. 

\writestatus{loading}{Context Font Macros / UNICODE Support}

%D I wrote this module when \WangLei\ asked me how to use
%D Chinese in \CONTEXT. From the samples he sent me, I deduced
%D that some mixture of one and two byte encoding was used,
%D which he confirmed. Since \TEX\ normally does not use the
%D characters $>127$, so as long as the two byte characters
%D have a first character with code $>127$, we can use active
%D characters to handle them. In an optimistic mood, I called
%D this module the \UNICODE\ font module. In the module that 
%D handles Chinese, we will see that some more interpretation 
%D is involved, which is why the macros handling those 
%D characters look ahead. 

\startmessages  dutch  library: fonts
     21: het is veiliger om (pdf)eTeX te gebruiken
\stopmessages

\startmessages  english  library: fonts
     21: using (pdf)eTeX is more save 
\stopmessages

\startmessages  german  library: fonts
     21: Verwenden von (pdf)eTeX ist sicherer
\stopmessages

\startmessages  italian  library: fonts
     21: l'uso di (pdf)eTeX � pi� sicuro
\stopmessages

\startmessages  norwegian  library: fonts
     21: � bruke (pdf)eTeX er tryggere 
\stopmessages

\unprotect 

%D \macros 
%D   {handleunicodeflowglyph, uchar, 
%D    handleunicodeglyph, insertunicodeglyph,
%D    unicodeposition, unicodeone, unicodetwo}
%D
%D For the moment \UNICODE\ support is rather primitive but 
%D nevertheless effective. The reference to \UNICODE\ is not 
%D entirely correct, since in many cases one will use \quote 
%D {older} mappings, but in principle, \UNICODE\ can be 
%D supported. 
%D 
%D We expect each character to come as two eight bit
%D characters. Those doubles are handled by making all
%D characters in the range $>127$ active, so that they can
%D pick up the next one, and act upon both their values.
%D Internally only numbers are used. A first implementation
%D simply internally prefixed the second part of the \UNICODE\
%D pair with \type {\string} or \type {\char}, but this was
%D not that handy when it came to testing those values.
%D Because in principle we are dealing with an encoding, the
%D making active is handled in \type {enco-uni}. 
%D 
%D There are two commands to handle unicode characters: 
%D 
%D \starttypen 
%D \handleunicodeflowglyph{number}{character}
%D \uchar{number}{number}
%D \stoptypen
%D 
%D The first one can be assigned to an active character, the 
%D second one can be used to directly access a glyph. Both 
%D command call \type {\handleunicodeglyph} that in turn 
%D calls \type {\insertunicodeglyph}. Both can be overruled 
%D in specialized modules. The low level command \type 
%D {\unicodeglyph} can best be left untouched, which is not 
%D so much a problem because there is a hook into this macro:
%D \type {\unicodecharcommand}. 
%D 
%D In most cases one will redefine \type {\handleunicodeglyph}
%D in such a way that it identifies special situations first,
%D takes some actions next, calls \type {\insertunicodeglyph},
%D if needed with \type {\unicodecharcommand} changed, and
%D finally does some finishing: 
%D 
%D \starttyping 
%D \def\handleunicodeglyph
%D   {take actions based on \unicodeone-two-position cq. \nextglyph 
%D    redefine \unicodecharcommand if needed
%D    expand \insertunicodeglyph 
%D    take some final actions}
%D \stoptyping 

\newcount\unicodeposition

% \def\handleunicodeflowglyph#1#2%
%   {\def\dohandleunicodeflowglyph{\dodohandleunicodeflowglyphA{#1}{#2}}%
%    \futurelet\nextglyph\dohandleunicodeflowglyph}
% 
% %D The first alternative (A) takes a character as second 
% %D argument and is used in the primary handler introduced 
% %D earlier. 
% 
% \def\dodohandleunicodeflowglyphA#1#2%
%   {\bgroup
%    \def\unicodeone{#1}%  
%    \edef\unicodetwo{\number\expandafter`\string#2}% takes also <128 ! 
%    \unicodeposition=\unicodeone\unicodetwo\relax
%    \handleunicodeglyph
%    \egroup}
% 
% %D The second alternative (B) takes numbers and is used in 
% %D the \type {\uchar} command. This command takes two 
% %D decimal numbers. 
% 
% \def\dodohandleunicodeflowglyphB#1#2%
%   {\bgroup
%    \def\unicodeone{#1}%  
%    \def\unicodetwo{#2}%  
%    \unicodeposition=\unicodeone\unicodetwo\relax
%    \handleunicodeglyph
%    \egroup}
%
% \unexpanded\def\uchar%
%   {\let\nextglyph\relax
%    \dodohandleunicodeflowglyphB}

%D The multistep approach is needed to pick up the second 
%D token, since this token can have any value and any 
%D catcode. 

%\def\handleunicodeflowglyph#1%
%  {\bgroup
%   \edef\unicodeone{#1}%
%   \@EA\afterassignment\@EA\dohandleunicodeflowglyph
%     \@EA\chardef\@EA\nexttoken\@EA`\string}
%
%\def\dohandleunicodeflowglyph
%  {\futurelet\nextglyph\dodohandleunicodeflowglyph}
%
%\def\dodohandleunicodeflowglyph%
%  {\edef\unicodetwo{\the\nexttoken}% 
%   \unicodeposition\unicodeone\unicodetwo\relax
%   \handleunicodeglyph
%   \egroup}

% the \relax trick prevents eating up the space (needed for 
% korean

\def\handleunicodeflowglyph#1#2%
  {\bgroup
   \edef\unicodeone{#1}%
   \@EA\afterassignment\@EA\dohandleunicodeflowglyph
     \@EA\chardef\@EA\nexttoken\@EA`\string#2\relax}

\def\dohandleunicodeflowglyph\relax
  {\futurelet\nextglyph\dodohandleunicodeflowglyph}

\def\dodohandleunicodeflowglyph
  {\edef\unicodetwo{\the\nexttoken}%
   \unicodeposition\unicodeone\unicodetwo\relax
   \handleunicodeglyph
   \egroup}

\unexpanded\def\uchar#1#2% use as standalone glyph  
  {\bgroup
   \let\nextglyph\relax
   \edef\unicodeone{#1}%  
   \edef\unicodetwo{#2}% 
%   \unicodeposition=\unicodeone\unicodetwo\relax
   \unicodeposition\unicodeone\unicodetwo\relax
   \handleunicodeglyph
   \egroup}

\def\dohandleucflowglyph
%  {\unicodeposition=\unicodeone\unicodetwo\relax
  {\unicodeposition\unicodeone\unicodetwo\relax
   \handleunicodeglyph
   \egroup}

\unexpanded\def\uc#1#2% used in tricky situations
  {\bgroup
   \edef\unicodeone{#1}%
   \edef\unicodetwo{#2}%
   \futurelet\nextglyph\dohandleucflowglyph}

% Alternative, handles [char >127]{number} too. 
%
% \def\handleunicodeflowglyph#1%
%   {\bgroup
%    \edef\unicodeone{#1}%
%    \futurelet\nexttoken\dohandleunicodeflowglyph}
% 
% \def\dohandleunicodeflowglyph%
%   {\ifx\nexttoken\bgroup
%      \expandafter\dohandleunicodeflowglyphA
%    \else
%      \expandafter\dohandleunicodeflowglyphB
%    \fi}
% 
% \def\dohandleunicodeflowglyphA#1%
%   {\chardef\nexttoken=#1\relax
%    \dodohandleunicodeflowglyph}
% 
% \def\dohandleunicodeflowglyphB%
%   {\@EA\afterassignment\@EA\dodohandleunicodeflowglyph
%      \@EA\chardef\@EA\nexttoken\@EA=\@EA`\string}
% 
% \def\dodohandleunicodeflowglyph%
%   {\futurelet\nextglyph\dododohandleunicodeflowglyph}
% 
% \def\dododohandleunicodeflowglyph% 
%   {\edef\unicodetwo{\the\nexttoken}% 
%    \unicodeposition=\unicodeone\unicodetwo\relax
%    \handleunicodeglyph
%    \egroup}
%
% Can be used with (the less byte hungry alternative): 
%
% sub unirecode
%  { my ($a,$b) = @_ ;
%    if ((ord($b)<0x80)&&($b !~ /[a-zA-Z0-9]/))
%      { print "$b" ; ++$recoded ;  
%       #return "\\uc\{" . ord($a) . "\}\{". ord($b) . "\}" }  
%        return "$a\{". ord($b) . "\}" }  
%    else
%      { return "$a$b" } } 

\def\insertunicodeglyph
  {\unicodeglyph\unicodeone\unicodetwo}

\let\handleunicodeglyph=\insertunicodeglyph

%D One can use the \type {\unicodeposition} in the macros 
%D that handle pre and post material. 

%D \macros 
%D   {unicodestyle, unicodecharcommand}
%D
%D Each character pair will become one glyph. Because \TEX\ 
%D cannot handle fonts with more that 256 characters, we use 
%D \TFM\ files for each range. The first character of the pair 
%D is appended to the name of a font, and the second is used to
%D access the glyph in that font. This means that a particular 
%D font is split up in subfonts with names in the range: 
%D 
%D \starttypen
%D <filename>80 ... <filename>ff
%D \stoptypen
%D 
%D The \type {<filename>} as well as the composed name are
%D mapped ones. The next macros take care of this mapping.
%D Let us assume that the next mapping has taken place, 
%D 
%D \starttyping 
%D \definefontsynonym [UnicodeRegular] [gbsong] 
%D \stoptyping
%D 
%D Let us also assume that we are dealing with the range \type
%D {b1}. Given that a font name results from:  
%D 
%D \starttyping 
%D \truefontname{\truefontname{UnicodeRegular}b1}
%D \stoptyping
%D 
%D we get \type {gbsongb1}. The outer \type {\truefontname} 
%D takes care of additional mapping, so when we say: 
%D 
%D \starttyping 
%D \definefontsynonym [gbsongb1] [gbsong-b1] 
%D \stoptyping
%D 
%D the filename used will be \type {gbsong-b1}. From the next 
%D definition it will be clear that other fontshapes are also 
%D supported. The prefix \type {Unicode} is mapped! 
%D
%D The command \type {\unicodecharcommand} can be used to 
%D handle special cases. At that moment \type {1em} is known.  

\def\unicodestyle
  {\truefontname\s!Unicode\fontstylesuffix}

\let\unicodecharcommand\firstofoneargument

% \unexpanded\def\unicodeglyph#1#2% watch the double mapping 
%   {\bgroup
%    \bodyfontsize=\unicodescale\bodyfontsize
%    \font\unicodefont=\truefontname{\truefontname\unicodestyle\lchexnumbers{#1}}
%      at \currentfontscale\bodyfontsize
%    \unicodestrut\unicodefont\unicodecharcommand{\char#2\relax}%
%    \egroup}

\unexpanded\def\unicodeglyph#1#2% watch the double mapping
  {\bgroup
   \getvalue{@@\currentucharmapping\strippedcsname\uchar}{#1}{#2}%
   \bodyfontsize\unicodescale\bodyfontsize
   \font\unicodefont=\truefontname{\truefontname\unicodestyle\unicodeone}
     at \currentfontscale\bodyfontsize
   \unicodestrut\unicodefont\unicodecharcommand{\char\unicodetwo\relax}%
   \egroup}

%D \macros 
%D   {currentucharmapping,defineucharmapping}
%D
%D A (plane,char) pair can be remapped using a uchar mapping 
%D function. The default mapping is to convert the plane to a
%D lowercase hexadecimal number, and leave the number 
%D untouched. The current remapping is kept in a macro.

\let\currentucharmapping\s!default

\def\defineucharmapping#1%
  {\setvalue{@@#1\strippedcsname\uchar}}

\defineucharmapping{\s!default}#1#2%
  {\edef\unicodeone{\lchexnumbers{#1}}\edef\unicodetwo{#2}}

%D An example of a remapping is the following: 
%D 
%D \starttypen 
%D \defineucharmapping{GBK}#1#2%
%D   {\unicodeposition=#1
%D    \advance\unicodeposition -129
%D    \multiply\unicodeposition 190
%D    \advance\unicodeposition #2
%D    \advance\unicodeposition-\ifnum#2>127 65\else64\fi
%D    \dorepositionunicode}
%D \stoptypen
%D 
%D This maps the GBK vector onto a compact GBK one. The 
%D auxiliary macro is defined here as a goody. 

\def\dorepositionunicode
  {\DoDiv\unicodeposition by256to\scratchcounter
   \advance\scratchcounter 1
   \edef\unicodeone{\ifnum\scratchcounter<10 0\fi\the\scratchcounter}%
   \DoMod\unicodeposition by256to\scratchcounter
   \edef\unicodetwo{\the\scratchcounter}}

%D \macros 
%D   {setunicodestrut, setunicodescale, nextglyph, 
%D    handleunicodeglyph, insertunicodeglyph}
%D
%D A careful analysis of the previous macros, learns that the 
%D process of mapping comes down to: 
%D 
%D \startopsomming[opelkaar,n]
%D \som  taking care of preceding material (and spacing) 
%D \som  defining the font at \type {\currentfontscale} $\times$ 
%D       \type {\unicodescale} $\times$ \type {\bodyfontsize} 
%D \som  inserting a \type {\unicodestrut}
%D \som  inserting the character (glyph) 
%D \som  executing some actions afterwards
%D \stopopsomming
%D
%D The actions before and after placing the glyph, is up to 
%D the user supplied handler. This handler (\type 
%D {\handleunicodeglpyh}) must, at a certain moment, insert 
%D the glyph using \type {\insertunicodeglyph}

\def\setunicodescale#1%
  {\def\unicodescale{#1}}

\def\dosetunicodestrut#1#2% height depth
  {\def\unicodestrut
     {\vrule
        \!!width\zeropoint
        \!!height#1\ht\strutbox
        \!!depth#2\dp\strutbox
        \relax}}

\def\setunicodestrut#1#2% height depth
  {\ifdim#1\ht\strutbox>\zeropoint
     \dosetunicodestrut{#1}{#2}%
   \else\ifdim#1\dp\strutbox>\zeropoint
     \dosetunicodestrut{#1}{#2}%
   \else
     \let\unicodestrut\empty
   \fi\fi}

%D The additional scaling and strut default to: 

\setunicodescale{1}
\setunicodestrut{1}{1}

%D The actual code for the additional actions as well as 
%D specific spacing is handled outside these routines. The 
%D character after the two that are under treatment is 
%D available in \type {\nextglyph}.

%D \macros 
%D   {defineunicodefont, setupunicodefont}
%D
%D Apart from this rather low level implementation, we also 
%D provide a more user friendly alternative. Given that one 
%D has defined: 
%D 
%D \starttypen 
%D \defineunicodefont
%D   [SimChi] [SimplifiedChinese] 
%D   [   \c!schaal=0.85,
%D       \c!hoogte=1.25,
%D       \c!diepte=1.00,
%D   \c!interlinie=yes,
%D    \c!conversie=\chinesenumber,
%D     \c!commando=\handlechineseunicodeglyph]
%D \stoptypen
%D 
%D Together with: 
%D 
%D \starttypen 
%D \definefontsynonym [SimplifiedChineseRegular] [gbsong]  
%D \definefontsynonym [SimplifiedChineseSlanted] [gbsongsl] 
%D \stoptypen
%D 
%D we can now switch to Simplified Chinese by saying \type 
%D {SimChi}. Some values can be changed afterwards with 
%D
%D \starttypen 
%D \setupunicodefont[SimChi][...=...]
%D \stoptypen
%D
%D Specific initializations can be assigned to \type 
%D {commands}. 

\beginTEX 

  \def\unicodeTEXwarning
     {\writeline\showmessage\m!fonts{21}\empty\writeline
      \global\let\unicodeTEXwarning\relax}

\endTEX


\beginETEX \protected

  \let\unicodeTEXwarning\relax

\endETEX

\def\defineunicodefont
  {\dotripleempty\dodefineunicodefont}

\def\dodefineunicodefont[#1][#2][#3]%
  {\unicodeTEXwarning
   \doifassignmentelse{#3}
     {\setupunicodefont[#1][#3]}
     {\doifelsenothing{#3}
        {\setupunicodefont[#1][#3]}
        {\copyparameters
           [\??uc#1][\??uc#3]
           [\c!hoogte,\c!diepte,\c!schaal,\c!commandos,
            \c!interlinie,\c!commando,\c!conversie]}}%
   \doifelsenothing{#2}
     {\setvalue{#1}{[uc font #1 undefined]}}
     {\setvalue{\??uc#1\c!file}{#2}%
      \doifundefined{\??ff#2\s!Bold}
        {\definefontsynonym[#2\s!Bold]       [#2\s!Regular]%
         \definefontsynonym[#2\s!Slanted]    [#2\s!Regular]%
         \definefontsynonym[#2\s!Italic]     [#2\s!Regular]%
         \definefontsynonym[#2\s!BoldSlanted][#2\s!Slanted]%
         \definefontsynonym[#2\s!BoldItalic] [#2\s!Italic]}%
      \unexpanded\setvalue{#1}{\enableunicodefont{#1}}}}

\def\setupunicodefont
  {\dodoubleempty\dosetupunicodefont}

\def\dosetupunicodefont[#1][#2]% also predefines 
  {\doifundefined{\??uc#1\c!commando}
     {\copyparameters
        [\??uc#1][\??uc\s!default]
        [\c!hoogte,\c!diepte,\c!schaal,\c!commandos,
         \c!interlinie,\c!commando,\c!conversie]}%
   \getparameters[\??uc#1][#2]}
   
\def\enableunicodefont#1%
  {\definefontsynonym[\s!Unicode][\getvalue{\??uc#1\c!file}]%
   \def\unicodescale             {\getvalue{\??uc#1\c!schaal}}%
   \def\unicodeheight            {\getvalue{\??uc#1\c!hoogte}}%
   \def\unicodedepth             {\getvalue{\??uc#1\c!diepte}}%
   \def\unicodedigits            {\getvalue{\??uc#1\c!conversie}}%
   \def\handleunicodeglyph       {\getvalue{\??uc#1\c!commando}}%
   \enableregime[unicode]%
   \doifvalue{\??uc#1\c!interlinie}\v!ja{\stelinterliniein\relax}%
   \getvalue{\??uc#1\c!commandos}\relax}

%D \macros 
%D   {unicodedigits}
%D 
%D For convenience we also predefine a number conversion 
%D macro: 

\let\unicodedigits\number

%D Because we cannot be sure of the pressence of all font 
%D styles, we remap some by default. 

\definefontsynonym [\s!Unicode\s!Bold]        [\s!Unicode\s!Regular]
\definefontsynonym [\s!Unicode\s!Slanted]     [\s!Unicode\s!Regular]
\definefontsynonym [\s!Unicode\s!Italic]      [\s!Unicode\s!Regular]
\definefontsynonym [\s!Unicode\s!BoldSlanted] [\s!Unicode\s!Slanted]
\definefontsynonym [\s!Unicode\s!BoldItalic]  [\s!Unicode\s!Italic]

\setupunicodefont
  [\s!default]
  [\c!hoogte=1,
   \c!diepte=1,
   \c!schaal=1,
   \c!interlinie=\v!ja,
   \c!commando=\insertunicodeglyph,
   \c!conversie=\number]

\protect \endinput 
