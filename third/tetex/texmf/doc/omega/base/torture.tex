% This file is changed by Roozbeh Pournader to make it work
% with Omega 1.23 and fixing obvious bugs
\documentclass[a4paper,11pt]{article}
\usepackage{amsmath}
\usepackage{omega}
%\usepackage[dvips]{draftcopy}\draftcopyName{\today}{140}
\def\shortarab#1{{\pushocplist\ArabicOCP\fontfamily{omarb}\selectfont#1\popocplist}}
\def\shortberber#1{{\pushocplist\ArabicBerberOCP\fontfamily{omarb}\selectfont#1\popocplist}}
\def\shortgreek#1{{\pushocplist\GreekOCP\fontfamily{omlgc}\selectfont#1\popocplist}}
\def\shortlatberber#1{{\pushocplist\LatinBerberOCP\fontfamily{omlgc}\selectfont#1\popocplist}}
\def\shorttifi#1{{\pushocplist\TifinaghOCP\fontfamily{omlgc}\selectfont#1\popocplist}}
\def\shortpashto#1{{\pushocplist\AfghaPashtoOCP\fontfamily{omarb}\selectfont#1\popocplist}}
\def\shortpashtop#1{{\pushocplist\PakiPashtoOCP\fontfamily{omarb}\selectfont#1\popocplist}}
\def\shortsindhi#1{{\pushocplist\SindhiOCP\fontfamily{omarb}\selectfont#1\popocplist}}
\def\tl#1#2#3#4#5#6{\hline\rule[-5pt]{0pt}{14pt}\texttt{#1}&\shortarab{#1}&\texttt{#2}&\shortarab{#2}&\texttt{#3}&\shortarab{#3}&
\texttt{#4}&\shortarab{#4}&\texttt{#5}&\shortarab{#5}&\texttt{#6}&\shortarab{#6}\\}
%
\def\ttl#1#2#3{\hline\rule[-5pt]{0pt}{14pt}\texttt{#1}&\shortlatberber{#1}&\shortberber{#1}&\shorttifi{#1}&
\texttt{#2}&\shortlatberber{#2}&\shortberber{#2}&\shorttifi{#2}&
\texttt{#3}&\shortlatberber{#3}&\shortberber{#3}&\shorttifi{#3}\\}
%
\def\stl#1#2#3#4#5#6{\hline\rule[-5pt]{0pt}{14pt}\texttt{#1}&\shortsindhi{#1}&\texttt{#2}&\shortsindhi{#2}&\texttt{#3}&\shortsindhi{#3}&
\texttt{#4}&\shortsindhi{#4}&\texttt{#5}&\shortsindhi{#5}&\texttt{#6}&\shortsindhi{#6}\\}
\def\patl#1#2#3#4#5#6{\hline\rule[-5pt]{0pt}{14pt}\texttt{#1}&\shortpashto{#1}&\texttt{#2}&\shortpashto{#2}&\texttt{#3}&\shortpashto{#3}&
\texttt{#4}&\shortpashto{#4}&\texttt{#5}&\shortpashto{#5}&\texttt{#6}&\shortpashto{#6}\\}
\def\paptl#1#2#3#4#5#6{\hline\rule[-5pt]{0pt}{14pt}\texttt{#1}&\shortpashtop{#1}&\texttt{#2}&\shortpashtop{#2}&\texttt{#3}&\shortpashtop{#3}&
\texttt{#4}&\shortpashtop{#4}&\texttt{#5}&\shortpashtop{#5}&\texttt{#6}&\shortpashtop{#6}\\}
\begin{document}
\setcounter{page}{63}
\title{Multilingual Typesetting with \OMEGA, a Case Study: Arabic}
\author{Yannis Haralambous\thanks{Atelier Fluxus Virus, 187, rue Nationale,
59800 Lille, France, \texttt{yannis@fluxus-virus.com}}
\and
John Plaice\thanks{School of Computer Science and Engineering,
The University of New South Wales, Sydney 2052 Australia,
\texttt{plaice@cse.unsw.edu.au}}
}
\date{}
\maketitle
\begin{abstract}
In this paper we describe the internal structure of the Arabic script
package for the \OMEGA{} typesetting system, as well as the techniques
and tools used for its development. This package allows typesetting
using regular \LaTeX{} styles, in all Arabic alphabet languages:
Arabic, Berber, Farsi, Urdu, Pashto, Sindhi, Uighur, etc.

We also give a description of the character codes added to Unicode, to
obtain the Unicode++ encoding, used by the \OMEGA{} system for
typesetting purposes.
\end{abstract}

\section{Overview of the \OMEGA{} Arabic Script Package}

Typesetting with \OMEGA{} is a process similar to typesetting with
\TeX: the user prepares a ``source'' file, containing the text of
\hisher{} document and a certain number of macro-commands for
attribute changes of the text (font characteristics, language, case,
etc.), references to figures (included in graphical format files on
disk) and other material included in or accompanying the text.

Once this source file prepared, \OMEGA{} is launched: it reads the
file, expands the commands and typesets the text accordingly. To
perform this task, \OMEGA{} loads and executes several \OTP{}s
(\OMEGA{} Translation Processes), which take care of low level
properties of the document (contextual analysis of the script, case
switching according to script and language, etc.). It also uses
different fonts, most of which are \emph{virtual}, in the sense that
they themselves call other fonts. On a higher level, such a document
uses \LaTeX{} packages, some of them modified to take advantage of the
additional features of \OMEGA{} vs.\ \TeX.

The leading idea of the \OMEGA{} Arabic Script Package (as of any
\OMEGA{} language package) is that the low level properties of the
script have to be separated from higher level typesetting
commands. For example, contextual analysis of the Arabic script has to
be completely independent of the \LaTeX{} command level, so that one
can use Arabic text in any context (inside a table or a formula, or
deeply nested inside several \LaTeX{} environments and commands, etc.)
and under any circumstances, as in the following example, which has been
typeset with ordinary \LaTeX{} environments and macros:
 
{%\pardir TRT\textdir TRT
\pushocplist\ArabicOCP\fontfamily{omarb}\selectfont
\begin{center}\begin{tabular}{|c|c|}\hline
{\textdir TRT HayA"t} & {\textdir TRT mayyit}\\\hline
{\mathdir TLT$\displaystyle\int_{\text{\textdir TRT Sif<>r}}^{\hbox dir TRT{\textdir TRT ghyr maH<>duUd}}f(x)\,dx$} & {\textdir TRT 'aanA}\\\hline
\end{tabular}\end{center}
\popocplist}
 
There are two key aspects to Arabic script typesetting,
unfortunately of unequal complexity: the first one is contextual
analysis, that is the fact that Arabic letters change shape according
to their position in a word, or according to the fact that they are
part of an abbreviation, etc. This aspect can be handled easily and
efficiently by \OTP{}s.  The second aspect is more global: it is the
fact that Arabic script is written from right to left.

Two methods can be applied: the first one is to change the default
direction of the whole document. This method is extremely efficient
when the document is entirely in Arabic, or if left-to-right text
excerpts are exceptional. Being global, this method applies also to
page-level typesetting methods, such as the order of columns in a
multicolumn environment, etc. Of course, mathematical formulas are not
affected by this global direction change.

The second method is to keep left-to-right as default direction and to
temporarily switch to right-to-left for every Arabic script
sentence. This can be practical for a document where Arabic excerpts
are exceptional.

\section{Parts of the \OMEGA{} Arabic Script Package}

This package consists of the following elements:

\begin{enumerate}
\item{}\tolerance=3000 The \texttt{OmegaSerifArabic} PostScript fonts:
files \texttt{omsea1.pfb}, \texttt{omsea2.pfb}, \texttt{omsea3.pfb}
and the corresponding AFM files. A Sans-serif font
(\texttt{Omega\-Sans\-Arabic}), as well as additional styles of the
Serif font are under development.
\item{}\tolerance=3000 The virtual font \texttt{omrl}: files
\texttt{omrl.ovf}, \texttt{omrl.ofm}, \texttt{omsea1.tfm},
\texttt{omsea2.tfm}, \texttt{omsea3.tfm}.
\item{} The configuration file \texttt{omrl.cfg}, which is used by the
PERL utility MakeOVP to create the virtual font out of the AFM files
and other information.
\item{} A certain number of \OTP{}s: 
\begin{enumerate}
\item{} \texttt{7arb2uni.otp}, 7-bit Arabic/Farsi transcription to Unicode;
\item{} \texttt{7ber2uni.otp}, 7-bit Berber transcription to Unicode;
\item{} \texttt{7urd2uni.otp}, 7-bit Urdu transcription to Unicode;
\item{} \texttt{7pas2uni.otp}, 7-bit Afghanistani Pashto transcription to Unicode;
\item{} \texttt{7pap2uni.otp}, 7-bit Pakistani Pashto transcription to Unicode;
\item{} \texttt{7snd2uni.otp}, 7-bit Sindhi transcription to Unicode;
\item{} \texttt{uni2cuni.otp}, contextual analysis, sending Unicode++ to cUnicode++
(`c' for `contextual');
\item{} \texttt{cuni2oar.otp}, cUnicode++ to \texttt{omrl} font.
\end{enumerate}
These \OTP{}s are available in human-readable and compiled binary
format (OCP), the latter being loaded by \OMEGA{} on runtime.
\item{} A \LaTeX{} style (\texttt{arabic.sty}) defining a command that
will activate and deactivate the \OTP{}s.
\item{} Documentation and test files (\texttt{testarab.tex},
\texttt{testsind.tex}).
\end{enumerate}

\section{Installation of the \OMEGA{} Arabic Script Package}

To use the \OMEGA{} Arabic Script Package you must have \OMEGA{}
version 1.45 or higher installed on your machine. Place OFM, OVF, TFM
and OCP files where the system expects to find them (if in doubt,
consult the \texttt{texmf.conf} file). Keep the \texttt{arabic.sty}
file somewhere where it can be found by \OMEGA{}. Finally add the
following few lines to the \texttt{psfonts.map} configuration file of
\texttt{odvips}:
\begin{verbatim}
omsea1 OmegaSerifArabicOne </foo/omsea1.pfb
omsea2 OmegaSerifArabicTwo </foo/omsea2.pfb
omsea3 OmegaSerifArabicThree </foo/omsea3.pfb
\end{verbatim}
where \texttt{/foo} stands for the absolute path of the directory
containing the PFB files.

This is all you need to do: you can start already by launching
\OMEGA{} on files \texttt{testarab.tex} and \texttt{testsind.tex}.

In the following sections we will describe the use of the package,
from the end users' point of view.  We will assume that the user is
familiar with the \TeX{} typesetting system and the \LaTeX{} macro
package.

\section{Basic Macros}

Before starting a new document one has to choose if the ``background
language'' is going to be an Arabic alphabet language, in other terms,
if we expect pages and columns to be typeset from right to left, and
the whole global page design to be right-to-left oriented.

If this the case, then the macro
\verb=\GlobalArabic[=\texttt{\textit{language}}\verb=]= has to be used
in the document header, where the optional argument
\texttt{\textit{language}} is one of the following: \texttt{arabic}
(by default), \texttt{farsi}, \texttt{urdu}, \texttt{pashto},
\texttt{sindhi}, \texttt{custom}.

This macro will switch the global typesetting direction of the
document to right-to-left and will launch the \OTP s necessary for the
language chosen.

Inside the document, independently of the choice of background
language, one can use \LaTeX{} environments \texttt{arabic},
\texttt{berber}, \texttt{farsi}, \texttt{urdu}, \texttt{pashto},
\texttt{pashtop}, \texttt{sindhi} to switch to the corresponding
language, and \texttt{latin} or \texttt{greek} to switch to a Latin
alphabet language or some flavour of Greek. It should be noted that
these macros are only temporary and will be adapted to a more global
language-switching scheme, currently being elaborated by the \LaTeX3
and \OMEGA{} working groups.

\section{Input of Arabic Alphabet Text}

\subsection{You Have an Arabic Keyboard}

If you have an Arabic Keyboard, containing sufficiently many keys for
the language you want to typeset (for example, with a standard Arabic
keyboard one can perhaps typeset Farsi, possibly Urdu but not Pashto
and certainly not Sindhi), you need to configure \OMEGA{} to your
\emph{input encoding}, by providing the appropriate input \OTP{} by
use of the \verb=\ArabicInputEncoding= macro, which you have to place
in the header of your document. We have already written such \OTP{}s
for three input encodings: Macintosh Arabic (\texttt{applemac},
covering Arabic, Farsi, Urdu), Windows Arabic (\texttt{1256}, covering
Arabic and Farsi), MS-DOS Arabic ASMO (\texttt{708}, covering Arabic
only) and ISO~8859-6 (\texttt{iso8859-6}, covering only Arabic).  If
your equipment is not in this list, go to section~\ref{writingOTPs} to
see how to write your own \OTP{}s.

\subsection{You Don't Have an Arabic Keyboard}

In that case you can use a Latin transcription: we have prepared ASCII
Latin transcriptions for each of the main Arabic-alphabet languages:
Arabic, Berber, Farsi, Urdu, Pashto (Afghanistani and Pakistani),
Sindhi. Here they are:

\subsubsection{Arabic/Farsi Transcription}\label{arabtrans}

\begin{center}
\begin{tabular}{|c|c||c|c||c|c||c|c||c|c||c|c|}
\tl{A}{p}{z}{`}{m}{I}
\tl{'a}{j}{zh}{gh}{n}{y}
\tl{'i}{H}{s}{f}{'n}{'y}
\tl{'A}{kh}{sh}{q}{-h}{||}
\tl{"A}{ch}{S}{v}{"h}{E}
\tl{b}{d}{D}{k}{e}{}
\tl{t}{dh}{T}{g}{U}{LLah}
\tl{th}{r}{Z}{l}{'u}{SLh}
\hline
\end{tabular}
\end{center}

\noindent
Remarks:

\begin{enumerate}
\item The \emph{tah marbutah} \shortarab{"h} can be written in two
ways: \texttt{"h} or \texttt{"t}.
\item The \emph{waw} \shortarab{w} can be written in two ways:
\texttt{w} or \texttt{U}.
\item The hyphen in front of the transcription of \shortarab{h} is
only necessary to prevent confusion between cases such as \texttt{kh}
(\shortarab{kh}) and \texttt{k-h} (\begin{arab}k-h\end{arab}).  We
suggest you use it all the time.
\item VERY IMPORTANT: the duplication of consonants (\emph{shaddah})
is obtained by writing the consonants twice. So for example,
\texttt{Dmm"h} will produce \begin{arab}Dmm"h\end{arab} and not
\begin{arab}Dm-m"h\end{arab}; to obtain the latter, type \texttt{Dm-m"h},
as for example in the word \begin{arab}t-tHrrk\end{arab}, which
presents both cases, and which is typed \texttt{t-tHrrk}.
\end{enumerate}

Vowels and other diacritics are obtained in the following way: (they
are typed after the consonant to which they belong)

\begin{center}
\begin{tabular}{|l|c|}
\hline fatha & \texttt{a}\\
\hline kasra & \texttt{i}\\
\hline damma & \texttt{u}\\
\hline soukoun & \texttt{<>}\\
\hline vertical fatha & \texttt{a|}\\
\hline fathatan & \texttt{aN}\\
\hline kasratan & \texttt{iN}\\
\hline dammatan & \texttt{uN}\\\hline
\end{tabular}\end{center}

Example: it is a trivial task now to welcome you to this system of
Arabic input, by saying

\begin{verbatim}
\begin{arab}
\Huge
'aahlAaN wa sahlAaN!
\end{arab}
\end{verbatim}

{\pardir TRT\textdir TRT
\begin{center}
\begin{arab}
\Huge
'aahlAaN wa sahlAaN!
\end{arab}
\end{center}
}

\noindent
Example of vowelized Arabic:\\[8pt]

{\pardir TRT\textdir TRT
\begin{quote}
\pushocplist\ArabicOCP\fontfamily{omarb}\selectfont\LARGE li'aannahaA
"Al<>'Ana laA tufakkiru fiI naf<>sihaA, walakinnahaA tufakkiru fiI
'aakhaway<>haA wafiI "Al<>khaTari "AlladhiI laHiqahumaA.  \popocplist
\end{quote}
}

\noindent transcribed: 

\begin{quote}
\texttt{li'aannahaA "Al<>'Ana laA tufakkiru fiI naf<>sihaA,\\
walakinnahaA tufakkiru fiI 'aakhaway<>haA\\
wafiI "Al<>khaTari "AlladhiI laHiqahumaA.}
\end{quote}

\subsubsection{Urdu Transcription}

The Urdu transcription is similar to the Arabic/Farsi one described
above, with a few additional characters, and one exception.

The additional characters are \shortarab{'t}, \shortarab{'d} and
\shortarab{'r}, transcribed by \texttt{'t}, \texttt{'d},
\texttt{'r}. The exception concerns the two different uses of the
\emph{hah} glyph \shortarab{h}. In Urdu it can be used as the second
part of a digraph, such as for example
\begin{smallurdu}jh\end{smallurdu}, in which case we transcribe it as
\texttt{-h}; it can also be the standard consonant \emph{hah}, in
which case we transcribe it by \texttt{x}.  Notice the four forms of
the latter in Urdu: \begin{smallurdu}x-x-x x\end{smallurdu}, while in
Arabic the same letter is written \begin{smallarab}h-h-h
h\end{smallarab}.

\noindent
Example: 

{\pardir TRT\textdir TRT
\begin{quote}
\pushocplist\UrduOCP\fontfamily{omarb}\selectfont xmArI Trf prAnE
zmAnE my'n dstUr t-hA kx Agr ksI shkhS kU kAghdh pr kchh lk-hA xUA grA
p'rA ml jAtA tU Uh As przE kU AHtyAT sE A't-hA kr kxy'n rk-h dytA yA
pAnI mI'n bxA dytA tAkx lk-hE xU'yE HrUf kI bE HrmtI nx xU.
\popocplist
\end{quote}}

\noindent
transcribed: 

\begin{quote}
\texttt{xmArI Trf prAnE zmAnE my'n dstUr t-hA kx Agr ksI\\
shkhS kU kAghdh pr kchh lk-hA xUA grA p'rA ml jAtA tU Uh\\
As przE kU AHtyAT sE A't-hA kr kxy'n rk-h dytA yA pAnI mI'n\\
bxA dytA tAkx lk-hE xU'yE HrUf kI bE HrmtI nx xU.}
\end{quote}

\subsubsection{Pashto Transcription}

The Pashto transcription is similar to the Arabic/Farsi one described
beyond, with a few additional characters and some exceptions. We are
proposing two \OTP{}s, using the same transcription, for the two
flavors of written Pashto: Afganistani and Pakistani.

1. Afghanistani Pashto

\begin{center}
\begin{tabular}{|c|c||c|c||c|c||c|c||c|c||c|c|}
\patl{A}{'z}{'r}{D}{g}{-y}
\patl{b}{c}{z}{T}{l}{e}
\patl{p}{H}{zh}{Z}{m}{ay}
\patl{t}{kh}{'g}{`}{n}{ey}
\patl{'t}{d}{s}{gh}{'n}{||}
\patl{'s}{'d}{sh}{f}{w}{}
\patl{j}{dh}{x}{q}{-h}{LLah}
\patl{ch}{r}{S}{k}{L}{SLh}
\hline
\end{tabular}
\end{center}

2. Pakistani Pashto

\begin{center}
\begin{tabular}{|c|c||c|c||c|c||c|c||c|c||c|c|}
\paptl{A}{'z}{'r}{D}{g}{-y}
\paptl{b}{c}{z}{T}{l}{e}
\paptl{p}{H}{zh}{Z}{m}{ay}
\paptl{t}{kh}{'g}{`}{n}{ey}
\paptl{'t}{d}{s}{gh}{'n}{||}
\paptl{'s}{'d}{sh}{f}{w}{}
\paptl{j}{dh}{x}{q}{-h}{LLah}
\paptl{ch}{r}{S}{k}{L}{SLh}
\hline
\end{tabular}
\end{center}

Nevertheless, one should be aware that an automatic transcription from
one glyph set to the other is not possible because, for example, a
letter such as \begin{pashto}x\end{pashto} is not used in Pakistani
Pashto and can be replaced by \begin{pashto}kh\end{pashto} or
\begin{pashto}sh\end{pashto}, depending on its pronunciation in a given word.

\noindent
Example of Afghanistani Pashto: 

{\pardir TRT\textdir TRT
\begin{quote}
\pushocplist\AfghaPashtoOCP\fontfamily{omarb}\selectfont k-h ghUA'ray
chh d`ql yh zyAn AUDrrpUh shay dA U mnI || chh `ql hghh.  qUtUnh
p-hs'rI kxI wzhnI zhh zhUnde wlA'rdI. zhUndUn p-h`ml AUArAd-h
wlA'rdI. ghUxtnh lUArAd-h d-hre-yshr ft ASl AUAsAs dI. cUmrh chh `ql
zyAtebz hghUmrh ArAd-h D`yf-h kebzI.  \popocplist
\end{quote}}

\noindent
and the same in Pakistani Pashto: 

{\pardir TRT\textdir TRT
\begin{quote}
\pushocplist\PakiPashtoOCP\fontfamily{omarb}\selectfont k-h ghUA'ray
chh d`ql yh zyAn AUDrrpUh shay dA U mnI || chh `ql hghh.  qUtUnh
p-hs'rI kxI wzhnI zhh zhUnde wlA'rdI. zhUndUn p-h`ml AUArAd-h
wlA'rdI. ghUxtnh lUArAd-h d-hre-yshr ft ASl AUAsAs dI. cUmrh chh `ql
zyAtebz hghUmrh ArAd-h D`yf-h kebzI.  \popocplist
\end{quote}}

\noindent
transcribed:

\begin{quote}
\texttt{k-h ghUA'ray chh d`ql yh zyAn AUDrrpUh shay dA\\
U mnI || chh `ql hghh. qUtUnh p-hs'rI kxI wzhnI zhh zhUnde\\
wlA'rdI. zhUndUn p-h`ml AUArAd-h wlA'rdI. ghUxtnh lUArAd-h\\
d-hreyshr ft ASl AUAsAs dI. cUmrh chh `ql zyAtebz hghUmrh\\
ArAd-h D`yf-h kebzI.}
\end{quote}

A variant form \shortpashto{^^^^015d} of \shortpashto{'g} is provided
in the font. The user can change the \OTP{}s (see~\ref{writingOTPs})
so that the former is used instead of the latter.

\subsubsection{Sindhi Transcription}

Sindhi being a language with many more letters than Arabic, and using
Arabic letters in a way quite different than Arabic, it is not
surprising that the Sindhi transcription is fundamentally different
from the Arabic, Farsi, Urdu and Pashto ones. As a matter of fact we
have tried to use as few non-alphabetic characters as possible,
following a more-or-less rational scheme loosely based on the
correspondence between Sindhi written in Arabic and in Devanagari
script and the standard transcription of the latter.  Since shadda is
much more seldom in Sindhi than in Arabic, the ``double consonant $=$
consonant $+$ shadda'' convention is not valid in this transcription;
instead we propose a transcription of the shadda diacritic:
\texttt{+}.

\begin{center}
\begin{tabular}{|c|c||c|c||c|c||c|c||c|c||c|c|}
\stl{A}{p}{dh}{sh}{kh}{y}
\stl{'A}{ph}{.=d}{.s}{.n}{'y}
\stl{b}{j}{.d}{.z}{g}{meN}
\stl{=b}{=j}{.dh}{..t}{=g}{||eN}
\stl{bh}{=n}{=z}{..z}{l}{||}
\stl{t}{c}{r}{`}{m}{}
\stl{th}{ch}{.r}{gh}{n}{}
\stl{.t}{.h}{z}{f}{'n}{}
\stl{.th}{=kh}{zh}{q}{U}{LLah}
\stl{=s}{d}{s}{k}{-h}{SLh}
\hline
\end{tabular}
\end{center}

\noindent
Remarks:

\begin{enumerate}
\item The transcription \texttt{/} is used for constructions such as
\begin{sindhi}b/\end{sindhi} (\texttt{b/}),
\begin{sindhi}t/\end{sindhi} (\texttt{t/}), \begin{sindhi}kh/\end{sindhi}
(\texttt{kh/}), etc.
\item The \emph{waw} \shortarab{w} can be written in two ways:
\texttt{w} or \texttt{U}.
\end{enumerate}
  
\noindent
Example:

{\pardir TRT\textdir TRT
\begin{quote}
\pushocplist\SindhiOCP\fontfamily{omarb}\selectfont tn-hn kry AsAn khy
pn-hnjy =z-hnn khy sjA=g rkh'nU pUndU ||eN pn-hnjy jdUj-hd meN .=dA-hp
pydA kr'ny. AhU b/ m`lUm kr'nU pUndU t/ sndh meN hr 'A'yy wqt chA chA
thy r-hyU 'Ahy ||eN dshmn AsAn jy ||eN AsAn jy jdUj-hd jy khlAf k-h.rA
k-h.rA g-hA.t g-h.ry r-hyU 'Ahy.  \popocplist
\end{quote}}

\noindent
transcribed:

\begin{quote}
\texttt{tn-hn kry AsAn khy pn-hnjy =z-hnn khy sjA=g rkh'nU\\
pUndU ||eN pn-hnjy jdUj-hd meN .=dA-hp pydA kr'ny. AhU b/\\
m`lUm kr'nU pUndU t/ sndh meN hr 'A'yy wqt chA chA thy r-hyU\\
'Ahy ||eN dshmn AsAn jy ||eN AsAn jy jdUj-hd jy khlAf k-h.rA\\
k-h.rA g-hA.t g-h.ry r-hyU 'Ahy.}
\end{quote}

\subsubsection{Berber Transcription}

The Berber transcription is different from the previous ones because
it is based on a tri-alphabetic system (Tifinagh, Latin and Arabic
alphabets).\footnote{The reader can find more information in \emph{Un
syst^^^^00e8me \TeX{} berb^^^^00e8re}, ^^^^00c9tudes et Documents
Berb^^^^00e8res, 11 (1994), La bo^^^^00eete ^^^^00e0
Documents/^^^^00c9disud, Paris (France).} The goal of this
transcription is to enable output in the three alphabets, out of the
same code. In particular, since Latin alphabet has upper and lower
case, it should be possible to distinguish these (and of course ignore
the distinction when typesetting in Arabic or Tifinagh). In the table
below, all transcribed letters are in lowercase ASCII, but can very
well be written also in uppercase, producing the same result:
\texttt{Tifinagh}, \texttt{tifinagh} or \texttt{TIFINAGH} will all
three produce \begin{arab}tyfynAgh\end{arab}.

\begin{center}
\begin{tabular}{|c|c|c|c||c|c|c|c||c|c|c|c|}\hline
Tr. & Lat. & Ar. & Tif. & Tr. & Lat. & Ar. & Tif. & Tr. & Lat. & Ar. & Tif. \\\hline
\ttl{a}{.h}{.s}
\ttl{b}{i}{t}
\ttl{c}{j}{.t}
\ttl{gh}{k}{u}
\ttl{d}{l}{x}
\ttl{.d}{m}{z}
\ttl{.e}{n}{.z}
\ttl{f}{.n}{.i}
\ttl{g}{q}{--}
\ttl{.g}{r}{}
\ttl{h}{s}{}
\hline
\end{tabular}
\end{center}

\noindent
Remarks:

\begin{enumerate}
\item Letter \shortarab{U} can also be transcribed \texttt{w}.
\item Letter \shortarab{I} can also be transcribed \texttt{y}.
\item The stroke \shortberber{^^^^063f} is not to be confused with the
graphical connecting stroke \emph{keshideh}.  It is placed between
words and plays a grammatical role.
\item Duplication of consonants (\emph{shaddah}) again is transcribed
by writing the corresponding consonant twice.
\end{enumerate}

\noindent
Example: 

{\pardir TRT\textdir TRT
\begin{quote}
\pushocplist\ArabicBerberOCP\fontfamily{omarb}\selectfont Tifinagh,
d--tira timezwura n .imazighen.  Llant di tmurt--nnegh dat tira n
ta.erabt d--tla.tinit. Nnulfant--edd dat .imir n ugellid
Masinisen. .Imazighen n .imir--en, ttarun--tent ghefi.zra, degg
.ifran, ghef .igduren, maca tiggti ghef i.zekwan~: ttarun fell--asen
.isem n umettin, d wi--t--ilan, d wayen yexdem di tudert--is akken ur
t ttettun .ina.tfaren.  \popocplist
\end{quote}}

\noindent
transcribed:

\begin{quote}\small
\texttt{Tifinagh, d--tira timezwura n .imazighen.\\ 
Llant di tmurt--nnegh dat tira n ta.erabt d--tla.tinit.\\
Nnulfant--edd dat .imir n ugellid Masinisen. .Imazighen n\\
.imir--en, ttarun--tent ghefi.zra, degg .ifran, ghef .igduren,\\
maca tiggti ghef i.zekwan~: ttarun fell--asen .isem n umettin,\\ 
d wi--t--ilan, d wayen yexdem di tudert--is akken ur t ttettun\\
.ina.tfaren.}
\end{quote}

\noindent
The same code will produce the following output in the Tifinagh alphabet:

\begin{quote}
\begin{tifinagh}Tifinagh, d--tira timezwura n .imazighen. 
Llant di tmurt--nnegh dat tira n ta.erabt d--tla.tinit.  Nnulfant--edd
dat .imir n ugellid Masinisen. .Imazighen n .imir--en, ttarun--tent
ghefi.zra, degg .ifran, ghef .igduren, maca tiggti ghef i.zekwan~:
ttarun fell--asen .isem n umettin, d wi--t--ilan, d wayen yexdem di
tudert--is akken ur t ttettun .ina.tfaren.\end{tifinagh}
\end{quote}

\noindent
and the following one in the Latin alphabet:

\begin{quote}
\begin{latberber}Tifinagh, d--tira timezwura n .imazighen. 
Llant di tmurt--nnegh dat tira n ta.erabt d--tla.tinit.  Nnulfant--edd
dat .imir n ugellid Masinisen. .Imazighen n .imir--en, ttarun--tent
ghefi.zra, degg .ifran, ghef .igduren, maca tiggti ghef i.zekwan~:
ttarun fell--asen .isem n umettin, d wi--t--ilan, d wayen yexdem di
tudert--is akken ur t ttettun .ina.tfaren.\end{latberber}
\end{quote}

\section{Writing Your Own Transcription}\label{writingOTPs}

We have developed and presented in this paper a certain number of
Arabic alphabet language transcriptions for two reasons: first, to
show the possibilities and power of \OMEGA, and second, to give a
starting point for the user to create \hisher{} own transcriptions.

The process of creating a new transcription is twofold: the first
part, which can be very difficult and painful, consists of finding the
combination of letters, digits and ASCII symbols which will transcribe
each character; the second one, which is straightforward (modulo some
precautions) is to implement this in \OMEGA{} by writing the
appropriate \OTP.

\subsection{A Good Transcription: Is it Possible?}

There are (at least) two goals for a good transcription:

\begin{enumerate}
\item \emph{It has to be readable and easily memorizable}. In other
words, \texttt{AHmd} is better than \texttt{'.hmd}, for denoting
\begin{smallarab}AHmd\end{smallarab} : although an apostrophe can be
considered a logical choice for transcribing an alif and the period in
front of the h may denote that it is an emphatic `h' sound, taking an
A for alif and a capital H for the emphatic h is more readable; also
using rules such as ``uppercase ASCII characters transcribe emphatic
letters'' is an easy way to memorize the transcriptions of
\shortarab{H}, \shortarab{T}, \shortarab{D}, \shortarab{S},
\shortarab{Z}.
\item \emph{It has to be complete and avoid ambiguities}. Of course
all letters of the target language have to be covered, but having many
letters to transcribe leads sometimes to ambiguities: for example
taking \texttt{h} for \shortarab{h}, \texttt{k} for \shortarab{k} and
\texttt{kh} for \shortarab{kh} are perfectly logical choices;
nevertheless there is a hitch: when you need to transcribe
\begin{smallarab}k-h\end{smallarab} you are tempted to write simply
\texttt{kh} and this will of course produce \shortarab{kh}
instead. The solution we have given to this problem is to type a
hyphen between the letters which are not considered as a `digraph',
but this is only a compromise solution: the user must constantly be
aware of this problem, and this is hardly the case when you are
concentrated in your text...
\end{enumerate}

It is clear that these two goals are contradictory: an accurate and
unambiguous transcription has to be complicated and will be difficult
to read and memorize; a friendly and easily readable transcription
will be full of ambiguities.

An additional problem when making a transcription is to choose between
\emph{(etymo)logical}, \emph{phonetic} and \emph{graphical}
representations of characters. A typical example is the standard
\OMEGA{} transcription of Greek: \texttt{w} is chosen for letter
\shortgreek{w}, this is a purely \emph{graphical} choice: the `w'
looks like an omega, but has absolutely no other relation with,
neither historical nor phonetic (the letter omega represents the sound
`o' in modern Greek); \texttt{b} is chosen for letter \shortgreek{b},
this is an \emph{etymological} choice: the Latin `B' derives from the
ancient Greek `B', otherwise \shortgreek{b} looks quite different than
`b' and is pronounced `v' in modern Greek; finally, \texttt{x} is a
\emph{phonetic} transcription of letter \shortgreek{x}; clearly they
do not bear any resemblance, and historically it is not clear (at
least to the author) why `x' should be derived from \shortgreek{x}
(their positions in the alphabet is quite different as well, and this
is an argument speaking against an etymological relation between the
letters).

The reader may object that this distinction between etymological,
phonetic and graphical representations is not relevant for Arabic
alphabet transcriptions; actually this is only partly true: take for
example \texttt{bh} for \shortsindhi{bh}, this is an
\emph{etymological} transcription in the sense that it reflects the
standard transcription of the Indic alphabet letter which corresponds
to that Sindhi letter. Also \texttt{`} for ayn is in some sense a
\emph{graphical} representation: it has been chosen because it
resembles the IPA transcription of the ayn, which is ^^^^0295. For the
same reason, \texttt{'} has been chosen for the hamza with carrier (in
\shortarab{'a}, \shortarab{'u}, etc.): the hamza's IPA transcription
is ^^^^0294.

We hope to have convinced the reader that the making of a
transcription is a difficult task, needing a lot of thought,
compromises and tests. Once again, we would like to emphasize the fact
that our transcriptions are only temptative proposals and should not
be taken as standards of any kind; after all the power of \OMEGA\ is
that it can work with any input transcription without affecting
further processing, be it contextual analysis, diacritic placement or
esthetical ligaturing.

In the next section we will see how to implement a new transcription
or change an existing one by writing/modifying an \OTP\ file. But
first some generalities on the \OTP{}s used by the Arabic \OMEGA\
system.

\subsubsection{The \OTP{}s used by the Arabic \OMEGA{} system}

When \OMEGA{} reads the text flow it places letters, digits and
punctuation (whatever is not an escape or special character) into a
buffer. When it encounters a special character it stops buffering and
executes one after the other all currently active \OTP{}s on the
buffer.  In theory, \OTP{}s could be used to arbitrarily send
character combinations to other combinations: one could very well
imagine an \OTP{} sending the string "Yannis" to "John" and "John" to
"Yannis", or "Microsoft Word" to
"^^^^02a7\kern-1pt^^^^04a9^^^^03be^^^^0468^^^^029a"; nevertheless,
such an \OTP{} would not be of general use...

Our development has mainly been focused in building \OTP{}s in
accordance to the following scheme:
$$
\boxed{\text{Input text}} \xrightarrow{\text{\texttt{foo2uni}}} \boxed{\text{Unicode++}} 
\xrightarrow{\text{\texttt{uni2foo}}} \boxed{\text{DVI output}} 
$$
where \texttt{foo2uni} sends text encoded in an arbitrary encoding
into Unicode++ (Unicode++ is Unicode extended for the needs of
\OMEGA{} and typography), and \texttt{uni2foo} converts
Unicode++-encoded data into the encoding of the output font. By this
method we are able to keep completely separate input encoding and font
encoding.

In the case of Arabic things are slightly more complicated since an
additional step is needed: contextual analysis. This is where our
scheme proves to be extremely efficient: by performing contextual
analysis on the level of Unicode++, and hence obtaining the following
new scheme:
$$
\boxed{\text{Input text}} \xrightarrow{\text{\texttt{foo2uni}}} \boxed{\text{Unicode++}} 
\xrightarrow{\text{\texttt{uni2cuni}}} \boxed{\text{cUnicode++}} 
\xrightarrow{\text{\texttt{cuni2oar}}} \boxed{\text{DVI output}}
$$
we still remain independent of both the input and the font
encoding. This means that if we need to adapt \OMEGA{} to a new Arabic
encoding we only need to indicate which code position corresponds to
which Unicode character, and, on the other hand, if we want to adapt a
new font to \OMEGA, we only need to indicate which font position
corresponds to which contextual form of which character, in
cUnicode++.

In the next section we will partly describe the syntax of \OTP{} files
by giving examples of \texttt{foo2uni} cases.

\subsection{Implementing a Transcription}

The \OTP{} files we will need for input encoding $\to$ Unicode++
transformations use only part of the syntax of \OTP{}
files.\footnote{The \texttt{uni2cuni} \OTP{} file already needs more
complicated constructions.}  Such an \OTP{} file is of the following
form:

\begin{verbatim}
input: 1;
output: 2;
expressions:
...
...
\end{verbatim}

\noindent where \texttt{input: 1; output: 2;} means that input is
8-bit while output is 16-bit, and \texttt{...} are lines of the
following form:

\begin{verbatim}
before => after ;
\end{verbatim}

\noindent where \texttt{before} is an expression before the
transformation, and \texttt{after} after it.  For example,

\begin{verbatim}
`a' => "o" ;
\end{verbatim}

\noindent will transform all `a's in the file into `o's. 

How do we describe characters and strings? On the left side of
\texttt{=>} we can only put separate characters: they can be written
either as ``grave accent+ASCII character+apostrophe'' or as
\texttt{@"XYZT} where \texttt{XYZT} are hexadecimal digits: in this
case we are not restricted to ASCII characters. The latter syntax can
also be used on the right side. For example,

\begin{verbatim}
`i'`j' => @"0133 ;
@"008E => @"00E9 ;
\end{verbatim}

\noindent will send the string `ij' to the Unicode++ character
representing the Dutch ^^^^0133 ligature, and the 8-bit code 8E (a
Macintosh `e' with acute accent) to the Unicode++ character 00E9
(which is the Unicode `e' with acute accent).

On the right side of \texttt{=>} we can also write complete strings,
possibly containing \OMEGA{} commands, which will be forwarded to the
next \OTP{} or to the typesetting engine of \OMEGA. For example,

\begin{verbatim}
`~' => "\penalty10000" ;
\end{verbatim}

\noindent sends the tilde character to the \TeX{} command of infinite
penalty.\footnote{By this we obtain the same result as in \TeX{} but
without turning tilde into an active character, a fact that \TeX{}
users will surely appreciate.} We can also use ranges on the left
side: for example, \texttt{`a'-`k'} means ``all characters between a
and k''.

By using parentheses and the vertical bar on the left side, we obtain
the Boolean `or' operator:

\begin{verbatim}
(`E'|`e') => ;
\end{verbatim}

\noindent for example, will send both uppercase and lowercase letters
`e' to nothing (a transformation which would leave Perec's book
\emph{La disparition} unchanged\footnote{Although there are rumors
that there is a single `e' in that book... The authors were not able
to find it yet.}).

This operator becomes even more useful by the fact that we can use on
the right side the exact character matched on the left side: the
commands \verb=\1=, \verb=\2=, ... , \verb=\9= used on the right side
stand for the first, second, ..., ninth character matched on the left
side. For example:

\begin{verbatim}
`c'(`a'|`e'|`i'|`o'|`u')`t' => "m" \1 "p" ;
\end{verbatim}

\noindent will send cat, cet, cit, cot, cut respectively to map, mep,
mip, mop, mup.

We can go even further: \OTP{} syntax allows us to add or substract a
fixed offset to the characters matched on the left side. For example:

\begin{verbatim}
`a'-`z' => #(\1 - @"0020) ;
\end{verbatim}

\noindent will substract 20 from the code position of the character
found on the left side. The characters on the left side being
precisely lowercase letters, this offset will turn them into uppercase
ones.

\subsubsection{Examples}

The beginning of the \OTP{} \texttt{7arb2uni}, used to send the ASCII
transcription of Arabic to Unicode++, described in~\ref{arabtrans}, to
Unicode++, looks like this:

\begin{verbatim}
input: 1;
output: 2;

expressions:

`L'`L'`a'`h'      => @"FDF2 ;
`S'`L'`h'         => @"FDFA ;

`|'`|'`|'`|'      => @"0621 @"0651 ;
`|'`|'            => @"0621 ;
`z'`h'`z'`h'      => @"0698 @"0651 ;
`z'`h'            => @"0698 ;
`z'`z'            => @"0632 @"0651 ;
`z'               => @"0632 ;
`y'`y'            => @"064A @"0651 ;
`y'               => @"064A ;
`v'`v'            => @"06A4 @"0651 ;
`v'               => @"06A4 ;
`u'`N'            => @"064C ;
`u'               => @"064F ;
\end{verbatim}

Let us take a closer look at these lines. The left sides
\texttt{`L'`L'`a'`h'} and \texttt{`S'`L'`h'} correspond to the
(religious) ligatures \shortarab{LLah} and \shortarab{SLh} which
appear in the \emph{Arabic Presentation Forms} part of Unicode, that's
why the code positions we send them to are so high. The line
\texttt{`|'`|'`|'`|'} corresponds to a double hamza; according to our
transcription rules, by writing a letter's transcription twice without
intermediate hyphen, we get the letter followed by a \emph{shaddah}
diacritic. On the right side of \texttt{`|'`|'`|'`|'} you see two
codes: 0621 stands for the stand-alone hamza in Unicode++, and 0651
for the \emph{shaddah}. The next line will send \texttt{||} to the
stand-alone hamza.

WARNING: the order of these lines is very important: transformations
are matched in the order lines are read. By putting the double hamza
before the single one, \OMEGA{} will first look for a double hamza and
\emph{only if it does not find any} will then proceed to transforming
a single one.

For the same reason digraphs such as \texttt{zh} must appear before
their first letter in the \OTP{} file (and trigraphs before the
starting digraph, etc.). That's why the order of lines starting with a
`z' is `zhzh', `zh', `zz', `z'.%
\footnote{There is a simple way of avoiding ordering problems: after
having written this part of the \OTP{} file, run a line sorting
program on it so that lines are sorted in \emph{inverse}
lexicographical order. This will automatically place trigraphs before
digraphs before singletons, etc.}

Our sample file ends like this:

\begin{verbatim}
`h'`h'    => #(@"0647) #(@"0651) ;
`h'       => #(@"0647) ;
`-'`-'`-' => @"2014;
`-'       =>  ;
.         => #(\1) ;
\end{verbatim}

This means that after having entered all digraphs using `h' as second
character, we enter the stand-alone `h', first as a double letter, and
secondly as a single letter. Finally we send the triple hyphen to an
m-dash `---' and the single hyphen to nothing: its purpose is to
prevent combinations of letters to be interpreted as digraphs: when
reading \texttt{k-h}, \OMEGA{} will not match it with \texttt{kh}: it
will first match \texttt{k} with letter kaf, then send the hyphen to
the vacuum of non-existence and when arriving to the \texttt{h} the
\texttt{k} will already be matched so that it is too late to construct
a \texttt{kh} digraph.

The period at the beginning of the last line is part of the \OTP{}
syntax we have not seen yet: it means `any character'.  Since this is
the last line of the file, we can interpret it rather like `any still
not matched character'. This line simply sends any character not yet
matched to itself.

\subsection{Wrapping it up}

Once the \OTP{} file has been written or modified, one only needs to
compile it (by using the \texttt{otp2ocp} utility) and place it where
\OMEGA{} expects to find it. On the \LaTeX{} command level, \OTP{}s
are loaded via the \verb=\ocp= command, in a way similar to fonts: to
load the file \texttt{foo2uni} one will write

\begin{verbatim}
\ocp\FooUni=foo2uni
\end{verbatim}

Of course this is preferably done inside a \LaTeX{} package or style
file: the final user should not need to deal with or understand this
kind of code. Once the \OTP{}s are loaded they are combined into
\emph{lists}. In this way we can push or pop simultaneously \OTP{}s
on/from a stack. This is useful because a language switch usually
requires several \OTP{}s to be changed at once. To define \OTP{} lists
we use the following syntax:

\begin{verbatim}
\ocplist\ArabicOCP=
\addbeforeocplist 100 \ArabUni
\addbeforeocplist 200 \UniCUni
\addbeforeocplist 300 \CUniArab
\nullocplist
\end{verbatim}

The numbers (100, 200, 300) allow us to introduce additional \OTP{}s,
if necessary, between the already defined ones.  Finally, to
activate/desactivate an \OTP{} list, we use the commands
\verb=\pushocplist= (followed by the name of the \OTP{} list) and
\verb=\popocplist=. To take a real life example,

\begin{verbatim}
\ocp\ArabUni=7arb2uni
\ocp\UniCUni=uni2cuni
\ocp\CUniArab=cuni2oar
\ocplist\ArabicOCP=
\addbeforeocplist 100 \ArabUni
\addbeforeocplist 200 \UniCUni
\addbeforeocplist 300 \CUniArab
\nullocplist
\pushocplist\ArabicOCP
\end{verbatim}

\noindent is sufficient to load all \OTP{}s necessary for typesetting
in the Arabic language.

\section{Availability and Further Information}

The \OMEGA{} system is entirely in the public domain. It can be
obtained from any CTAN server. The latest information on \OMEGA{} and
its Arabic system can be found on the \OMEGA{} server:

$$\text{\texttt{http://www.ens.fr/omega}}$$

\noindent courtesy of the ^^^^00c9cole Normale Sup^^^^00e9rieure de
Paris.

\section{Samples}

Starting from next page, a few samples (Arabic, Berber, Sindhi). For
these examples we have switched the background language to Arabic, so
that even page numbers are in Arabic.

\newpage
\makeatletter
\def\@evenfoot{\reset@font\hfil{\textdir TLT\thepage}\hfil}
\let\@oddfoot=\@evenfoot
\makeatother
\pagedir TRT
\bodydir TRT
\pardir TRT
\textdir TRT
\def\latinit#1{{\fontfamily{omlgc}\selectfont\pushocplist\BasicLatinOCP%
\textdir TLT #1\popocplist}}
\def\rmdefault{omarb}
\fontfamily{omarb}\selectfont
\pushocplist\ArabicOCP


\subsection{'aTfAl AlghAb"t}

kAn l'aHd AlmlUk AlqdmA|| 'akht t`ysh m`h fI qSrh, b`d 'an mAt-t
zUjt-h, wtrkt lh mn Al'awlAd thlAth"t: 'amyryn w'amyr"t. wqd AzdAd Hbb
Almlk l'awlAd-h, b`d wfA"t wAldt-hm Almlk"t, w'aHbbhm HbbA kthyrA;
ly`wwDhm mA fqdUh mn `Tf 'ammhm wHbbhA lhm, wtfkyr hA fyhm; fkAn ys'al
`nhm kllmA HDr, wyfkkr fyhm kllmA dkhl, wywSI bhm kllmA khrj, wyTlbhm
kllmA jls ltnAwl T`Am Al'ifTAr 'aU AlghdA|| 'aU AlshshAI 'aU Al`shA||.

mHHm"t 'akhyhA l'awlAd-h, wSmm-mt fymA bynhA wbyn nfs-hA 'an t`ml srrA
kll wsyl"t m-mkn"t l'ib`Ad-hm `n 'abyhm wAlttkhllS mnhm.

wfI yUm mn Al'ayyAm kAn Al'amyrAn yl`bAn m` 'akht-hmA Al'amyr"t fI
HdA'yq AlqSr b`d khrUj Almlk, fshUUqt-hm `mmt-hm wHbb-bt 'ilyhm
Aldhdh-hAb m`hA 'ilI AlghAb"t l-ll`Ab fyhA, w-w`dt-hm 'an tryhm
'ashyA|| jmyl"t w'al`AbA ldhydh"t sArr"t tHt Al'ashjAr hnAk.

fSddq Al'amyrAn wAl'amyr"t mA qAlt-h `mmt-hm, wlm y`rfUA mA tkhfyh
`nhm mn Alshshrr, wdhhbUA m`hA l-ll`b wAlrryAD"t fI 'alghAb"t,
wmshAhd"t Al'ashA|| Aljmyl"t fyhA, wr'uy"t Al'al`Ab Alghryb"t tHt
'ashjArhA.

wqd sh`r Al'aTfAl bsrUr kthyr `nd mAkhrjUA m` `mmt-hm lhdhh
AlrrHl"t. w'akhdhUA ymshUn m`hA fI AlghAb"t HttI wSlUA 'ilI wsThA,
f'aHssUA bAltt`b Alshshdyd, wThrt `lAmAt-h fI mshyt-hm, w`lI wjUh-hm
b`d hdhh AlrrHl"t AlTTUyl"t Almt`b"t AlltI lm yjrrbUhA mn qbl.  UlmAA
sh`rt Al`mm"t bshdd"t t`bhm, qAlt lhm: nAmUA hnA tHt hdhh Alshshjr"t
HttI tHDr AlHUryyAt ltl`b 'amAmkm 'al`AbA lm trUhA, wstjdUn fI
mshAhdt-hA kll ldhdh"t wsrUr.  \popocplist

\pushocplist\ArabicBerberOCP

\subsection{Allal i useqdc n y.drisn \OMEGA\ d-tamazight}

%\noindent{\leaders\hrule height0.5pt\hfill}
%\par

A dd nessken s wayes yif useqdec n \OMEGA\ i tira s tutlayt tamazight,
ama s tifinagh, ama s isekkilen ila.taniyen. Newwi-dd tamazight am,
tutlayt yeddren (yettwarun s tifinagh tiynayin)~: izmer umdan ad
iseddu yall tighura n usuddes n tira, i waraten ussnanen, itekniken
negh i wid n tsikkla, am wid ssexdamen i usemsaru n tfransist.

\OMEGA, d ameslay n usmihel i usuddes n tira. Am-wakken ne.zra, d ayen
i dd yttakken i.zubba.z war taggara i useqdec d usihrew, maca issefk
ad ilmed uqeddac kra tussniwin. Nunz-as, ta.z.zayt n ulmad-a, nezmer a
tt nsifess s useqdec n inagrawen n urmas n tira, isegh.zanen n usmihcl
n waraten, ittwassnen a.tas (wid ittnuzun, srayn ghef umdan, wid
izemren ad ssxedmen tazmert tasemsirawt n kra inagrawen imehlanen am
wid n \latinit{Macintosh, Windows, Unix}.

Tan.da tamzwarut n \OMEGA{} --- ghas tin ay ittalasen ism n \OMEGA{}
---, us tli ageruedm i uqeddac. Am gg imeslayen n usmihel akk, ad yaru
wmdan ahil, deffir, a t issefsu akken a t yessughal s anqal n
tmacint. Di \OMEGA{}, ahil d ara n u.dris (n.t.te.dn ghur-s kra n
tsun.diwin i usbuni d tghessa tame.z.zult).  Asefsu, d aselkem n wahil
\OMEGA~; angal n tmacint ara dd iffghen, d win, i d aglam n usebter ay
ittusuddsen, Iqqim-dd imir-n usemsaru.

Akala-ya, yezmer a t yaf yefregh win inumen iseqdac n i.drisen ghef
\latinit{Macintosh, Windows}, d wiyi.d, i degg a.dris a dd iffegh di
tsemsarut akken yella gg uqdil [Anagraw-a yettwassnen s yism-is
imiwzil, s tglizit \latinit{<<~wysiwyg~>>}, ycsseghla.d kra~: a.dris
ara dd yesuffegh uqeddac, ad yili ghas s tseddi umi yessawe.d ugdil~;
asgmu.d ara dd tsuffegh tsemsarut, yesmer ad yili yuser kra.]

Iwakken ad yeqqim useqdec sray f umdan, yezmer ad yessexdem asegh.zan
ittwassnen d allaeln i urmas. Taghessa tame.z.zult n wara (ighfawen,
tifula, tiseddarin, tizmilin tinaddayin, timitar tinmudag, asmel n
tektabin), a tt yessyghal si tbunit n usegh.zan-nni gher tsun.diwin n
\OMEGA. Imir, \OMEGA, ad issefsu angal-nni a dd yessuffegh a.dris
yuq.zen taghessa tame.z.zult tamezwarut, maca tira-ines ad ilint
ulaghent ugar.  \popocplist

\pushocplist\SindhiOCP
\subsection{ktyn kr mU.ryA j.=d-hn}

%\noindent{\leaders\hrule height0.5pt\hfill}
%\par

tn-hn kry AsAn khy pn-hnjy =z-hnn khy sjA=g rkh'nU pUndU ||eN pn-hnjy
jdUj-hd meN .=dA-hp pydA kr'ny. AhU b/ m`lUm kr'nU pUndU t/ sndh meN
hr 'A'yy wqt chA chA thy r-hyU 'Ahy ||eN dshmn AsAn jy ||eN AsAn jy
jdUj-hd jy khlAf k-h.rA k-h.rA g-hA.t g-h.ry r-hyU 'Ahy.


AsAn khy AhA b/ =khbr hj'n g-hrjy t/ AsAn jy 'As pAs ||eN ysgrdA'yy
meN chA chA thy r-hyU 'Ahy. hndstAn meN chA thy r-hyU 'Ahy, AfghAnstAn
meN chA thy r-hyU 'Ahy. `rAq ||eN AyrAn meN chA thy r-hyU 'A-hy ||eN
'AmrykA ||eN sUUyt yUnyn chA chA sUcy r-hyA 'Ahn. j.=d-hn AsAn s=jy
dnyA jy syAst ty ||eN s=jy dnyA jy jdUj-hd ty ||eN s=jy dnyA jy
tndylyn ty n..zr rkhndAsUn ||eN An-hn tbdylyn jy A=srn khy pn-hnjy
mlk, qUm ||eN `rAm ty pUndy .=dsndAsUn t/ An-hn tbdylyn mAn k-h.rA
mnfy ||eN k-h.rA m=sbt A=sr 'Ahn. t.=d-hn 'yy AsAn pn-hnjy jdUj-hd khy
b-htr b/ kry sg-hndAsUn t/ chU.tkAry UArU .hl b/ =gUly UyndAsUn.


r=gU =gAl-hyUn kndy ||eN n`rn h'nndy AsAn jy qUm chAlyh sAl py.rA'yUn
||eN `=zAb bhU=gyA 'Ahn ||eN An-hn n`rn AsAn jy qUm lA||i Udhyk
py.rA'yUn ||eN `=zAb nAzl kyA 'Ahn . jyk.=d-hn AsAn meN A=j qUm jy
Amyd pydA thy 'Ahy t/ AhA AsAn jy `ml ||eN AsAn jy by lU=s jdUj-hd jy
kry pydA thy 'Ahy ||eN mA'y-hU AsAn .=dAn-hn wAj-hA'yy rhyA 'Ahn. t/
AsAn 'yy 'AhyUn jyky kj-h n/ kj-h kndAsUn. pr AsAn khy .=ds'nU 'Ahy t/
dnyA jy Andr chA thy rhyU {}'Ahy ||eN AsAn jU dshmn ky'yn .hAltn khy
pn-hnjn mfAdn meN ktb 'A'n'n jy kUshsh kry rhyU 'Ahy, An-hy||a jy
lA||i .zrUry 'Ahy t/ AsAn pA'n meN .=dAhp pydA kryUn ||eN pA'n meN
=jA'n jU hk Usy` =khzAnU pydA kryUn. ||eN AsAn mUjUd-h .sUrt .hAl khy
smj-h'n lA||i rUzmrh jy my.dyA ||eN dnyA jy Andr thynd.r kArUA'yyn ty
g-hry n..zr rkhUn t/ dshmn ||eN jAr.hyt psnd qUtUn ||eN AsAn ty qAb.z
qUtUn dnyA jy Andr thynd.r tbdylyn khy sndn .hq meN ||eN sndn mfAdn jy
.hq meN, sndh ty jAr.hyt qA'nm rkh'n jy .hq meN, sndh khy mstql qb.zy
meN kr'n jy .hq meN ky'yn ktb 'A'ny r-hyUn 'Ahn.

\popocplist
\end{document} 
