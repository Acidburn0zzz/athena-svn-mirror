\documentclass{amsart}
\usepackage{amsrefs}

%\usepackage{dual} % not released yet, needs more work
%\setlength{\dualindent}{-2em}

\newenvironment{dual}{%
  \par\medskip
  \trivlist\item[]%
}{%
  \endtrivlist
}

\newcommand{\backup}{%
  \vspace*{-\baselineskip}\vspace*{-\medskipamount}\nopagebreak
}

\newtheorem{thm}{Theorem}[section]

\begin{document}
\title{Citation tests}
\author{Michael Downes}

   The following examples are derived from
   \emph{Homology manifold bordism} by Heather Johnston and Andrew
   Ranicki (Trans.\ Amer.\ Math.\ Soc.\ \textbf{352} no 11 (2000), PII: S
   0002-9947(00)02630-1).

\bigskip \noindent \rule{\columnwidth}{0.5pt}\par

\setcounter{section}{3}
\begin{dual}
The results of Johnston \cite{Jo} on homology
manifolds are extended here. It is not
possible to investigate transversality by
geometric methods---as in \cite{Jo} we employ
bordism and surgery instead.
\end{dual}

%Kirby and Siebenmann \cite{KS} (III,\S 1),
\begin{dual}
The proof of transversality is indirect,
relying heavily on surgery theory\mdash see
Kirby and Siebenmann \cite{KS}*{III, \S 1},
Marin \cite{M} and Quinn \cite{Q3}. We shall
use the formulation in terms of topological
block bundles of Rourke and Sanderson
\cite{RS}.
\end{dual}

\begin{dual}
$Q$ is a codimension $q$ subspace by Theorem
4.9 of Rourke and Sanderson \cite{RS}.
(Hughes, Taylor and Williams \cite{HTW}
obtained a topological regular neighborhood
theorem for arbitrary submanifolds \dots.)
\end{dual}

%Wall \cite{Wa} (Chapter 11) obtained a
\begin{dual}
Wall \cite{Wa}*{Chapter 11} obtained a
codimension $q$ splitting obstruction \dots.
\end{dual}

\begin{dual}
\dots\ following the work of Cohen \cite{Co}
on $PL$ manifold transversality.
\end{dual}

\begin{dual}
In this case each inverse image is
automatically a $PL$ submanifold of
codimension $\sigma$ (Cohen \cite{Co}), so
there is no need to use $s$-cobordisms.
\end{dual}

%Quinn (\cite{Q2}, 1.1) proved that \dots
\begin{dual}
Quinn \cite{Q2}*{1.1} proved that \dots
\end{dual}

\begin{dual}\backup
\begin{thm}[The additive structure of
  homology manifold bordism, Johnston
  \cite{Jo}]
\dots
\end{thm}
\end{dual}

\begin{dual}
For $m\geq 5$ the Novikov-Wall surgery theory
for topological manifolds gives an exact
sequence (Wall \cite{Wa}*{Chapter 10}.
\end{dual}

\begin{dual}
The surgery theory of topological manifolds
was extended to homology manifolds in Quinn
\cites{Q1,Q2} and Bryant, Ferry, Mio
and Weinberger \cite{BFMW}.
\end{dual}

\begin{dual}
The 4-periodic obstruction is equivalent to
an $m$-dimensional homology manifold, by
\cite{BFMW}.
\end{dual}

\begin{dual}
Thus, the surgery exact sequence of
\cite{BFMW} does not follow Wall \cite{Wa} in
relating homology manifold structures and
normal invariants.
\end{dual}

\begin{dual}
\dots\ the canonical $TOP$ reduction
(\cite{FP}) of the Spivak normal fibration of
$M$ \dots
\end{dual}

\begin{dual}\backup
\begin{thm}[Johnston \cite{Jo}]
\dots
\end{thm}
\end{dual}

\begin{dual}
Actually \cite{Jo}*{(5.2)} is for $m\geq 7$,
but we can improve to $m\geq 6$ by a slight
variation of the proof as described below.
\end{dual}

\begin{dual}
(This type of surgery on a Poincar\'e space
is in the tradition of Lowell Jones
\cite{Jn}.)
\end{dual}

\bibliographystyle{amsxport}
\bibliography{jr}

\end{document}
