%%
%% This is file `dbold.tex',
%% generated with the docstrip utility.
%%
%% The original source files were:
%%
%% dinbrief.dtx  (with options: `dinbriefold')
%% 
%% =======================================================================
%% 
%% Copyright (C) 1993, 96, 97 by University of Karlsruhe (Computing Center).
%% Copyright (C) 1998, 2000   by University of Karlsruhe (Computing Center)
%%                            and Richard Gussmann.
%% All rights reserved.
%% For additional copyright information see further down in this file.
%% 
%% This file is part of the DINBRIEF package
%% -----------------------------------------------------------------------
%% 
%% It may be distributed under the terms of the LaTeX Project Public
%% License (LPPL), as described in lppl.txt in the base LaTeX distribution.
%% Either version 1.1 or, at your option, any later version.
%% 
%% The latest version of this license is in
%% 
%%         http://www.latex-project.org/lppl.txt
%% 
%% LPPL Version 1.1 or later is part of all distributions of LaTeX
%% version 1999/06/01 or later.
%% 
%% 
%% For error reports in case of UNCHANGED versions see readme files.
%% 
%% Please do not request updates from us directly.  Distribution is
%% done through Mail-Servers, TeX organizations and others.
%% 
%% If you receive only some of these files from someone, complain!
%% 
%%
%% \CharacterTable
%%  {Upper-case    \A\B\C\D\E\F\G\H\I\J\K\L\M\N\O\P\Q\R\S\T\U\V\W\X\Y\Z
%%   Lower-case    \a\b\c\d\e\f\g\h\i\j\k\l\m\n\o\p\q\r\s\t\u\v\w\x\y\z
%%   Digits        \0\1\2\3\4\5\6\7\8\9
%%   Exclamation   \!     Double quote  \"     Hash (number) \#
%%   Dollar        \$     Percent       \%     Ampersand     \&
%%   Acute accent  \'     Left paren    \(     Right paren   \)
%%   Asterisk      \*     Plus          \+     Comma         \,
%%   Minus         \-     Point         \.     Solidus       \/
%%   Colon         \:     Semicolon     \;     Less than     \<
%%   Equals        \=     Greater than  \>     Question mark \?
%%   Commercial at \@     Left bracket  \[     Backslash     \\
%%   Right bracket \]     Circumflex    \^     Underscore    \_
%%   Grave accent  \`     Left brace    \{     Vertical bar  \|
%%   Right brace   \}     Tilde         \~}
%%
 %
 % example letter, example receiver addresses
 %
 % Most of the following code has taken from the dinbrief released
 % at May 11th, 1992. This code is originated by Rainer Sengerling.
 %
 % The following example contains all commands of the old dinbrief.sty
 % API (application programming interface). We recommend to use the
 % new dinbrief API which is much more powerfull. The outcome may be
 % quiet unsatisfied if you mix both APIs. We had made a few changes
 % to adapt the file to \LaTeXe.
 %
\expandafter\ifx\csname documentclass\endcsname\relax
    \documentstyle[german]{dinbrief}
    \typeout{Using the command \string\documentstyle.}
  \else
    \documentclass[10pt]{dinbrief}
    \usepackage{german}
    \typeout{Using the command \string\documentclass.}
  \fi

\Etihoehe{41mm}%
\Etirand{46.3mm}%
\Etizahl{6}%
\makelabels
\spare{1}
 %\pagestyle{empty}
\begin{document}
 %
 % Musterbrief -- Anfang
 % In der vorliegenden Version (beachte Auskommentierungen mit %)
 % liefert er das Anwendungsbeispiel 4 von DIN 5008
 %
\begin{letter}{%
Stadt G"ottingen\\
Stadtbauamt\\
Postfach 28 17\par
3400 G"ottingen
}
\Postvermerk{Einschreiben}
\Behandlungsvermerk{E\ i\ l\ t}
\Datum{G"ottingen, 29.04.86}
\Absender{Klaus Waldmann\\B"urgerstra"se 135\\3400 G"ottingen\\::Tel.\
(05 51) 9 34 56}
\Retourlabel
\Fenster
\Retouradresse{K. Waldmann $\cdot$ B"urgerstr.\ 135 $\cdot$
               3400 G"ottingen}
\Betreff{Mein Bauvorhaben Waldrebenweg 9}
\signature{K. Waldmann}
\opening{Sehr geehrte Damen und Herren,}
den Antrag zur Genehmigung f"ur den Neubau eines Einfamilienhauses auf
meinem Grund\-st"uck G"ottingen, Waldrebenweg 9, hat der Architekt,
Herr Dipl.-Ing.\ G. Schwarz, mit allen erforderlichen Unterlagen am
03.01.86 eingereicht. Die Baugenehmigung habe ich bis heute nicht
erhaltern.

\Einrueckung{%Der folgende Text wird eingerueckt
Da ich die Finanzierung des Bauvorhabens ohne die Baugenehmigung nicht
endg"ultig kl"aren kann und der Beginn der Bauarbeiten nicht verz"ogert
werden soll, bitte ich dringend, das Genehmigungsverfahren zu
beschleunigen.
}  % Ende der Einrueckung

Gleichzeitig bitte ich um Auskunft, ob damit zu rechnen ist, da"s der
nur geschotterte Waldrebenweg in absehbarer Zeit zu einer "`Stra"se im
vorl"aufigen Ausbau"' umgestaltet wird.

\anlagenrechts
\Anlagen{2 Anlagen}
\Verteiler{Verteiler:\\Landratsamt}

\closing{Mit freundlichen Gr"u"sen}
\ps{Also bis bald!}
\end{letter}

 % normgerecht geschriebene Adressen

\begin{letter}{Frau\\Erika Werner\\bei M"uller\\Bahnhofstr.\ 4 -- 6

8580 Bayreuth}
\end{letter}

\begin{letter}{Frau\\ Annemarie Hartmann\\Vogelsangstr.\ 17 II\par
2870 Delmenhorst}
\Postvermerk{Briefdrucksache}
\end{letter}

\begin{letter}{02694/73\\Herrn Gutsverwalter\\Dipl.-Ldw.\ Otto Winter\\
Hauptstr.\ 3

8221 Alm Post Neukirchen}
\Postvermerk{Nicht nachsenden}
\end{letter}

\begin{letter}{Eheleute\\Erika und Hans M"uller\\Hochstr.\ 4

4709 Bergkamen}
\Postvermerk{Warensendung}
\end{letter}

\begin{letter}{Herrn Staatsanwalt\\Dr.\ Ernst Meyer und Frau\\
Talblick 2

8200 Rosenheim}
\Postvermerk{Eilzustellung -- auch nachts}
\end{letter}

\begin{letter}{Herrn Rechtsanwalt\\Dr.\ Otto Freiherr von Bergheim\\
Leonhard-Eck-Str.\ 7 W 36

8000 M"unchen 19}
\Postvermerk{Einschreiben -- R"uckschein}
\end{letter}

\begin{letter}{Herrn Direktor\\Dipl.-Kfm.\ Kurt Gr"aser\\Massivbau AG\\
Postfach 21 03 14

5600 Wuppertal 21}
\end{letter}

\begin{letter}{Frau Luise Weber\\Herrn Max Weber\\Rosenstra"se 35

7030 B"oblingen}
\end{letter}

\begin{letter}{Lack- und Farbwerke\\Dr.\ Hans Sendler \& Co.\\
Abt.\ FDM 412/10\\Postfach 80 19 36\par
6230 Frankfurt 80}
\end{letter}

\begin{letter}{Lehmann \& Krause KG\\z.\ H. Herrn E. Winkelmann\\
Gartenhaus III r.\\Johannisberger Str.\ 5 a\par 1000 Berlin 31}
\end{letter}

\begin{letter}{W"aschegro"shandel\\Robert Bergmann\\Venloer
Stra"se 80 -- 82\par 5000 K"oln 30}
\end{letter}

\begin{letter}{Firma\\Otto Pfleiderer\\Braunenweiler\\Hauptstr.\ 5\par
7968 Saulgau 1}
\Postvermerk{Drucksache}
\end{letter}

\begin{letter}{Amtsgericht Leer\\Grundbuchamt\\Postfach 11 24\par
2950 Leer}
\end{letter}

\begin{letter}{Regierungspr"asident\\Dezernat 44.II.2\\
Postfach 59 07\par 4400 M"unster}
\end{letter}

\begin{letter}{Nassauische Heimst"atte GmbH\\
Abt.\ Landestreuhandstelle\\Postfach 10 29 17\par 6000 Frankfurt 1}
\end{letter}

\begin{letter}{Volksbank Friedberg\\Hauptzweigstelle Bad Nauheim\\
Aliceplatz 4\par 6350 Bad Nauheim}
\end{letter}

\begin{letter}{VEB Ph"onix-Apparatewerk\\
Absatzabteilung\\Inselstr.\ 14/20

DDR-7021 Leipzig}
\Postvermerk{Einschreiben}
\end{letter}

\begin{letter}{Mevrouv J. de Vries\\ Poste restante A. Cuypstraat\\
Postbus 99730\\1000 NA AMSTERDAM

NIEDERLANDE}
\end{letter}

\begin{letter}{Monsieur P. Dubois\\Expert en assurances\\
Escalier 3, b\^atiment C\\4, rue Jean Jaur\`es

F-58500 CLAMECY}
\end{letter}

\begin{letter}{Mr.\ W. Smith\\514 Kingsbridge Road\\PURLEY, SURREY\\DE
1\\GROSSBRITANNIEN CRZ 4TH}
\end{letter}

\end{document}
\endinput
%%
%% End of file `dbold.tex'.
