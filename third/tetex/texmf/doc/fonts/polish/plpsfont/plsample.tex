%&mex
%%% File in Polish: PL fonts in Type1 format---presentation. To be processed
%%% with MeX (Polish plain) format
%%% Plik w notacji prefiksowej (/a/c/e/l/n/o/s/x/z/A/C/E/L/N/O/S/X/Z)
%% Polskie fonty PostScript-owe PL -- przyk/lady i u/zycie
%% Autor: Janusz Marian Nowacki, Grudzi/adz, e-mail: j.nowacki@gust.org.pl
%% wersja 2, aktualizacja i opracowanie StaW (maj 2000)

\prefixing
\tracinglostchars=0
\font\cpsmcp=plcsc10 scaled \magstep5
\font\dunh=pldunh10
\font\sst=plbx10
\font\bt=plb10
\font\sc=plr7 at 7pt
\baselineskip 12pt
\spaceskip.333em plus.333em minus.111em
\def\LaTeX{L\raise.42ex\hbox{\sc\kern-.5em A}\kern-.15em\TeX}
\def\LaMeX{L\raise.42ex\hbox{\sc\kern-.4em A}\MeX}
\clubpenalty=10000
\widowpenalty=\clubpenalty
\hyphenpenalty=100

\newdimen\FontDimen
\def\InstallFont#1#2#3{%
  \def#1{\afterassignment#2\FontDimen=}
     \def#2{\font\currfont #3 at\FontDimen\relax\currfont
     \advance\FontDimen by.2\FontDimen % opcja
     \baselineskip=\FontDimen % opcja
     \spaceskip.333em plus.333em minus.111em % opcja
 }}
\InstallFont{\ag}{\xag}{plbx10}
\InstallFont{\fwg}{\xfwg}{plssdc10}
\InstallFont{\ap}{\xap}{plr10}

\def\tagchar#1{\setbox0
  \hbox{\char#1}\ifdim\wd0>0pt\box0\hskip.3em plus.3em minus.1em\fi}

\def\tagfont#1#2{%\hangafter=1\hang
  \par\noindent
  \font\tf=#1 %at 10pt
 {\tf\count0=0\loop\tagchar{\count0}%
    \ifnum\count0<255\advance\count0by1\repeat} % zamiast 127
  \par\smallskip
}

\def\textsample#1#2{
 {\font\ts=#1
 \ts\par
 \baselineskip=1.2em
 #2\par
 \vskip 8pt}
 \rm
}
%%%%%%%%%%%%%%%%%%%%%%%%%%%%%%%%%%%%%%%%%%%%%%%%%%%%%%%%%%%%%%%%%%%%%%
\null
\vskip 1in
\centerline{\ag24pt Janusz Marian Nowacki}
\medskip
\centerline{\ag12pt J.Nowacki@gust.org.pl}
\vfil
{\parindent=100pt \obeylines \parskip=0pt \cpsmcp \baselineskip=1.05em
Fonty
PostScriptowe
Computer
Modern
Wersja
Polska
PL
}
\vfil
\centerline{\fwg24pt BACHO\TeX'97}
\medskip
\centerline{\ag12pt Wersja zmieniona -- BACHO\TeX'2000}
\eject
%
\headline{\ap8pt\ifodd\pageno Fonty PS Computer Modern -- wersja polska PL
\hfil \the\pageno
\else \the\pageno \hfil Fonty PS Computer Modern -- wersja polska PL \fi}

\def\chapter#1{{\fwg18pt\noindent #1\endgraf\bigskip}}

\null
\bigskip
\chapter{Computer Modern PostScript Fonts}
\bigskip

{\leftskip.3in\hrule height0pt \dunh \baselineskip 18pt \parindent2em
The Computer Modern typefaces were created in the spirit of the typeface
Monotype Modern 8A by Professor Donald E. Knuth of Stanford University. The
Computer Modern family includes a~large set of scientific and mathematical
figures and symbols and is widely used with implementations of the \TeX\
typesetting system. ``Computer Modern Typefaces'', Volume E of ``Computers 
and Typesetting'' (D.~E.~Knuth, Addison Wesley), is the definitive
source for the Computer Modern faces. ``The \TeX book'' (D.~E.~Knuth,
Addison Wesley), Appendix~F, is also a~good reference.

This package contains 75 faces from the Computer Modern family in Adobe
PostScript Type~1 form. These fonts can be used with all PostScript printers
and Adobe Type Manager.

Bogus/law Jackowski and Marek Ry/cko prepared (in METAFONT) a~Polish version
of the Computer Modern fonts, namely, PL fonts, to be used with the Polish
version of the plain format, namely, \MeX. The fonts presented herewith are
essentialy PL fonts in PostScript Type~1 format.

}

%\vfil\eject

\bigskip \bigskip \bigskip
\chapter{Fonty postscriptowe Computer Modern}
\vskip -.15in
\chapter{Wersja polska w~uk/ladzie PL}

{\leftskip.3in\hrule height0pt \dunh \baselineskip 18pt \parindent2em
Rodzina pism Computer Modern zosta/la stworzona w~duchu czcionek Monotype
Modern~8A przez profesora Donalda E.~Knutha z~Uniwersytetu Stanforda w~USA.
Rodzina Computer Modern zawiera du/zy zestaw znak/ow tekstowych, naukowych
i~matematycznych, i~jest szeroko wykorzystywana do sk/ladania tekst/ow za
pomoc/a systemu \TeX. Podstawowym /xr/od/lem wiedzy o~fontach z~rodziny
Computer Modern jest podr/ecznik ,,Computer Modern Typefaces'',
(D.~E.~Knuth, Addison Wesley, seria ,,Computers and Typesetting'', tom~E).
Godny polecenia w~tym wzgl/edzie jest r/ownie/z podr/ecznik ,,The \TeX
book'' (D.~E.~Knuth, Addison Wesley, dodatek~F).

Dost/epny w~archiwach \TeX-owych pakiet postscriptowej wersji czcionek
rodziny Computer Modern zawiera 75 kroj/ow, kt/ore mog/a by/c u/zywane na
wszystkich drukarkach i~na/swietlarkach postscriptowych oraz w~systemie
Windows z~zastosowaniem programu Adobe Type Manager.

Polskim odpowiednikiem bitmapowych font/ow Computer Modern s/a kroje PL
przygotowane wraz z~polskim formatem \MeX\ przez Bogus/lawa Jackowskiego
i~Marka Ry/cko.  Prezentowane w~niniejszej publikacji fonty postscriptowe s/a
odpowiednikiem font/ow PL (w~oryginale przygotowanych za pomoc/a METAFONT-a).

}

\vfil\eject

\chapter{Wst/ep}

{\baselineskip 13pt \parindent 1.5em \parskip3pt
\noindent Pewnego pi/eknego dnia, w~1997 roku, na li/scie
dyskusyjnej GUST-u pojawi/la si/e mi/la, przynajmniej dla mnie, wiadomo/s/c
-- w~archiwach \TeX-owych s/a dost/epne PostScript-owe wersje font/ow CM
udost/epnione przez American Mathematical Society. Rado/s/c moja by/laby
jednak wi/eksza gdyby czcionkami tymi by/lo mo/zna sk/lada/c teksty w~j/ezyku
Mickiewicza i~S/lowackiego, nie wspominaj/ac ju/z o~Reju.

Kto/s powinien wi/ec ,,zlokalizowa/c'' fonty AMS-CM. Zdecydowanie najlepiej
zrobi/lby to Bogus/law Jackowski, gdy/z nikt inny nie posiada wi/ekszej
wiedzy o~fontach PL ni/z ich autor. Jest jednak jeden problem -- doba ma
tylko 24~godziny.

Przyst/api/lem wi/ec w~miar/e moich mo/zliwo/sci i~wiedzy do ich spolszczenia
kieruj/ac si/e nast/epuj/acymi za/lo/zeniami:

\item{1.} Postscriptowe fonty PL powinny by/c dok/ladnymi, lub prawie
dok/ladnymi, odpowiednikami bitmapowych font/ow PL*.PK. U/zytkownik nie musi
uczy/c si/e nowej obs/lugi font/ow w~plikach \TeX-owych. Przy wykorzystaniu
font/ow postscriptowych bez problem/ow powinny da/c si/e obejrze/c
i~wydrukowa/c dokumenty wcze/sniej z/lo/zone za pomoc/a font/ow PK.

\item{2.} W~sk/lad pakietu font/ow postscriptowych PL wchodz/a
pliki *.PFB zawieraj/ace opis kszta/ltu poszczeg/olnych znak/ow. Dla
kompletno/sci do/laczono pliki *.TFM, standardowe pliki metryczne,
co zapewnia dok/ladnie takie samo pozycjonowanie znak/ow i~tekstu jak
z~wykorzystaniem font/ow PK.

\item{3.} Fonty pierwotnie by/ly przeznaczone do stosowania w~\TeX-u.
W~pierwszej wersji (z 1997~r.) nie zawiera/ly np. plik/ow *.PFM przydatnych 
do bezpo/sredniego z~nich korzystania w~systemach Windows. Fonty posiada/ly
taki sam uk/lad znak/ow (encoding) jak fonty PL*.PK, dzi/eki czemu dla \TeX-a
nie trzeba by/lo stosowa/c plik/ow przekodowa/n (*.ENC).  Poniewa/z nie by/lo
mo/zliwe bezproblemowe wykorzystywanie ich w~systemach Windows, w~niniejszej
dystrybucji zmieniono wewn/etrzny uk/lad font/ow, do/l/aczono pliki *.PFM, 
*.ENC, a~tak/ze, dla kompletno/sci, pliki *.AFM ({\bf UWAGA:} plik/ow tych 
nie nale/zy poddawa/c konwersji do *.TFM za pomoc/a programu {\tt afmtotfm} 
lub podobnych; nale/zy u/zywa/c wy/l/acznie dost/epnych plik/ow *.TFM).

\item{4.} Zawarto/s/c oryginalnych plik/ow CM*.PFB podda/lem najpierw
jedynie przekodowaniu do standardu PL, uzupe/lniaj/ac fonty o~znaki 
stosowane w~j/ezyku polskim. W~obecnej dystrybucji uk/lad znak/ow
zosta/l dostosowany do potrzeb u/zytkownik/ow system/ow Windows; dla
\TeX-a dodano pliki przekodowania (*.ENC).

\item{5.} Podczas projektowania znak/ow nieistniej/acych w~zestawie AMS
kierowa/lem si/e cennymi regu/lami i~uwagami zawartymi w~pliku {\tt
mexinfo.tex} pakietu \MeX. Fragmenty tego pliku wykorzysta/lem r/ownie/z
w~przyk/ladach zastosowania poszczeg/olnych kroj/ow.

\item{6.} Nie zajmuj/e si/e programowaniem METAFONT-owym, nie mog/lem wi/ec
wykorzysta/c, co do piksela, zawartych tam informacji o~polskich znakach (np.
poziome offsety elementu ,,acute''). Kierowa/lem si/e wi/ec w/lasn/a
intuicj/a i~w~tym miejscu b/ed/a zapewne minimalne r/o/znice w~por/ownaniu
z~oryginalnymi fontami PL*.PK.

\item{7.} {\bf Tak jak fonty AMS, r/ownie/z ich polska wersja, s/a dobrem
wsp/olnym naszej \TeX-owej spo/leczno/sci. Przygotowanie pakietu traktuj/e
jako prezent dla cz/lonk/ow i~sympatyk/ow Grupy U/zytkownik/ow Systemu \TeX\
w~Polsce w~pi/at/a rocznic/e powstania organizacji.}

}
\bigskip

\noindent\it Dzi/ekuj/e Bogus/lawowi Jackowskiemu i~Piotrowi Pianowskiemu za
/zyczliw/a pomoc w~zdobywaniu wiedzy o~budowie i~dzia/laniu font/ow, oraz
w~trakcie pracy nad prezentowanymi fontami. Dzi/ekuj/e r/ownie/z Tomkowi
Przechlewskiemu za wst/epne przetestowanie font/ow PL i~zg/loszone uwagi.
Obecna, dostosowana do wykorzystania r/ownie/z w~systemach Windows, wersja 
powsta/la w ramach tzw. JNS TEAM (Bogus/law Jackowski, Janusz M.~Nowacki,
Piotr Strzelczyk).

\vfil\eject

\chapter{Fonty tekstowe, przyk/lady}

\vskip.2in
\parindent 0pt

{\sst PL Bold}: plb10
\tagfont{plb10}{}
\textsample{plb10}{Pakiet \MeX\ powsta/l na bazie poprzedniej polskiej 
adaptacji \TeX-a, nosz/acej nazw/e LeX.}

{\sst PL Bold Extended}: plbx5, plbx6, plbx7, plbx8, plbx9, plbx10, plbx12
\tagfont{plbx10}{}

\bt plbx5:
\textsample{plbx5}{Sk/lada/ly si/e na ni/a pliki makr oraz rodzina font/ow P1.
BJ\&MR, autorzy LeX-a, pracowali nad t/a wersj/a od listopada 1987 do grudnia
1989. \MeX\ jest udoskonalon/a wersj/a LeX-a, wykorzystuj/ac/a nowe 
mo/zliwo/sci \TeX-a 3.x.}


\bt plbx6:
\textsample{plbx6}{Zestaw wzorc/ow, zawieraj/acy regu/ly dzielenia polskich
wyraz/ow wykorzystany w~for\-macie LeX zosta/l zaprojektowany i~utworzony
przez HK w~1987, a~przystosowany do formatu LeX, 
przetestowany i~uzupe/lniony -- przez BJ\&MR w~latach 1987--1989.}


\bt plbx7:
\textsample{plbx7}{Do rozpocz/ecia pracy z~polsk/a wersj/a format/ow PLAIN 
oraz LaTeX\ niezb/edne jest utworzenie za pomoc/a wersji inicjalizacyjnej 
programu \TeX\ (INITEX)}

\bt plbx8:
\textsample{plbx8}{plik/ow mex.fmt i~lamex.fmt (obecnie platex.fmt)
i~umieszczenie ich w~odpowiednim miejscu zale/znym od u/zytego systemu 
operacyjnego i~od implementacji TeX-a. }

\bt plbx9:
\textsample{plbx9}{W plikach FMT zapami/etane s/a w~skondensowanej postaci 
makra oraz regu/ly przenoszenia wyraz/ow.}


\bt plbx10:
\textsample{plbx10}{Wi/ekszo/s/c najnowszych implementacji \TeX-a, w~tym 
em\TeX\ (au\-tor\-stwa Eberharda Mattesa), czy te\TeX{} (au\-tor\-stwa 
Thomasa Essera) ma mo/zliwo/s/c automatycznej zamiany kod/ow znak/ow 
czytanych z~pliku}

\bt plbx12:
\textsample{plbx12}{na kody wewn/etrzne \TeX-a. Pozwala to na wykorzystanie
pe/lni mo/zliwo/sci \TeX-a, przy jednoczesnym uniezale/znieniu si/e od 
sposobu kodowania polskich liter w~danej instalacji komputera.}

{\sst PL Bold Extended Slanted}: plbxsl10
\tagfont{plbxsl10}{}
\textsample{plbxsl10}{W pakiecie MeX znajduj/a si/e pliki wsadowe (ang. batch
files) s/lu/z/ace do generowania plik/ow FMT pod kontrol/a systemu operacyjnego
MS-DOS przy u/zyciu em\TeX-a. W~dystrybucji te\TeX{} format \MeX{} jest
generowany z~odpowiedniego menu programu texconfig.}

\eject

{\sst PL Bold Extended Text Italic}: plbxti10
\tagfont{plbxti10}{}
\textsample{plbxti10}{W standardowej konfiguracji formatu \MeX\ i~PLa\TeX\
polskie znaki diakrytyczne zapisywane s/a jako zwyk/le znaki o~kodach 
zale/znych od standardu stosowanego}

{\sst PL Caps and Small Caps}: plcsc10
\tagfont{plcsc10}{}
\textsample{plcsc10}{na danym komputerze (zwykle Mazovia, cp1250 lub 
iso8859-2). Pozwala
to na bardzo wygodne sk/ladanie tekst/ow z~u/zyciem \TeX-a: polskie litery
widoczne s/a na}

{\sst PL Dunhill}: pldunh10
\tagfont{pldunh10}{}
\textsample{pldunh10}{\baselineskip16pt ekranie w~trakcie przygotowywania
tekstu, ponadto mo/zliwe jest definiowanie polskich komend \TeX-owych.\endgraf}

{\sst PL Funny Roman}: plff10
\tagfont{plff10}{}
\textsample{plff10}{Niezale/znie od standardu kodowania polskich znak/ow 
mo/zliwe jest w~formatach \MeX\ i~PLa\TeX\ w/l/aczenie za pomoc/a komendy
$\backslash$prefixing tzw. notacji prefiksowej.}


{\sst PL Funny Italic}: plfi10
\tagfont{plfi10}{}
\textsample{plfi10}{(Notacj/e bezprefiksow/a, obowi/azuj/aca na pocz/atku
przetwarzania ka/zdego dokumentu, przywraca komenda $\backslash$nonprefixing.)}


{\sst PL Fibonacci}: plfib8
\tagfont{plfib8}{}
\textsample{plfib8}{Poniewa/z znak // zosta/l w~notacji prefiksowej wykorzystany
do oznaczania polskich liter, konieczna jest dodatkowa konwencja notacyjna
pozwalaj/aca}


{\sst PL Italic Typewriter}: plitt10
\tagfont{plitt10}{}
\textsample{plitt10}{uzyska/c w~druku symbol ,,ciach''. W~MeX-u przyj/eto
naturaln/a zasad/e reprezentowania znaku ,,ciach'' poprzez ////.}

%Mo/zna r/ownie/z
%u/zywa/c komendy {\string\normalslash}

\eject

{\sst PL Roman}: plr5, plr6, plr7, plr8, plr9, plr10, plr12, plr17
\tagfont{plr10}{}
\bt plr5:
\textsample{plr5}{Podczas obowi/azywania notacji prefiksowej znak ,,//'' jest
znakiem aktywnym (kategoria 13). /Swiadomie zrezygnowano ze stosowanego 
niekiedy sposobu uzyskiwania znak/ow diakrytycznych poprzez ligatury.}

\bt plr6:
\textsample{plr6}{Nale/zy pami/eta/c, /ze w~czasie pisania do pliku za 
pomoc/a komendy $\backslash$write makra s/a rozwijane. Dotyczy to 
w~szczeg/olno/sci makra ,,//'' i~trzeba na to zwraca/c uwag/e przy pisaniu 
do plik/ow, }


\bt plr7:
\textsample{plr7}{kt/ore maj/a by/c potem czytane przez \TeX-a. 
Format PLa\TeX\ automatycznie przejmuje kontrol/e nad odpowiednim pisaniem 
i~czytaniem standardowych plik/ow pomocniczych (np.~pliku ze spisem tre/sci)}


\bt plr8:
\textsample{plr8}{``Anglosaski'' spos/ob pisania cudzys/low/ow jest niezgodny
z~polskimi zwyczajami. Polski cudzys/l/ow otwieraj/acy znajduje si/e na wysoko/sci
linii bazowej pisma i~ma kszta/lt zbli/zony do dw/och przecink/ow.}

\bt plr9:
\textsample{plr9}{Formaty \MeX{} czy PLa\TeX{} oraz fonty PL zosta/ly tak
skonstruowane, /ze polski cudzys/l/ow otwieraj/acy zapisuje si/e w~pliku 
danych wej/sciowych}


\bt plr10:
\textsample{plr10}{jako dwa przecinki, a~cudzys/l/ow zamykaj/acy (podobnie 
jak w~j/ezyku angielskim) jako dwa apostrofy. To s/a w/la/snie ,,polskie'' 
cudzys/lowy.}


\bt plr12:
\textsample{plr12}{Niekiedy w~j/ezyku polskim u/zywany jest te/z drugi rodzaj
cudzys/low/ow, tzw. cudzys/lowy <<francuskie>>. Lewy francuski cudzys/l/ow
oznaczany}

\bt plr17:
\textsample{plr17}{jest dwoma znakami mniejszo/sci, a~prawy dwoma znakami
wi/ekszo/sci.}

{\sst PL Slanted}: plsl8, plsl9, plsl10, plsl12
\tagfont{plsl8}{}
\bt plsl8:
\textsample{plsl8}{Z punktu widzenia \TeX-a zasady dzielenia wyraz/ow 
zale/z/a zar/owno od j/ezyka, jak te/z od uk/ladu liter w~bie/z/acym foncie. 
W~zwi/azku z~tym }

\vbox{
\bt plsl9:
\textsample{plsl9}{dla \TeX-a istnieje kilka ,,j/ezyk/ow polskich''.
W~konsekwencji je/sli u/zytkownik prze/l/acza si/e na font o~innym uk/ladzie 
(innych kodach) polskich }
}

\bt plsl10:
\textsample{plsl10}{znak/ow diakrytycznych, to towarzyszy/c temu powinno
prze/l/aczenie \TeX-a na inne regu/ly przenoszenia wyraz/ow.}

\bt plsl12:
\textsample{plsl12}{Pakiet \MeX\ przygotowany jest do pracy z~fontami 
w~uk/ladzie PL }

\eject

{\sst PL Slanted Typewriter}: plsltt10
\tagfont{plsltt10}{}
\textsample{plsltt10}{jako podstawowymi oraz dodatkowo z~fontami 
zawieraj/acymi polskie litery w~po/lo/zeniach zgodnych z~kodem Mazovia, 
Latin~2 oraz P1.}

{\sst PL Sans Serif}: plss8, plss9, plss10, plss12, plss17
\tagfont{plss8}{}
\bt plss8:
\textsample{plss8}{U/zytkownik, zale/znie od wielko/sci dost/epnej pami/eci, 
mo/ze wbudowa/c do formatu regu/ly przenoszenia dla innych uk/lad/ow polskich 
font/ow. Wystarczy w~tym celu}


\bt plss9:
\textsample{plss9}{dokona/c prostej modyfikacji pliku konfiguracyjnego
MEXCONF.TEX. Odpowiednie makra formatu \MeX\ lub \LaMeX\ automatycznie
przeczytaj/a odpowiedni/a liczb/e razy}

\bt plss10:
\textsample{plss10}{ten sam plik zawieraj/acy polskie regu/ly przenoszenia
(PLHYPH.TEX), za ka/zdym razem interpretuj/ac go w~inny spos/ob.}


\bt plss12:
\textsample{plss12}{Standardowym uk/ladem polskiego fontu jest uk/lad PL 
(p.~rozdzia/l ,,Fonty PL''). Dopuszczalne jest jednak stosowanie font/ow 
o~innych ni/z PL uk/ladach polskich}

\vbox{
\bt plss17:
\textsample{plss17}{znak/ow, a~nawet mieszanie r/o/znych uk/lad/ow w~obr/ebie jednego
dokumentu.}
}

{\sst PL Sans Serif Bold Extended}: plssbx10,
\tagfont{plssbx10}{}
\textsample{plssbx10}{Fonty PL przeznaczone s/a do sk/ladania polskich lub
polsko\=angielskich tekst/ow przy u/zyciu format/ow \MeX\ i~\LaMeX{} czy 
PLa\TeX.}

{\sst PL Sans Serif Bold Extended Italic}: plssbi10
\tagfont{plssbi10}{}
\textsample{plssbx10}{ Autorzy starali si/e zachowa/c ,,ducha CM-/ow'' przy
projektowaniu polskich znak/ow diakrytycznych, np.~przekre/slenie ma/lej 
litery ,,/l'' zosta/lo nieco tylko powi/ekszone w~stosunku do przekre/slenia 
font/ow rodziny CM}

{\sst PL Sans Serif Demibold Condensed}: plssdc10
\tagfont{plssdc10}{}
\textsample{plssdc10}{umieszczanego w~sk/ladzie komendami \TeX-owymi
$\backslash$l oraz $\backslash$L, chocia/z w~niekt/orych fontach autorzy 
ch/etnie widzieliby przekre/slenie znacznie wyra/xniejsze.}

\eject
{\sst PL Sans Serif Italic}: plssi8, plssi9, plssi10, plssi12, plssi17
\tagfont{plssi10}{}
\bt plssi8:
\textsample{plssi8}{Podobnie podci/ecia wstawiane automatycznie pomi/edzy 
znakami (implicit kerns) by/ly dodawane raczej ostro/znie. np.~zdaniem 
autor/ow podci/ecie mi/edzy}

\bt plssi9:
\textsample{plssi9}{polskim otwieraj/acym cudzys/lowem (,,) a~liter/a ,,W'' 
b/ad/x ,,T'' mog/loby by/c znacznie wi/eksze, gdyby nie to, /ze w~fontach 
CM nie ma /zadnego podci/ecia mi/edzy liter/a ,,A'' a~cudzys/lowem 
zamykaj/acym ('').}

\bt plssi10:
\textsample{plssi10}{ Fonty PL, podobnie jak macierzyste fonty CM, 
projektowane by/ly pod k/atem zastosowa/n profesjonalnych, tzn. urz/adze/n 
o~wi/ekszych rozdzielczo/sciach. Tym niemniej sporo wysi/lku w/lo/zono 
w~prawid/low/a}

\bt plssi12:
\textsample{plssi12}{ dyskretyzacj/e przy ma/lych rozdzielczo/sciach 
(dla potrzeb wydruk/ow testowych na drukarkach mozaikowych 9- i~24-ig/lowych). 
Gdyby mimo} 

\bt plssi17:
\textsample{plssi17}{ to u/zytkownik mia/l inne zdanie ni/z METAFONT na temat
dyskretyzacji,}

{\sst PL Sans Serif Quotation Style}: plssq8
\tagfont{plssq8}{}
\textsample{plssq8}{ w~pakiecie em\TeX\ autorstwa Eberharda Mattesa znajduje 
si/e program o~nazwie PKEDIT pozwalaj/acy r/ecznie korygowa/c wygenerowane 
za pomoc/a METAFONT-a mapy bitowe.}

{\sst PL Sans Serif Quotation Italic Style}: plssqi8
\tagfont{plssqi8}{}

\textsample{plssqi8}{Fonty PL stanowi/a rozszerzenie standardowego zestawu
rodziny Computer Modern (CM) o~polskie znaki diakrytyczne i~bazuj/a na tych
samych plikach parametrycznych co fonty rodziny CM. Zmiany parametr/ow
odbiegaj/ace zbyt daleko od standardu zaproponowanego przez D.~E.~Knutha
}


{\sst PL Typewriter Caps and Small Caps}: pltcsc10
\tagfont{pltcsc10}{}
\textsample{pltcsc10}{ mog/a spowodowa/c nieoczekiwane efekty (do b/l/edu 
w~trakcie oblicze/n w/l/acznie). Przyk/ladowo, za/lo/zone zosta/lo, 
/ze kursywa i~grotesk wykluczaj/a si/e.}


{\sst PL Typewriter Extended}: pltex8, pltex9, pltex10
\tagfont{pltex10}{}
\bt pltex8:
\textsample{pltex8}{Font PLTEX10 to po prostu font CMTEX10 -- zmieniono dla
jednolitosci nazwe; font CMTEX10 odzwierciedla uklad klawiatury
}


\vbox{
\bt pltex9:
\textsample{pltex9}{ na komputerze uzywanym przez profesora Knutha 
(D.~E.~Knuth, ,,Computer Modern Typefaces'', str.~568), tym samym }
}

\vbox{
\bt pltex10:
\textsample{pltex10}{nie powinien raczej zawierac polskich znakow
diakrytycznych.}
}

{\sst PL Text Italic}: plti7, plti8, plti9, plti10, plti12
\tagfont{plti10}{}

\bt plti7:
\textsample{plti7}{ Uk/lad font/ow PL charakteryzuje si/e tym, 
/ze znaki o~kodach mniejszych ni/z 128 s/a identyczne ze znakami rodziny 
CM, natomiast polskim znakom}

\bt plti8:
\textsample{plti8}{ diakrytycznym przypisane zosta/ly kody wi/eksze ni/z 127
(innych cech TeX-a 3.x nie wykorzystuje si/e). Polskie znaki diakrytyczne 
maj/a kody zgodne}

\bt plti9:
\textsample{plti9}{ z~uk/ladem ECM (Extended Computer Modern), natomiast
cudzys/lowy maj/a kody niestandardowe (w~ECM cudzys/lowy maj/a kody mniejsze 
ni/z 127)}


\bt plti10:
\textsample{plti10}{Fonty matematyczne PLMI, PLMIB oraz PLEX nie zawieraj/a
polskich znak/ow diakrytycznych i~nie r/o/zni/a si/e od odpowiednich font/ow 
CM;}

\bt plti12:
\textsample{plti12}{ przedrostki CM zosta/ly zmienione na PL jedynie dla
zachowania jednorodno/sci.}

{\sst PL Typewriter}: pltt8, pltt9, pltt10, pltt12
\tagfont{pltt10}{}
\bt pltt8:
\textsample{pltt8}{Fonty PLSY zosta/ly poszerzone (w~stosunku do CMSY) 
o~znaki ,,wi/eksze-r/owne'' i~,,mniejsze-r/owne'' odpowiadaj/ace polskim 
zwyczajom drukarskim (dolna kreska sko/sna zamiast}


\bt pltt9:
\textsample{pltt9}{ poziomej); na konieczno/s/c takiego rozszerzenia 
zwr/oci/l uwag/e p.~W/lodzimierz J.~Martin; majusku/ly kaligraficzne tych 
font/ow (opisane w~pliku CALU.MF) nie zosta/ly uzupe/lnione o~polskie znaki 
diakrytyczne.}


\bt pltt10:
\textsample{pltt10}{Ze wzgl/edu na ograniczenie liczby r/o/znych wysoko/sci 
znak/ow w~foncie do 15 niezerowych, zrezygnowano z~kusz/acego wype/lniania 
,,g/ornej po/l/owki'' fontu. }


\bt pltt12:
\textsample{pltt12}{Ponadto w~programach METAFONT-owych podj/ete zosta/ly 
zabiegi maj/ace na celu minimalizacj/e liczby r/o/znych wysoko/sci. 
Niestety, mimo to w~niekt/orych przypadkach liczba r/o/znych niezerowych 
wysoko/sci}

{\sst PL Unslanted Text Italic}: plu10
\textsample{plu10}{ przekracza 15. W~takiej sytuacji METAFONT automatycznie
zmienia niekt/ore spo/sr/od wysoko/sci. Oznacza to, /ze w~pewnych (bardzo 
z/lo/sliwych i~-- miejmy nadziej/e -- niezwykle rzadkich) przypadkach mo/ze 
si/e} 

\vbox{{\sst PL Variable Typewriter}: plvtt10
\tagfont{plvtt10}{}
\textsample{plvtt10}{ zdarzy/c, /ze tekst czysto angielski z/lo/zony fontami 
PL zostanie inaczej prze/lamany na strony ni/z tekst z/lo/zony fontami CM.}

}

\chapter{Fonty nietekstowe, przyk/lady}

\baselineskip1.2\baselineskip\hrule height0pt
{\sst PL Bold Symbols}: plbsy5, plbsy6, plbsy7, plbsy8, plbsy9, plbsy10
\tagfont{plbsy10}{}
\vskip 3pt

{\sst PL Inch}: plinch
\vskip -48pt
\textsample{plinch}{/AB/C/Z}
\vskip 4pt

{\sst PL Math Extension}: plex9, plex10
\tagfont{plex10}{}
\vskip 4pt

\advance\baselineskip-2pt
{\sst PL Math Italic}: plmi5, plmi6, plmi7, plmi8, plmi9, plmi10, plmi12
\tagfont{plmi10}{}
\vskip 4pt

{\sst PL Bold Extended Math Italic}: plmib10
\tagfont{plmib10}{}
\vskip 4pt

\vbox{{\sst PL Symbols}: plsy10, plsy5, plsy6, plsy7, plsy8, plsy9,
\tagfont{plsy10}{}
}
\eject
%%---------------------
\chapter{Instalacja i~wykorzystanie}

\baselineskip 12pt
Zasady instalacji i~wykorzystania font/ow PostScript-owych by/ly ju/z
wielokrotnie omawiane podczas spotka/n Bachotkowych przez Bogus/lawa
Jackowskiego. Prezentowano r/ownie/z zastosowanie i~instalacj/e program/ow
umo/zliwiaj/acych stosowanie PostScript-u w~\TeX-u: DVIPS i~Ghostscript.
W~tym miejscu pozwol/e sobie przypomnie/c najwa/zniejsze tylko aspekty
zagadnienia.

\parindent 1.5em

W~celu zainstalowania font/ow PostScript-owych PL nale/zy:

\smallskip

\item{1.} Zainstalowa/c program T.~Rokickiego DVIPS, przetwarzaj/acy 
pliki DVI do postaci PostScript-owej (program ten mo/ze by/c ju/z
zainstalowany gdy/z wchodzi w~sk/lad wsp/o/lczesnych dystrybucji
oprogramowania \TeX-owego).

\medskip
\item{2.} Skopiowa/c wszystkie pliki pl*.pfb do katalogu, w~kt/orym
przechowujemy fonty postscriptowe np.:

\medskip
{\tt c:\string\tex\string\texmf\string\fonts\string\type1} \ lub \
{\tt usr//tex//texmf//fonts//type1}
\medskip

\item{3.} Do pliku {\tt psfonts.map}, znajduj/acego si/e w~katalogu 
przeszukiwanym przez program DVIPS, doda/c zawarto/s/c pliku 
{\tt pl.map} za/l/aczonego do dystrybucji font/ow PLPSFONT.

\medskip
Poni/zsze, dodatkowe kroki dotycz/a jedynie wersji DOS-owej programu DVIPS:

\medskip
\item{1.} W~pliku konfiguracyjnym DVIPS o~nazwie {\tt config.ps} nale/zy
odszuka/c lini/e rozpoczynaj/ac/a si/e liter/a {\tt T} i~umie/sci/c w~niej 
/scie/zk/e dost/epu do naszych dotychczasowych plik/ow TFM, np.:
\medskip

{\tt
* Paths:

T.; lofo;c:\string\tex\string\tfm
}
\medskip

\item{2.} W~tym samym pliku, w~linii rozpoczynaj/acej si/e liter/a {\tt H}
okre/sli/c /scie/zk/e dost/epu do plik/ow zawieraj/acych fonty PostScript-owe PFB,
np:

\medskip
{\tt
* PostScript header \& (maybe) PS font paths:

H.; c:\string\tex\string\dvips\string\ps;c:\string\psfonts;c:\string\psfonts\string\enc;c:\string\tex\string\fonts
}
\medskip

\noindent Po udanej instalacji z~font/ow postscriptowych korzystamy 
dok/ladnie tak samo jak dotychczas z~font/ow bitmapowych. Mo/zliwe s/a 
wi/ec wszystkie wywo/lania font/ow opisane w~,,The \TeX Book'', a~wi/ec:

\medskip
{\tt\string\font\string\qq=plr12 \string\qq\space\space Bacho\string\TeX

\%

\string\font\string\qq\space\space plr10 at 12pt
\string\qq\space\space Bacho\string\TeX

\%

\string\font\string\qq\space\space plr10 scaled\string\magstep1
\string\qq\space\space Bacho\string\TeX

\%

\string\font\string\qq\space\space plr10 scaled 1200
\string\qq\space\space Bacho\string\TeX
}

\medskip
\noindent
Co daje nast/epuj/acy rezultat:

\medskip
{\font\qq plr12 \qq Bacho\TeX
%
\font\qq plr10 at 12pt \qq Bacho\TeX
%
\font\qq plr10 scaled\magstep1 \qq Bacho\TeX
%
\font\qq plr10 scaled 1200 \qq Bacho\TeX

}

\medskip
Mo/zna zastosowa/c r/ownie/z bardziej og/oln/a definicj/e, umieszczon/a 
w~bie/z/acym lub wczytywanym komend/a {\tt\string \input} pliku, w~postaci:
\medskip

\let\cc=\catcode
{\cc`\^^M=\active %
\gdef\losenolines{\cc`\^^M=\active \def^^M{\leavevmode\endgraf}}}
\def\literal {\begingroup \cc`\\=12 \cc`\{=12 \cc`\}=12 \cc`\$=12 \cc`\&=12
 \cc`\#=12 \cc`\%=12 \cc`\~=12 \cc`\_=12 \cc`\^=12 \cc`\*=12 \cc`\@=0
 \cc`\`=\active \losenolines \obeyspaces \tt}%
{\obeyspaces\gdef {\hglue.5em\relax}}

{\cc`\`=\active \gdef`{\relax\lq}}

\def\vquotingon{\cc`\"=\active}
\def\vquotingoff{\cc`\"=12}
\vquotingon
\def"{\literal\leavevmode\hbox\bgroup\com}
%`\leavevmode' starts a~new paragraph, if needed.

\def\beginliteral{\medskip \literal \cc`\"=12%
  \parskip0pt \baselineskip11pt \thatisit}

{\cc`\@=0 \cc`\\=12 @cc`@^^M=@active %
 @gdef@com#1"{#1@egroup@endgroup} %
 @gdef@thatisit^^M#1\endliteral{#1@endgroup@smallskip}}
\vquotingoff


\beginliteral
\newdimen\FontDimen
\def\InstallFont#1#2#3{%
  \def#1{\afterassignment#2\FontDimen=}
     \def#2{\font\currfont #3 at\FontDimen\relax\currfont
     \advance\FontDimen by.2\FontDimen         \% opcja
     \baselineskip=\FontDimen                   \% opcja
     \spaceskip.333em plus.333em minus.111em \% opcja
 }}
%
\InstallFont{\roman}{\xroman}{plr10}
\endliteral

\medskip
\noindent Teraz wystarczy napisa/c:

\medskip
\beginliteral
\roman67.85pt BACHO\TeX'97
\endliteral
\medskip

\noindent 
aby otrzyma/c:

\bigskip\bigskip
\centerline{\ap67.85pt BACHO\TeX'97}

\bigskip\bigskip\bigskip
\chapter{Powodzenia!!!}

\bye
