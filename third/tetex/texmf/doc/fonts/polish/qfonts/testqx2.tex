%& --translate-file=il2-pl
%% Begin of file `testqx2.tex'. Public domain.
%% [E] This example file was previously part of `pl-qx' package.
%% Piotr K\l{}osowski <pklosows@press.umcs.lublin.pl>
%% 2001.05.31. Changed by StaW to adapt it to the new QX encoding.
%% [PL] przyk�ad prezentuje znaki specjalne dost�pne w fontach rodziny
%% Quasi... i pochodzi z dawnego pakietu pl-qx autorstwa Piotra K�osowskiego
%% Uwaga: u�ycie kodowania QX wymaga zainstalowania font�w Quasi Times,
%% a tak�e pliku qxenc.def (zawarty jest on w pakiecie platex wer. 1.2.1)
\documentclass[11pt]{article}

\textwidth=18cm
\textheight=26cm
\hoffset=-0.5in
\voffset=-1.1in

\usepackage{polski}
%% 
%\usepackage{qpalatin} \def\FONT{Palatin}
%\usepackage{qbookman} \def\FONT{Bookman}
\usepackage{qtimes} \def\FONT{Times}
%\usepackage{qswiss} \renewcommand\familydefault{\sfdefault}\def\FONT{Swiss}
%\usepackage{qcourier} \def\FONT{Courier} % tutaj raczej nie zalecane u�ycie!!

\usepackage[QX]{fontenc}
\usepackage{tabularx,multicol}

\newcommand{\znak}[2]{%
\fontencoding{OT4}\selectfont\ttfamily \string #1
& \mbox{#1} 
& [#2]\hspace*{.5cc} \\ \hline
}
\newcommand{\akcent}[2]{%
\fontencoding{OT4}\selectfont\ttfamily \string#1\string{x\string}  
& #1{x} 
& [#2]\hspace*{.5cc} \\ \hline
}

%\prefixing
\raggedright

\begin{document}\footnotesize

\subsection*{Znaki specjalne osi�galne w kroju Quasi \FONT{}
w uk�adzie QX}

\begin{multicols}{2}

\subsubsection*{Znaki interpunkcyjne}
\begin{tabularx}{.9\hsize}{|X%
|>{\centering}p{3cc}%
|>{\raggedleft\arraybackslash}p{3cc}<{\hspace*{0.5cc}}|}\hline
Nazwa makra & Wygl�d & Numer \\ \hline
\znak{\textquotedblright}{34}
\znak{\textquoteright}{39}
\znak{\textexclamdown}{60}
\znak{\textquestiondown}{62}
\znak{\textquotedblleft}{92}
\znak{\textquoteleft}{96}
\znak{\textendash}{123}
\znak{\textemdash}{124}
\znak{\guillemotleft}{174}
\znak{\guillemotright}{175}
\znak{\quotedblbase}{255}
\end{tabularx}

\subsubsection*{Litery specjalne, ligatury}
\begin{tabularx}{.9\hsize}%
{|X%
|>{\centering}p{3cc}%
|>{\raggedleft\arraybackslash}p{3cc}<{\hspace*{0.5cc}}|}\hline
Nazwa makra & Wygl�d & Numer \\ \hline
\znak{\i}{16}
\znak{\j}{17}
\znak{\ss}{25}
\znak{\ae}{26}
\znak{\oe}{27}
\znak{\o}{28}
\znak{\AE}{29}
\znak{\OE}{30}
\znak{\O}{31}
\znak{\L}{138}
\znak{\textell}{142}
\znak{\l}{170}
\znak{\AA}{197}
\znak{\DH}{208}
\znak{\TH}{222}
\znak{\aa}{229}
\znak{\dh}{240}
\znak{\th}{254}
\znak{\SS}{---}
%\znak{\textsterling}{---}
\end{tabularx}

\subsubsection*{Symbole matematyczne}
\begin{tabularx}{.9\hsize}{|X%
|>{\centering}p{3cc}%
|>{\raggedleft\arraybackslash }p{3cc}<{\hspace*{0.5cc}}|}\hline
Nazwa makra & Wygl�d & Numer \\ \hline
\znak{\textgreater}{131}
\znak{\textxgeq}{132}
\znak{\textapprox}{133}
\znak{\textless}{136}
\znak{\textxleq}{137}
\znak{\textbraceleft}{157}
\znak{\textbraceright}{158}
\znak{\textdiv}{165}
\znak{\textminus}{168}
\znak{\texttimes}{169}
\znak{\textpm}{172}
\znak{\textinfty}{173}
\znak{\textperiodcentered}{189}
\znak{\textquotedbl}{190}
\znak{\textquotesingle}{191}
\znak{\textanglearc}{247}
\znak{\textdiameter}{248}
\end{tabularx}

\subsubsection*{Litery greckie}
\begin{tabularx}{.9\hsize}{|X%
|>{\centering}p{3cc}%
|>{\raggedleft\arraybackslash}p{3cc}<{\hspace*{0.5cc}}|}\hline
Nazwa makra & Wygl�d & Numer \\ \hline
\znak{\textDelta}{1}
\znak{\textbeta}{2}
\znak{\textdelta}{3}
\znak{\textpi}{4}
\znak{\textPi}{5}
\znak{\textSigma}{6}
\znak{\textmu}{7}
\znak{\textOmega}{10}
\end{tabularx}

\subsubsection*{Symbole tekstowe}
\begin{tabularx}{.9\hsize}{|X%
|>{\centering}p{3cc}%
|>{\raggedleft\arraybackslash}p{3cc}<{\hspace*{0.5cc}}|}\hline
Nazwa makra & Wygl�d & Numer \\ \hline
\znak{\ldots}{8}
\znak{\#}{35}
\znak{\$}{36}
\znak{\&}{38}
%\znak{\textast}{42}
%\znak{\textat}{64}
\znak{\textasciitilde}{140}
\znak{\textasciicircum}{141}
\znak{\dag}{143}
\znak{\ddag}{144}
\znak{\textdegree}{148}
\znak{\S}{159}
\znak{\textregistered}{163}
\znak{\copyright}{164}
\znak{\P}{176}
\znak{\textbullet}{180}
\znak{\textbackslash}{198}
\znak{\textcurrency}{215}
\znak{\textperthousand}{216}
\znak{\textbar}{223}
\znak{\textunderscore}{230}
\end{tabularx}

\subsubsection*{Znaki diakrytyczne}
\begin{tabularx}{.9\hsize}{|X%
|>{\centering}p{3cc}%
|>{\raggedleft\arraybackslash}p{3cc}<{\hspace*{0.5cc}}|}\hline
Nazwa makra & Wygl�d & Numer \\ \hline
\akcent{\`}{18}
\akcent{\'}{19}
\akcent{\v}{20}
\akcent{\u}{21}
\akcent{\=}{22}
  \akcent{\b}{22}
  \akcent{\r}{23}
  \akcent{\c}{24}
  \akcent{\d}{46}
  \akcent{\^}{94}
  \akcent{\.}{95}
  \akcent{\H}{125}
  \akcent{\~}{126}
  \akcent{\"}{127}
  \akcent{\k}{150}
\end{tabularx}

\end{multicols}

\end{document}

%% 
%% End of file `testqx2.tex'.
