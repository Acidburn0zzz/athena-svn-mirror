% E  Sample TeX file demonstrating how to separate by changing colors using
%    CMYK-HAX macros -- outline graphics
% P  Przyk/ladowy plik TeX-owy demonstruj/acy separacj/e kolor/ow poprzez
%    zamian/e kolor/ow za pomoc/a makr pakietu CMYK-HAX -- grafika obwiedniowa
%
\input epsf     % DVIPS standard distribution
\input cmyk-hax
\special{landscape}

\nopagenumbers
\voffset -0.25in
\hoffset -0.5in
\vsize 180mm
\hsize 270mm

\def\RED{%
 \setCMYKchange
   \forcolor{0 1 1 0}\changecolor{0 0 0 1}       % E change red to black
                                                 % P zamie/n czerwony na czarny
   \forcolor{0 0 1 0}\changecolor{0 0 0 0}       % E change yellow to white
                                                 % P zamie/n /z/o/lty na bia/ly
                     \addstroke{0 0 0 0}         % E add the white stroke to yellow
                                                 % P /z/o/lty zyskuje bia/l/a obw/odk/e
                     \strokehook{2 truemm setlinewidth % E of thickness 2 truemm
                                                       % P grubo/sci 2 truemm
                                 0 setlinejoin   % E with pointed joints
                                                 % P spiczaste po/l/aczenia
                                 2 setlinecap    % E with lenghtened tips
                                                 % P wyd/lu/zone ko/nce
                                 5 setmiterlimit % E with medium miter limits
                                                 % P obcinane /srednio daleko
                                }%
 \useCMYKchange}

\def\YELLOW{%
 \setCMYKchange
   \forcolor{0 0 1 0}\changecolor{0 0 0 1}  % E change yellow to black
                                            % P zamie/n /z/o/lty na czarny
   \forcolor{0 0 0 0}\delcolor              % E make white invisible
                                            % P uczy/n bia/ly niewidzialnym
   \forcolor{0 1 1 0}\delcolor              % E make red invisible
                                            % P uczy/n czerwony niewidzialnym
 \useCMYKchange}

\def\SPEC{%
 \setCMYKchange
   \forcolor{0 0 1 0}\changecolor{0 0 0 0}  % E change yellow to white
                                            % P zamie/n /z/o/lty na bia/ly
   \forcolor{0 0 0 0}\changecolor{0 0 0 1}  % E change white to black
                                            % P zamie/n bia/ly na czarny
   \forcolor{0 1 1 0}\delcolor              % E make red invisible
                                            % P uczy/n czerwony niewidzialnym
                     \addstroke{0 0 0 1}    % E add the black stroke to red
                                            % P czerwony zyskuje czarn/a obw/odk/e
                     \strokehook{2 truemm setlinewidth  % E of thickness 2 truemm
                                                        % P grubo/sci 2 truemm
                                 0 setlinejoin    % E with pointed joints
                                                  % P spiczaste po/l/aczenia
                                 2 setlinecap     % E with lenghten tips
                                                  % P wyd/lu/zone ko/nce
                                 5 setmiterlimit  % E with medium miter limits
                                                  % P obcinane /srednio daleko
                                }%
 \useCMYKchange}

\setbox1\hbox{\epsfysize 15cm \epsffile{gdansk.eps}}

\vbox{\centerline{%
   \PSbegingroup\RED{\copy1}\PSendgroup \hskip3cm
   \PSbegingroup\SPEC{\copy1}\PSendgroup}
 \kern-20mm
 \centerline{\vbox to0cm{\PSbegingroup\YELLOW{\copy1}\PSendgroup\vss}}}

\end
