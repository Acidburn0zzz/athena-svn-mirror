%%% File: MFobjs.tex
%%% A part of mfpic 0.6b 2003/01/02
%%%
%  Examples of \mfpic usage. For Metafont output.
\def\mac#1{\hskip0pt plus1.5pt{\tt \char`\\#1}}

\def\vs{\bigskip\filbreak}
\newsavepic\foo

\vs
{\sl \tmtitle {Basic Figures.}}

\nobreak\medskip
\noindent
\tmtitle {Point:}
\nobreak
\mfpic[15]{0}{1}{0}{1}
\point
{(0.5,0.5)}
\endmfpic

\savepic\foo
\mfpic[15]{-1}{1}{-1}{1}
\plotsymbol[3pt]{Plus}{dir 0}
\plotsymbol[3pt]{SolidCircle}{(0.5*dir 0)}
\plotsymbol[3pt]{SolidTriangle}{dir 60}
\plotsymbol[3pt]{Square}{0.5*dir 60}
\plotsymbol[3pt]{Star}{(0,0)}
\plotsymbol[3pt]{SolidSquare}{dir 120}
\plotsymbol[3pt]{Triangle}{0.5*dir 120}
\plotsymbol[3pt]{Circle}{dir 180}
\plotsymbol[3pt]{Cross}{0.5*dir 180}
\plotsymbol[3pt]{Diamond}{dir 240}
\plotsymbol[3pt]{SolidDiamond}{0.5*dir 240}
\plotsymbol[3pt]{SolidStar}{dir 300}
\plotsymbol[3pt]{Asterisk}{0.5*dir 300}
\endmfpic

\vs\noindent
\tmtitle {Plotting points with symbols:}
\nobreak\usepic\foo

\vs\noindent
\tmtitle {Plotting points with text:}
\nobreak
\mfpic[30]{0}{1}{0}{1}
\plottext{$*$}{(0,0),{(.25,.0625)},({.5},{.25}),(.75,.5625),(1,1)}
\plottext{g}{(0,1),(.25,.9375),(.5,.75),(.75,.4375),(1,0)}
\endmfpic

\vs
\noindent
\tmtitle {Line segments (Polygonal curve):}
\nobreak
\mfpic[15]{0}{2}{0}{1}
\lines{(0,0.25), (1,0.75), (1,0.25), (2,0.75)}
\endmfpic

\vs
\noindent
\tmtitle {Line segments with thicker pen:}
\nobreak
\mfpic[15]{0}{2}{0}{1}
\pen{1pt}
\lines{(0,0.25), (1,0.75), (1,0.25), (2,0.75)}
\endmfpic

\vs
\noindent
\tmtitle {Rectangle:}
\nobreak
\mfpic[15]{0}{1}{0}{1}
\rect{(0,0),(0.8,1)}
\endmfpic

\vs
\noindent
\tmtitle {Circle:}
\nobreak
\mfpic[15]{0}{1}{0}{1}
\circle{(0.5,0.5),0.5}
\endmfpic

\vs
\noindent
\tmtitle {Ellipse:}
\nobreak
\mfpic[15]{0}{2}{0}{1}
\ellipse{(1,0.5),1,0.5}
\endmfpic

\vs
\noindent
\tmtitle {Rotated Ellipse:}
\nobreak
\mfpic[15]{0}{2}{0}{1}
\ellipse[15 deg]{(1,0.5),1,0.4}
\endmfpic

\vs
\noindent
\tmtitle {Curve:}
\nobreak
\mfpic[15]{0}{2}{0}{1}
\curve{(0,0.25), (1,0.75), (1,0.25), (2,0.75)}
\endmfpic

\vs
\noindent
\tmtitle {Cyclic curve:}
\nobreak
\mfpic[20]{0}{2}{0}{1}
\cyclic{(0.5,0), (0,0.3), (0.5,0.8), (1.5,0.8), (2.0,0.3), (1.5,0),
  (1.0,0.3)}
\endmfpic

\vs
\noindent
\tmtitle {Cyclic curve with tension = 2:}
\nobreak
\mfpic[20]{0}{2}{0}{1}
\cyclic[2]{(0.5,0), (0,0.3), (0.5,0.8), (1.5,0.8), (2.0,0.3), (1.5,0),
  (1.0,0.3)}
\endmfpic

\vs
\noindent
\tmtitle {Cyclic curve with tension = 4:}
\nobreak
\mfpic[20]{0}{2}{0}{1}
\cyclic[4]{(0.5,0), (0,0.3), (0.5,0.8), (1.5,0.8), (2.0,0.3), (1.5,0),
  (1.0,0.3)}
\endmfpic

\bigskip\vs
{\sl \tmtitle {Dotted Figures.}}

\nobreak\medskip
\noindent
\tmtitle {Dotted Line segments:}
\nobreak
\mfpic[15]{0}{2}{0}{1}
\dotted\lines{(0,0.25), (1,0.75), (1,0.25), (2,0.75)}
\endmfpic

%%%%%%%
\vs
\noindent
\tmtitle {Rectangle plotted with triangles:}
\nobreak
\mfpic[20]{0}{1}{0}{1}
\plot{Triangle}\rect{(0,0),(0.8,1)}
\endmfpic

\vs
\noindent
\tmtitle {Ellipse plotted with stars, larger size and space:}
\nobreak
\mfpic[15]{0}{2}{0}{1}
\plot[3pt,6pt]{Star}\ellipse{(1,0.5),1,0.5}
\endmfpic

\vs
\noindent
\tmtitle {Cyclic curve plotted with squares, larger size, smaller space:}
\nobreak
\mfpic[20]{0}{2}{0}{1}
\plot[3pt,3pt]{Square}\cyclic {
   (0.5,0), (0,0.3),
   (0.5,0.8), (1.5,0.8),
   (2.0,0.3), (1.5,0),
   (1.0,0.3)}
\endmfpic

\vs
\noindent
\tmtitle {Curve plotted with (still larger) open circles:}
\nobreak
\mfpic[20]{0}{2}{0}{1}
\plot[4pt,6pt]{Circle}\curve{(0,0), (0.25,0.75), (0.5,0.5), (2,1)}
\endmfpic

\bigskip\vs
{\sl \tmtitle {Axes and Axis Marks.}}

\nobreak\medskip
\noindent
\tmtitle {Axes:}
\nobreak
\mfpic[15][20]{-2}{2}{-2}{2}
\axes
\endmfpic

\vs
\noindent
\tmtitle {Axis marks:}
\nobreak
\mfpic[15][20]{-2}{2}{-2}{2}
\axes
\xmarks{-1,1}
\ymarks{-1,1}
\endmfpic

\bigskip\vs
{\sl \tmtitle {Circular Arcs.}}

\nobreak\medskip
\noindent
\tmtitle {Circular arc, point-sweep form:}
\nobreak
\mfpic[15]{0}{2}{0}{1}
\point{(0,0.8),(2,0.8)}
\arc  {(0,0.8),(2,0.8),90}
\endmfpic

\vs
\nobreak
\noindent
\tmtitle {Circular arc, three-point form:}
\nobreak
\mfpic[15]{0}{2}{0}{1}
\point {(0,0.8),(1,0.1),(2,0.8)}
\arc[t]{(0,0.8),(1,0.1),(2,0.8)}
\endmfpic

\vs
\noindent
\tmtitle {Circular arc, polar form:}
\nobreak
\mfpic[15]{0}{1}{0}{1}
\point {(0,0)}
\arc[p]{(0,0),0,75,1}
\endmfpic

\vs
\noindent
\tmtitle {Circular arc, center-point form:}
\nobreak
\mfpic[15]{0}{1}{0}{1}
\point {(0,0),(1,0)}
\arc[c]{(0,0),(1,0),75}
\endmfpic

\bigskip\vs
{\sl \tmtitle {Polar Coordinates.}}

\nobreak\medskip
\noindent
\tmtitle{Polar point:}
\nobreak
\mfpic[15]{0}{1}{0}{1}
\point{\plr{(1,45)}}
\endmfpic

\vs
\noindent
\tmtitle {Polar lines:}
\nobreak
\mfpic[15]{0}{1}{0}{1}
\lines{\plr{(0.25,0), (1,30), (0.75,45), (1,60), (0.25,90)}}
\endmfpic

\vs
\noindent
\tmtitle {Polar curve:}
\nobreak
\mfpic[15]{0}{1}{0}{1}
\curve{\plr{(0.25,0), (1,30), (0.75,45), (1,60), (0.25,90)}}
\endmfpic

\vs
\noindent
\tmtitle {Polar Cyclic curve:}
\nobreak
\mfpic[15]{0}{1}{0}{1}
\cyclic{\plr{(0.25,0), (1,30), (0.75,45), (1,60), (0.25,90), (0.1,45)}}
\endmfpic

\bigskip\vs
{\sl \tmtitle {Other Figures.}}

\nobreak\medskip
\noindent
\tmtitle {Line segments (polar ``turtle''):}
\nobreak
\mfpic[15]{0}{2}{0}{1}
\turtle{\plr{(0,0), (0.8,0), (0.4,135), (0.4,-90), (0.4,-90), (0.4,45)}}
\endmfpic

\vs
\noindent
\tmtitle {Sector:}
\nobreak
\mfpic[15]{0}{1}{0}{1}
\sector{(0,0),1,0,75}
\endmfpic

\bigskip\vs
{\sl \tmtitle {Closed figures.}}

\nobreak\medskip
\noindent
\tmtitle {Polygon:}
\nobreak
\mfpic[20]{0}{2}{0}{1}
\lclosed\lines{(0,0), (0,0.5), (1,1), (1.5,0.5), (1,0.5)}
\endmfpic

\vs
\noindent
\tmtitle {Line closure of curve:}
\nobreak
\mfpic[20]{0}{2}{0}{1}
\lclosed\curve{(0.5,0), (0,0.3), (0.5,0.8), (1.5,0.8), (2.0,0.3),
   (1.5,0)}
\endmfpic

\vs
\noindent
\tmtitle {Line closure of circular arc:}
\nobreak
\mfpic[15]{0}{2}{0}{1}
\lclosed\arc{(0,0.8),(2,0.8),90}
\endmfpic

% change point disk diameter.
\pointsize=3pt

\vs
\noindent
\tmtitle{\mac{rhatch}\mac{draw}\mac{lclosed}\mac{connect}\mac{curve}\mac{lines}.}
\tmtitle{Points have been added to both ends of the two paths:}
\nobreak
\mfpic[10]{-1}{1}{-1}{1}
\rhatch\draw\lclosed\connect\curve{(1,0), (0.5,0.25), (0.5,0.5), (0,0.75)}
\lines{(-0.5,0.5), (-0.5,-0.5), (0.5,-0.75)}
\point{(1,0),(0,0.75), (-0.5,0.5), (0.5,-0.75)}
\endconnect
\endmfpic

\vs
\noindent
\tmtitle {Similar but Reversing the Polyline's endpoints:}
\nobreak
\mfpic[10]{-1}{1}{-1}{1}
\rhatch\draw\lclosed\connect
\curve{(1,0), (0.5,0.25), (0.5,0.5), (0,0.75)}
\reverse\lines{(-0.5,0.5), (-0.5,-0.5), (0.5,-0.75)}
\point{(1,0),(0,0.75), (-0.5,0.5), (0.5,-0.75)}
\endconnect
\endmfpic

\vs
\noindent
\tmtitle {Closed connected figure (cross-hatched and outlined)
consisting of
1 curve,
1 vertical line (with its ends shown by filled disks),
and
3 polar points (shown by open disks);
this figure is followed by another vertical line
which is off the right edge of the picture boundary:}

\nobreak\smallskip
\noindent
\mfpic[10]{-1}{1}{-1}{1}
\hatch\draw\lclosed\connect
\curve{(1,0), (0.5,0.25), (0.5,0.5), (0,0.75)}
\lines{(-0.8,-0.8), (-0.8,0.8)}
\point{\plr{(0,0), (2,-90), (sqrt2,-45)}}
\endconnect
\point{(-0.8,-0.8), (-0.8,0.8)}
{\pointfillfalse
 \point{\plr{(0,0), (2,-90), (sqrt2,-45)}}}
\lines{(2,-1), (2,1)}
\endmfpic

\vs
\noindent
\tmtitle {Constrained reverse video of the previous picture:}

\nobreak\smallskip
\noindent
\mfsrc{picture lastpic; lastpic := currentpicture;}
\mfpic[10]{-1}{1}{-1}{1}
\mfsrc{currentpicture :=
  picneg (lastpic)
    (rect ((xneg,yneg),(xpos,ypos)) transformed ztr);}
\endmfpic

\vs
\noindent
\tmtitle {Unconstrained reverse video of that picture:}

\nobreak\smallskip
\noindent
\mfpic[10]{-1}{1}{-1}{1}
\mfsrc{currentpicture :=
  lastpic picxor
    (interior rect ((xneg,yneg),(xpos,ypos)) transformed ztr);}
\endmfpic

% restore point disk diameter.
\pointsize=2pt

\bigskip\vs
{\sl \tmtitle {Arrows.}}

\nobreak\medskip
\noindent
\tmtitle {Basic arrow:}
\nobreak
\mfpic[15]{0}{2}{0}{1}
\arrow\lines{(0,0.25), (1,0.75), (1,0.25), (2,0.75)}
\endmfpic

\vs
\noindent
\tmtitle {Dotted arrow:}
\nobreak
\mfpic[15]{0}{2}{0}{1}
\arrow\dotted\lines{(0,0.25), (1,1), (1,0), (2,0.75)}
\endmfpic

\vs
\noindent
\tmtitle {Circular arc arrow:}
\nobreak
\mfpic[15]{0}{2}{0}{1}
\arrow\arc{(0,0.8),(2,0.8),90}
\endmfpic

\vs
\noindent
\tmtitle {Curved arrow:}
\nobreak
\mfpic[15]{0}{2}{0}{1}
\arrow\curve{(0,0.25), (1,0.75), (1,0.25), (2,0.75)}
\endmfpic

\vs
\noindent
\tmtitle {Bi-directional arrow:}
\nobreak
\mfpic[15]{0}{2}{0}{1}
\arrow\reverse\arrow\lines{(0,0.25), (1,0.75), (1,0.25), (2,0.75)}
\endmfpic

\vs
\noindent
\tmtitle {Double arrowhead:}
\nobreak
\mfpic[15]{0}{2}{0}{1}
\arrow\arrow[b2pt]\lines{(0,0.25), (1,0.75), (1,0.25), (2,0.75)}
\endmfpic

\bigskip\vs
{\sl \tmtitle {Shading and Filling.}}

\nobreak\medskip
\noindent
\tmtitle {Filled Rectangle:}
\nobreak
\mfpic[15]{0}{1}{0}{1}
\gfill\rect{(0,0), (0.8,1)}
\endmfpic

\vs
\noindent
\tmtitle {Polkadot Ellipse:}
\nobreak
\mfpic[15]{0}{3}{0}{2}
\draw\polkadot\ellipse{(1.5,1),1.5,1}
\endmfpic

\vs
\noindent
\tmtitle {Filled Cyclic curve:}
\nobreak
\mfpic[20]{0}{2}{0}{1}
\gfill\cyclic{(0.5,0), (0,0.3), (0.5,0.8), (1.5,0.8), (2.0,0.3),
  (1.5,0), (1.0,0.3)}
\endmfpic

\vs
\noindent
\tmtitle {Left-hatched Cyclic curve:}
\nobreak
\mfpic[20]{0}{2}{0}{1}
\lhatch\cyclic{(0.5,0), (0,0.3), (0.5,0.8), (1.5,0.8), (2.0,0.3),
  (1.5,0), (1.0,0.3)}
\endmfpic

\vs
\noindent
\tmtitle {Shaded Polygon, with Outline:}
\nobreak
\mfpic[20]{0}{2}{0}{1}
\shade\draw\lclosed\lines{(0,0), (0,0.5), (1,1), (1.5,0.5), (1,0.5)}
\endmfpic

\vs
\noindent
\tmtitle {\mac{shade}\mac{lclosed}\mac{curve}:}
\nobreak
\mfpic[20]{0}{2}{0}{1}
\shade\lclosed\curve{(0.5,0), (0,0.3), (0.5,0.8), (1.5,0.8), (2.0,0.3),
  (1.5,0)}
\endmfpic

\vs
\noindent
\tmtitle{same, but with \mac{shadewd} increased 50\%:}

\noindent
\tmtitle {\mac{shade}\mac{lclosed}\mac{curve}:}
\nobreak
\mfpic[20]{0}{2}{0}{1}
\shadewd{.75pt}
\shade\lclosed\curve{(0.5,0), (0,0.3), (0.5,0.8), (1.5,0.8), (2.0,0.3),
  (1.5,0)}
\endmfpic

\vs
\noindent
\tmtitle {Prefix example 1.}\par
\tmtitle {\mac{draw}\mac{shade}\mac{lclosed}\mac{lines};
Local coordinate change to shift and rotate second copy of polygon:}

\nobreak\smallskip
\noindent
\mfpic[15]{0}{5}{0}{1}
\draw\shade\lclosed
  \lines{(0,0), (0.5,1), (1.5,1), (2,0)}
\coords
  \shift{(3,0)}
  \rotatearound{(1,.5)}{90}
  \draw\shade\lclosed
    \lines{(0,0), (0.5,1), (1.5,1), (2,0)}
\endcoords
\endmfpic

\vs
\noindent
\tmtitle {Prefix context example 2:}\par
\tmtitle {The first polygon is the same as example~1, the second is
\mac{shade}\mac{lclosed}\mac{draw}\mac{lines}. Moreover,
\mac{shadespace} is 2pt, \mac{penwd} is 2pt, and a closed
rectangle is on top.}

\nobreak\smallskip
\noindent
\mfpic[15]{0}{5}{0}{1}
\shadespace 2pt
\pen{2pt}
\draw\shade\lclosed
  \lines{(0,0),(0.5,1),(1.5,1),(2,0)}
\shade\lclosed\draw
  \lines{(3,0),(3.5,1),(4.5,1),(5,0)}
\lclosed
  \lines{(0,0.25),(5,0.25),(5,0.75),(0,0.75)}
\endmfpic

\vs
\noindent
\tmtitle {Overlay example.}
\tmtitle {Third Polyline now has `\mac{gclear}' before it:}

\nobreak\smallskip
\noindent
\mfpic[15]{0}{5}{0}{1}
\shadespace 2pt
\pen{2pt}
\draw\shade\lclosed
  \lines{(0,0),(0.5,1),(1.5,1),(2,0)}
\shade\lclosed\draw
  \lines{(3,0),(3.5,1),(4.5,1),(5,0)}
\gclear\lclosed
  \lines{(0,0.25),(5,0.25),(5,0.75),(0,0.75)}
\endmfpic

\bigskip\vs
{\sl \tmtitle {Functions.}}

\nobreak\medskip
\noindent
\tmtitle {Function plot:}
\nobreak
\mfpic[20]{0}{1}{0}{1}
\function{0,1,0.125}{sind(180x)}
\endmfpic

\vs
\noindent
\tmtitle {Polygonal function plot:}
\nobreak
\mfpic[20]{0}{1}{0}{1}
\function[p]{0,1,0.125}{sind(180x)}
\endmfpic

\vs
\noindent
\tmtitle {Parametric function plot:}
\nobreak
\mfpic[20]{0}{1}{0}{1}
\parafcn{0,1,0.125}{sind(180t) * dir(90t)}
\endmfpic

\vs
\noindent
\tmtitle {Polygonal parametric function plot:}
\nobreak
\mfpic[20]{0}{1}{0}{1}
\parafcn[p]{0,1,0.125}{sind(180t) * dir(90t)}
\endmfpic

\vs
\noindent
\tmtitle {Polar function plot:}
\nobreak
\mfpic[20]{0}{1}{0}{1}
\plrfcn{0,90,10}{sind(2t)}
\endmfpic

\vs
\noindent
\tmtitle {Polygonal polar function plot:}
\nobreak
\mfpic[20]{0}{1}{0}{1}
\plrfcn[p]{0,90,10}{sind(2t)}
\endmfpic

\vs
\noindent
\tmtitle {Between functions (shaded):}
\nobreak
\mfpic[20]{0}{1}{0}{1}
\function{0,0.7,0.1}{x**2}
\function{0,0.7,0.1}{sind(180x)}
\shade\btwnfcn{0,0.7,0.1}{x**2}{sind(180x)}
\endmfpic

\vs
\noindent
\tmtitle {Polar region (shaded):}
\nobreak
\mfpic[20]{0}{1}{0}{1}
\shade\draw\plrregion{0,90,10}{sind(2t)}
\endmfpic

\bigskip\vs
{\sl \tmtitle {Labels and Captions.}}

\nobreak\medskip
\noindent
\tmtitle {Label:}
\nobreak
\mfpic[20]{-1}{1}{-1}{1}
% head
\circle{(0,0),1}
% nose
\point{(0,0)}
% eyes
{\pointfillfalse
 \point{(0.333,0.333), (-0.333,0.333)}}
% smile
\arc{(-0.333,-0.333),(0.333,-0.333),60}
% arrow from label to nose
\arrow\lines{(1,0.5), (0.1,0.05)}
\tlabel[Bl angle (.9,.45)](1,{0,5}){nose} % angle will be ignored
\endmfpic

\vs
\noindent
\tmtitle {Caption:}
\nobreak
\mfpic[20]{-1}{1}{-1}{1}
\circle{(0,0),1}
\point{(0,0)}
{\pointfillfalse
 \point{(0.333,0.333), (-0.333,0.333)}}
\arc{(-0.333,-0.333),(0.333,-0.333),60}
\tcaption{Smile!}
\endmfpic

\bigskip\vs
{\sl \tmtitle {EXTRAS.}}

\nobreak\bigskip
{\sl \tmtitle {Tiling.}}

\nobreak
\mfpicunit=1pt
\mfpic[20]{-3.5}{3.5}{-3.1}{3.5}
\tile{fred, 1pt, 10, 10, false}
\lines{(0,0), (5,5), (10,0)}
\point{(2.5,7.5)}
\endtile
\draw\tess{fred}\rect{(-3,-3),(3,3)}
\tcaption{Tesselation.}
\endmfpic

\def\mypoints{% abbreviation
\tlabelsep{2pt}
\point{(0,0),(0,1),(1,1),(1,0)}
\tlabels{[tr](0,0){P1}
         [br](0,1){P2}
         [bl](1,1){P3}
         [tl](1,0){P4}}}

\bigskip\vs
\noindent
\tmtitle {Open Quadratic B-spline:}

\nobreak
\bigskip
\noindent
\mfpic[50]{0}{1}{0}{1}
\mypoints
\qspline{(0,0),(0,1),(1,1),(1,0)}
\endmfpic

\vs
\noindent
\tmtitle {Open Quadratic B-spline closed by line segment:}

\nobreak
\bigskip
\noindent
\mfpic[50]{0}{1}{0}{1}
\mypoints
\lclosed\qspline{(0,0),(0,1),(1,1),(1,0)}
\endmfpic

\vs
\noindent
\tmtitle {Closed Quadratic B-spline:}

\nobreak
\bigskip
\noindent
\mfpic[50]{0}{1}{0}{1}
\mypoints
\closedqspline{(0,0),(0,1),(1,1),(1,0)}
\endmfpic

\vs
\noindent
\tmtitle {Open Quadratic B-spline closed by B\'ezier:}

\nobreak
\bigskip
\noindent
\mfpic[50]{0}{1}{0}{1}
\mypoints
\bclosed\qspline{(0,0),(0,1),(1,1),(1,0)}
\endmfpic

\vs
\noindent
\tmtitle {Open Cubic B-spline:}

\nobreak
\bigskip
\noindent
\mfpic[50]{0}{1}{0}{1}
\mypoints
\cspline{(0,0),(0,1),(1,1),(1,0)}
\endmfpic

\vs
\noindent
\tmtitle {Open Cubic B-spline closed by line segment:}

\nobreak
\bigskip
\noindent
\mfpic[50]{0}{1}{0}{1}
\mypoints
\lclosed\cspline{(0,0),(0,1),(1,1),(1,0)}
\endmfpic

\vs
\noindent
\tmtitle {Closed Cubic B-spline:}

\nobreak
\bigskip
\noindent
\mfpic[50]{0}{1}{0}{1}
\mypoints
\closedcspline{(0,0),(0,1),(1,1),(1,0)}
\endmfpic

\vs
\noindent
\tmtitle {Open Cubic B-spline closed by B\'ezier:}

\nobreak
\bigskip
\noindent
\mfpic[50]{0}{1}{0}{1}
\mypoints
\bclosed\cspline{(0,0),(0,1),(1,1),(1,0)}
\endmfpic

\vs
\noindent
\tmtitle {Open Cubic B-spline closed by Smooth path:}

\nobreak
\bigskip
\noindent
\mfpic[50]{0}{1}{0}{1}
\mypoints
\sclosed\cspline{(0,0),(0,1),(1,1),(1,0)}
\endmfpic

\vs
\noindent
\tmtitle {Open Cubic B-spline closed by Cubic B-spline:}

\nobreak
\bigskip
\noindent
\mfpic[50]{0}{1}{0}{1}
\mypoints
\cbclosed\cspline{(0,0),(0,1),(1,1),(1,0)}
\endmfpic

\vs
\noindent
\tmtitle {Open Curve:}

\nobreak
\bigskip
\noindent
\mfpic[50]{0}{1}{0}{1}
\mypoints
\curve{(0,0),(0,1),(1,1)}
\endmfpic

\vs
\noindent
\tmtitle {Curve closed by Smooth path:}

\nobreak
\bigskip
\noindent
\mfpic[50]{0}{1}{0}{1}
\mypoints
\sclosed\curve{(0,0),(0,1),(1,1)}
\endmfpic

\vs
\noindent
\tmtitle {Curve closed smoothly, keeping original curve unchanged:}

\nobreak
\bigskip
\noindent
\mfpic[50]{0}{1}{0}{1}
\mypoints
\uclosed\curve{(0,0),(0,1),(1,1)}
\endmfpic

\vs
\noindent
\tmtitle {Curve closed by Cubic B-spline:}

\nobreak
\bigskip
\noindent
\mfpic[50]{0}{1}{0}{1}
\mypoints
\cbclosed\curve{(0,0),(0,1),(1,1)}
\endmfpic

\vs
\noindent
\tmtitle {Curve closed by B\'ezier:}

\nobreak
\bigskip
\noindent
\mfpic[50]{0}{1}{0}{1}
\mypoints
\bclosed\curve{(0,0),(0,1),(1,1)}
\endmfpic

\bigskip\vs
\tmtitle{\sl Storing and reusing:\/ \mac{store} and \mac{mfobj}.}

\nobreak\medskip
\noindent
\tmtitle {Use of \mac{store} and \mac{mfobj} for a Dashed and Hatched Circle:}

\nobreak
\bigskip
\noindent
\mfpic[50]{0}{2}{0}{1}
\store{f}{\circle{(0.5,0.5),0.5}}
\coords
 \shift{(1,0)}
 \hatch\mfobj{f}
\endcoords
\dashed\mfobj{f}
\endmfpic

\vs
\noindent
\tmtitle {Example of \mac{store} and \mac{mfobj} for two dashed curves,}
\tmtitle {--- one a shifted copy of the other ---}
\tmtitle {and a shaded closed curve containing them both.}


\nobreak
\bigskip
\noindent
\mfpic[20]{0}{5}{0}{2}
\mfsrc{save f,g,h;}
\store{f}{\curve{(1,0),(0,1),(1,2)}}
\store{g}{\shiftpath{(3,0)}\reverse\mfobj{f}}
\store{h}{\lclosed\connect\mfobj{f}\mfobj{g}\endconnect}
 \lightershade
\shade\mfobj{h}
\dashed\mfobj{f}
\dashed\mfobj{g}
\endmfpic

\hatchspace3.5pt
\vs
\noindent
\tmtitle{Various transformation prefixes. \mac{rotatepath} and}
\tmtitle{\mac{reflectpath} compensate for different scales, the rest do not.}

\nobreak
\bigskip
\noindent
\mfpic[30][40]{-.2}{1.5}{-0.3}{1.3}
 \store{xq}{\rect{(0,0),(1,1)}}
\lightershade
\lightershade
 \shade\mfobj{xq}
 \draw \lhatch\rotatepath{(.5,.5),30}\mfobj{xq}
 \draw \rhatch\rotatepath{(.5,.5),60}\mfobj{xq}
 \tcaption{Rotate}
\endmfpic
%
\mfpic[30][40]{-.5}{1.5}{-0.3}{1.3}
 \store{xq}{\rect{(0,0),(1,1)}}
\lightershade
\lightershade
 \draw\lhatch\xslantpath{.5,1}\shade\mfobj{xq}
 \draw\rhatch\yslantpath{0,.5}\mfobj{xq}
 \tcaption{Slant}
\endmfpic
%
\mfpic[30][40]{-.2}{1.5}{-0.3}{1.3}
 \store{xq}{\rect{(0,0),(1,1)}}
\lightershade
\lightershade
 \draw\lhatch\xscalepath{0.5,1.3}\yscalepath{0.5,.6}\shade\mfobj{xq}
 \draw\rhatch\yscalepath{0,1.3}\xscalepath{0,.6}\mfobj{xq}
 \tcaption{x/y Scale}
\endmfpic
%
\mfpic[30][40]{-.2}{1.5}{-0.3}{1.3}
 \store{xq}{\rect{(0,0),(1,1)}}
\lightershade
\lightershade
 \draw\lhatch\scalepath{(.25,.25),1.2}\shade\mfobj{xq}
 \draw\rhatch\scalepath{(.25,.25),1.44}\mfobj{xq}
 \tcaption{Scale}
\endmfpic
%
\mfpic[30][40]{-.2}{1.6}{-0.3}{1.3}
 \store{xq}{\rect{(0,0),(1,1)}}
\lightershade
\lightershade
 \draw\rhatch\reflectpath{(.25,1),(1,.25)}\shade\mfobj{xq}
 \dashed\lines{(.25,1),(1,.25)}
 \tcaption{Reflect}
\endmfpic
%
\mfpic[30][40]{-.3}{1.5}{-0.3}{1.3}
 \store{xq}{\ellipse{(.65,.55),.5,.7}}
\lightershade
\lightershade
 \draw\lhatch\xyswappath\shade\mfobj{xq}
 \dashed\lines{(0,0),(1.3,1.3)}
 \tcaption{x/y Swap}
\endmfpic
\hss

\hatchspace3pt

\vs
\mfpicdebugtrue
\noindent\tmtitle {(Debugging turned on here.)}
\bigskip\vs
{\sl \tmtitle {Plotting data:}}

\nobreak\medskip
\noindent
\tmtitle {Plotting polygonal path with points from a file:}

\nobreak
\bigskip
\noindent
\mfpic[6]{-10}{10}{-2}{12}
\makepercentother
\using{#1% #2 #3}{(#1,#2)}
\makepercentcomment
\mfpdatacomment\#
\dashed[2pt,3pt]\polyline\datafile{data.dat}
\endmfpic

\vs
\mfpicdebugfalse
\noindent\tmtitle {(Debugging turned off here.)}
\bigskip\vs
\noindent
\tmtitle {Plotting smooth path with points from same file:}

\nobreak
\bigskip
\noindent
\mfpic[6]{-10}{10}{-2}{12}
\makepercentother
\using{#1% #2 #3}{(#1,#2)}
\makepercentcomment
\mfpdatacomment\#
\dashed[2pt,3pt]\curve[2]\datafile{data.dat}
\endmfpic

\vs
\noindent
\tmtitle {Faking a function from points with increasing x-values}
\tmtitle {AND plotting the nodes with open circles:}

\nobreak
\bigskip
\noindent
\mfpic[6]{-10}{10}{0}{12}
\draw\plotnodes {Circle}
\fcncurve{(-10,11.68), (-9,9.62), (-8.5,8.445), (-8,8.28),
(-7.5,5.865), (-7,5.5), (-5.5,3.445), (-5,1),
(-4.5,1.965), (-4,2.14), (-3.5,2.985), (-2.5,-1.215),
(-2,1.52), (-0.5,1.545), (0,0.42),
(0.5,-0.435), (1,0.96), (2.5,-1.235), (3,0.94),
(3.5,0.165), (4,0.28), (5,1.1), (5.5,3.145), (6,1.94),
(6.5,5.505), (7,3.44),
(9.5,8.925), (10,9.84)}
\axes
\endmfpic

\vs
\noindent
\tmtitle {Same as above, but with clipping:}

\nobreak
\bigskip
\noindent
\mfpic[6]{-10}{10}{0}{12}
\draw\plotnodes {Circle}
\fcncurve[1.2]{(-10,11.68), (-9,9.62), (-8.5,8.445), (-8,8.28),
(-7.5,5.865), (-7,5.5), (-5.5,3.445), (-5,1),
(-4.5,1.965), (-4,2.14), (-3.5,2.985), (-2.5,-1.215),
(-2,1.52), (-0.5,1.545), (0,0.42),
(0.5,-0.435), (1,0.96), (2.5,-1.235), (3,0.94),
(3.5,0.165), (4,0.28), (5,1.1), (5.5,3.145), (6,1.94),
(6.5,5.505), (7,3.44),
(9.5,8.925), (10,9.84)}
\axes
\tcaption{Moral: always leave room for your axes.}
\clipmfpic
\endmfpic

\vs
\noindent
\tmtitle{Plot several curves from datafiles with different dashing
patterns. Top 4 curves use parameter {\tt [s]}, bottom four use default
{\tt [p]}.}

\nobreak
\bigskip
\noindent
\mfpdatacomment\#
\mfpic[20][20]{-1}{4}{-4.5}{4.5}
\dashedlines
\plotdata[s1.2]{curves1.dat}
\plotdata[p]{curves2.dat}
\tlabeljustify{cr}
\tlabelsep{2pt}
\tlabels{(0, 4) {style: 0}
         (0, 3) {1}
         {(0, 2)} {2}
         ({0},{1}) {3}
         (0,-1) {4}
         (0,-2) {5}
         (0,-3) {0}
         (0,-4) {1}
}
\dotted\axis x
\tcaption{Plot with dashes}
\endmfpic
\quad
\mfpic[20][20]{-1}{4}{-4.5}{4.5}
\pointedlines
\plotdata[s]{curves1.dat}
\datapointsonly
\mfplinestyle{5}
\plotdata{curves2.dat}
\tlabeljustify{cr}
\tlabelsep{2pt}
\tlabels{(0, 4) {style: 0}
         (0, 3) {1}
         (0, 2) {2}
         (0, 1) {3}
         (0, 0) {Data only:}
         (0,-1) {4}
         (0,-2) {5}
         (0,-3) {6}
         (0,-4) {7}}
\dashed\axis y
\tcaption{Plot with points}
\endmfpic

\vs
\noindent
\tmtitle{Side axes, ticks, multiple labels and their adjustments.}

\nobreak
\bigskip
\noindent
\mfpic[50][50]{-1.1}{1.1}{-1.1}{1.1}
\sideheadlen5pt
\setaxismargins {.3}{0}{0}{0}
\doaxes{bl}
\sideheadlen=0pt
\axis r
\setbordermarks{inside}{ontop}{onright}{inside}
\lmarks{-.5,0,.5}
\bmarks{-.5,0,.5}
\rmarks{-.5,0,.5}
\tlabelsep{5pt}
\tlabels{
[tl](-1,1){TL}  [bl](-1,-1){BL}
[tr](1,1){TR} [br](1,-1){BR}}
\point{(-1,1),(-1,-1),(1,1),(1,-1)}
\tlabelsep{2pt}
\axislabels l{{$-0.5$}-.5, {$0$}0, {$0.5$}.5}
\endmfpic

\vs
\noindent
\tmtitle{Rectangle around given text, rounded corners. Use
\mac{everytlabel} to turn off interlineskip.}


\nobreak
\bigskip
\noindent
\mfpic[72]{0}{1}{0}{1}
 \tlabelsep{2pt}
\everytlabel{\offinterlineskip}
\lightershade
 \shade\rect{(0,0.25),(1,.75)}
 \plot[3pt,6pt]{Star}
   \gclear
     \tlabelrect[4pt]{(.5,.5)}{Look\\here!} %
\endmfpic

\vs
\noindent
\tmtitle{Oval sized to given text. Testing global setting of
\mac{everytlabel} by increasing baselines to 18pt.}
\everytlabel{\baselineskip 18pt\relax}

\nobreak
\bigskip
\noindent
\mfpic[72]{0}{1}{0}{1}
\penwd{1pt}
 \tlabelsep{2pt}
\lightershade
 \shade\rect{(0,0.25),(1,.75)}
 \draw
   \gclear
     \tlabeloval*(.5,.5){Sized to\\this text}
\penwd{0.3pt}
 \draw\tlabelrect*({.5},{0.5}){Sized to\\this text}
 \draw\tlabelrect(1.5,.5){Sized to\\this text}
\endmfpic

\vs
\noindent
\tmtitle{Text surrounded by ellipse with given ratio: w/h = $0.7$.}

\nobreak
\bigskip
\noindent
\mfpic[72]{0}{1}{0}{1}
\penwd{1pt}
 \tlabelsep{4pt}
\lightershade
 \shade\rect{(0,0.25),(1,.75)}
 \dashed
   \gclear
     \tlabelellipse[0.7](.5,.5){Unfilled\\ellipse}
\endmfpic
\everytlabel{}


\vs
\noindent
\tmtitle{Point grid, and line grid.}

\nobreak
\bigskip
\noindent
\mfpic[24]{-2}{2}{-2}{2}
\grid[2pt]{1,1}
\gridlines{1,1}
\endmfpic

\vs
\noindent
\tmtitle{Polar grid.}

\nobreak
\bigskip
\noindent
\mfpic[72]{-.5}{1}{-.5}{.5}
\plrgrid{.25,10}
\rect{(-.5,-.5),(1,.5)}
\endmfpic

\vs
\noindent
\tmtitle{Polar coordinate patch.}

\nobreak
\bigskip
\noindent
\mfpic[50]{0}{1.5}{-1}{1}
\plrpatch{.2,1.4,.3,-30,45,15}
\endmfpic

\vs
\noindent
\tmtitle{Pie chart.}

\nobreak
\bigskip
\noindent
\mfpic[50]{0}{3}{-1}{1}
\tlabelsep{3pt}
\tlabeljustify{cl}
\piechart{(1,0),.8}{78, 66, 54.7, 32, 5}
\draw\shade\piewedge1
\draw\shade\rect{(2,.8),(2.2,1)}
  \lightershade
\draw\shade\piewedge2
\draw\shade\rect{(2,.5),(2.2,.7)}

\draw\lhatch\piewedge[x.2]3
\draw\lhatch\rect{(2,.2),(2.2,.4)}

\draw\rhatch\piewedge4
\draw\rhatch\rect{(2,-.1),(2.2,.1)}

\draw\piewedge5
\draw\rect{(2,-.4),(2.2,-.2)}

\tlabels{%
(2.2,.9){78.0}
(2.2,.6){66.0}
(2.2,.3){54.7}
(2.2,.0){32.0}
(2.2,-.3){5.0}
}
\tcaption{Amounts (in billions)}
\endmfpic

\vs
\noindent
\tmtitle{Bar chart.}

\nobreak
\bigskip
\noindent
\mfpic[20][1]{0}{6.5}{-5}{85}
  \barchart[1,1,.3]{v}{-5.0,32.0,66.3,54.7,78.0,20.5}
  \draw\shade\chartbar1  \lightershade
  \draw\shade \chartbar2
  \draw\lhatch\chartbar3
  \draw\rhatch\chartbar4
  \draw\xhatch\chartbar5
  \draw\chartbar6
%
  \barchart[1,1,-.3]{v}{6.0,30.0,55.3,44.7,80.0,21.5}
  \draw\lhatch\chartbar1
  \draw\rhatch\chartbar2
  \draw\shade\chartbar3  \darkershade
  \draw\shade\chartbar4
  \draw\chartbar5
  \draw\xhatch\chartbar6
%
\axisheadlen0pt
\axes
\setymarks{inside}
\ymarks[2.5pt]{20,40,60,80}
%
\tlabelsep{3pt}
  \axislabels{y}{{20}20,{40}40,{60}60,{80}80}
\tlabelsep{5pt}
  \axislabels{x}{{A}1,{B}2,{C}3,{D}4,{E}5,{F}6}
\endmfpic

\vs
\noindent
\tmtitle{Almost the same, but bars clipped to an ellipse}

\nobreak
\bigskip
\noindent
\mfpic[20][1]{0}{6.5}{-5}{85}
  \barchart[1,1,.3]{v}{-5.0,32.0,66.3,54.7,78.0,20.5}
  \draw\shade\chartbar1  \lightershade
  \draw\shade \chartbar2
  \draw\lhatch\chartbar3
  \draw\rhatch\chartbar4
  \draw\xhatch\chartbar5
  \draw\chartbar6
%
  \barchart[1,1,-.3]{v}{6.0,30.0,55.3,44.7,80.0,21.5}
  \draw\lhatch\chartbar1
  \draw\rhatch\chartbar2
  \draw\shade\chartbar3  \darkershade
  \draw\shade\chartbar4
  \draw\chartbar5
  \draw\xhatch\chartbar6
\gclip\ellipse{(3,30),2,20}
%
\axisheadlen0pt
\axes
\setymarks{inside}
\ymarks[2.5pt]{20,40,60,80}
%
\tlabelsep{3pt}
  \axislabels{y}{{20}20,{40}40,{60}60,{80}80}
%\tlabelsep{5pt}
%  \axislabels{x}{{A}1,{B}2,{C}3,{D}4,{E}5,{F}6}
\endmfpic

\vfil

