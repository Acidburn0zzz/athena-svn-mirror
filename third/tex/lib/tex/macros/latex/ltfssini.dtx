% \iffalse meta-comment
%
% Copyright 1993 1994 1995 1996 1997
% The LaTeX3 Project and any individual authors listed elsewhere
% in this file. 
% 
% For further copyright information, and conditions for modification
% and distribution, see the file legal.txt, and any other copyright
% notices in this file.
% 
% This file is part of the LaTeX2e system.
% ----------------------------------------
%   This system is distributed in the hope that it will be useful,
%   but WITHOUT ANY WARRANTY; without even the implied warranty of
%   MERCHANTABILITY or FITNESS FOR A PARTICULAR PURPOSE.
% 
%   For error reports concerning UNCHANGED versions of this file no
%   more than one year old, see bugs.txt.
% 
%   Please do not request updates from us directly.  Primary
%   distribution is through the CTAN archives.
% 
% 
% IMPORTANT COPYRIGHT NOTICE:
% 
% You are NOT ALLOWED to distribute this file alone.
% 
% You are allowed to distribute this file under the condition that it
% is distributed together with all the files listed in manifest.txt.
% 
% If you receive only some of these files from someone, complain!
% 
% 
% Permission is granted to copy this file to another file with a
% clearly different name and to customize the declarations in that
% copy to serve the needs of your installation, provided that you
% comply with the conditions in the file legal.txt.
% 
% However, NO PERMISSION is granted to produce or to distribute a
% modified version of this file under its original name.
%  
% You are NOT ALLOWED to change this file.
% 
% 
% 
% \fi
% \iffalse
%%% From File: ltfssini.dtx
%% Copyright (C) 1989-1996 Frank Mittelbach and Rainer Sch\"opf,
%% all rights reserved.
%
%<*driver>
% \fi
% 
%
\ProvidesFile{ltfssini.dtx}
             [1996/12/06 v3.0h LaTeX Kernel (NFSS Initialisation)]
% \iffalse
\documentclass{ltxdoc}
\begin{document}
\DocInput{ltfssini.dtx}
\end{document}
%</driver>
% \fi
%
% \iffalse
%<+checkmem>\def\CHECKMEM{\tracingstats=2
%<+checkmem>  \newlinechar=`\^^J
%<+checkmem>  \message{^^JMemory usage: \filename}\shipout\hbox{}}
%<+checkmem>\CHECKMEM
% \fi
%
% \CheckSum{297}
%
%
% \GetFileInfo{ltfssini.dtx}
% \title{A new font selection scheme for \TeX{} macro packages\\
%        (Initialisation)\thanks
%       {This file has version number
%       \fileversion\ dated \filedate}}
%
% \author{Frank Mittelbach \and Rainer Sch\"opf}
%
% \maketitle
%
% This file contains the top level \LaTeX\ interface to the font
% selection scheme commands. See other parts of the \LaTeX\
% distribution, or \emph{The \LaTeX\ Companion} for higher level
% documentation of these commands.
%
% \StopEventually{}
%
%
% \changes{v3.0b}{1995/06/28}
%      {(DPC) Fix documentation typos}
% \changes{v3.0a}{1995/05/24}
%      {(DPC) Make file from previous file, lfonts.dtx 1995/05/23 v2.2e}
%
%
%
% \section{NFSS Initialisation}
%
% \iffalse
%<+checkmem>\CHECKMEM
% \fi
%
%
% \changes{v2.2a}{1994/10/14}{New coding for cfg files}
% \changes{v2.1a}{1993/12/03}{update for LaTeX2e}
% \changes{v1.2c}{1992/01/06}{added slitex code}
% \changes{v1.2d}{1992/03/21}{Renamed \cs{text} to \cs{nfss@text}
%                            to make it internal.}
% \changes{v1.2a}{1991/11/27}{All \cs{family}, \cs{shape} etc. renamed
%                        to \cs{fontfamily} etc.}
% \changes{v1.1i}{1990/04/02}{\cs{input} of files now handled
%                          by docstrip.}
% \changes{v1.1g}{1990/02/08}{Protected the commands 
%         \cs{family}, \cs{series},
%         \cs{shape}, \cs{size}, \cs{selectfont}, and \cs{mathversion}.}
% \changes{v1.1c}{1989/12/03}{Some internal macros renamed to make them
%           inaccessible.}
% \changes{v1.1b}{1989/12/02}{\cs{rmmath} renamed to \cs{mathrm}}
%
% \changes{v1.0i}{1989/11/07}{All family, series, and shape names
%           abbreviated.}
% \changes{v1.0g}{1989/05/22}{Lines shortened to 72 characters}
% \changes{v1.0f}{1989/04/29}{Corrections to \LaTeX\ tabular env.
%                         added.}
% \changes{v1.0e}{1989/04/27}{Definitions of \LaTeX\ symbols corrected.}
% \changes{v1.0d}{1989/04/26}{\cs{xpt} added.}
% \changes{v1.0c}{1989/04/21}{Changed to conform to fam.tex.}
% \changes{v1.0b}{1989/04/15}{\cs{mathfontset} renamed to
%                                              \cs{mathversion.}}
% \changes{v1.0a}{1989/04/10}{Starting with version numbers!
%           \cs{newif} for \cs{@tempswa} added since this switch is
%           unkown at the time when this file is read in.
%           (latex.tex is loaded later.)
%           \cs{math@famname} changed to \cs{math@version.}}
%
%
% \changes{v2.1j}{1994/05/13}{Removed file identification typeout}
%
% Finally, there are six commands that are to be used in \LaTeX{}
% and that we will therefore protect against expansion at the
% wrong point:
% |\fontfamily|, |\fontseries|, |\fontshape|, |\fontsize|,
% |\selectfont|, and |\mathversion|.
%
% \changes{v2.1i}{1994/05/12}{Moved \cs{fontfamily} to fam.dtx}
% \changes{v2.1i}{1994/05/12}{Moved \cs{fontencoding} to fam.dtx}
% \changes{v2.1i}{1994/05/12}{Moved \cs{fontseries} to fam.dtx}
% \changes{v2.1i}{1994/05/12}{Moved \cs{fontshape} to fam.dtx}
% \changes{v2.1i}{1994/05/12}{Moved \cs{fontsize} to fam.dtx}
% \changes{v2.1i}{1994/05/12}{Moved \cs{mathversion} to fam.dtx}
% \changes{v2.1i}{1994/05/12}{Moved \cs{selectfont} to tracefnt.dtx}
%
%
% 
% \subsection{Providing math \emph{versions}}
%
%  \LaTeX{} provides two \emph{versions}. We call them
%  \textsf{normal} and \textsf{bold}, respectively.
%    \begin{macrocode}
\DeclareMathVersion{normal}
\DeclareMathVersion{bold}
%    \end{macrocode}
%
%
%    Now we define the standard font change commands.
%    We don't allow the use of |\rmfamily| etc.\ in math mode.
%
%    First the changes to another \emph{family}:
%    \begin{macrocode}
\DeclareRobustCommand\rmfamily
        {\not@math@alphabet\rmfamily\mathrm
         \fontfamily\rmdefault\selectfont}
\DeclareRobustCommand\sffamily
        {\not@math@alphabet\sffamily\mathsf
         \fontfamily\sfdefault\selectfont}
\DeclareRobustCommand\ttfamily
        {\not@math@alphabet\ttfamily\mathtt
         \fontfamily\ttdefault\selectfont}
%    \end{macrocode}
%    Then the commands changing the \emph{series}:
%    \begin{macrocode}
\DeclareRobustCommand\bfseries
        {\not@math@alphabet\bfseries\mathbf
         \fontseries\bfdefault\selectfont}
\DeclareRobustCommand\mdseries
        {\not@math@alphabet\mdseries\relax
         \fontseries\mddefault\selectfont}
\DeclareRobustCommand\upshape
        {\not@math@alphabet\upshape\relax
         \fontshape\updefault\selectfont}
%    \end{macrocode}
%    Then the commands changing the \emph{shape}:
%    \begin{macrocode}
\DeclareRobustCommand\slshape
        {\not@math@alphabet\slshape\relax
         \fontshape\sldefault\selectfont}
\DeclareRobustCommand\scshape
        {\not@math@alphabet\scshape\relax
         \fontshape\scdefault\selectfont}
\DeclareRobustCommand\itshape
        {\not@math@alphabet\itshape\mathit
         \fontshape\itdefault\selectfont}
%    \end{macrocode}
%
%
%
% We also have to define the {\em emphasize\/} font change command
% (i.e.\ |\em|). This command will look is the current font is
% sloped (i.e.\ has a positive |\fontdimen1|) and will then
% select either |\upshape| or |\itshape|. 
% \changes{v1.2b}{1990/01/28}{Call to `@nomath added.}
%    \begin{macrocode}
\DeclareRobustCommand\em
        {\@nomath\em \ifdim \fontdimen\@ne\font >\z@ 
                       \upshape \else \itshape \fi}
%    \end{macrocode}
%
%
%  \begin{macro}{\not@math@alphabet}
%    This function generates an error message when it is called in
%    math mode. The same function should be defined in 
%    \texttt{newlfont.sty}.
% \changes{v1.4d}{1992/09/21}{Macro defined.}
% \changes{v2.1e}{1994/01/17}{Message changed}
% \changes{v2.1f}{1994/01/18}{Message corrected}
% \changes{v2.1g}{1994/04/22}{Message changed again}
% \changes{v2.2d}{1995/04/02}{add `noexpand to second part of message}
%    \begin{macrocode}
\def\not@math@alphabet#1#2{%
   \relax
   \ifmmode
     \@latex@error{Command \noexpand#1invalid in math mode}%
        {%
         Please 
         \ifx#2\relax
            define a new math alphabet^^J%
            if you want to use a special font in math mode%
          \else
%    \end{macrocode}
%    We have to a |\noexpand| below to prevent expansion of |#2|. In
%    case of |#1| we can omit this (due to the current definition of
%    robust commands since they do come out right there :-).
%    \begin{macrocode}
            use the math alphabet \noexpand#2instead of
            the #1command%
         \fi
         .
        }%
   \fi}
%    \end{macrocode}
%  \end{macro}
%
%
%
% Finally we provide two abbreviations to switch to the \LaTeX{} 
% \emph{versions}.
%    \begin{macrocode}
\def\boldmath{\@nomath\boldmath
              \mathversion{bold}}
\def\unboldmath{\@nomath\unboldmath
              \mathversion{normal}}
%    \end{macrocode}
% Here we switch to the default math version by defining the internal
% macro |\math@version|. We dare not to call |\mathversion|
% at this place because this would call |\glb@settings|.
%    \begin{macrocode}
\def\math@version{normal}
%    \end{macrocode}
%
% \subsection{Miscellaneous}
%
% \begin{macro}{\newfont}
% \changes{v1.2g}{1991/03/30}{Definition added.}
% \changes{v2.2e}{1995/05/23}{Font assignment made local again.}
% \begin{macro}{\symbol}
% \changes{v1.2g}{1991/03/30}{Definition added.}
%    We start by defining a few macros that are part of
%    standard \LaTeX's user interface. The use of these functions is
%    not encouraged, but they will allow to process older documents
%    without changes to the source.
%    \begin{macrocode}
\def\newfont#1#2{\@ifdefinable#1{\font#1=#2\relax}}
\def\symbol#1{\char #1\relax}
%    \end{macrocode}
% \end{macro}
% \end{macro}
%
% \begin{macro}{\@setfontsize}
% \begin{macro}{\@setsize}
%    This abbreviation is used by \LaTeX's user level size changing
%    commands, such as |\large|.
%    \begin{macrocode}
\def\@setfontsize#1#2#3{\@nomath#1%
%    \end{macrocode}
%    For the benefit of people relying on keeping the name of the
%    current font command saved in |\@currsize| we define it. To ensure
%    that |\@setfontsize| keeps being robust we omit this assignment
%    during times where |\protect| differs from |\@typeset@protect|.
% \changes{v1.4b}{1992/08/20}{Added `@currsize.}
% \changes{v2.2b}{1994/11/06}{Use \cs{@typeset@protect}}
%    \begin{macrocode}
    \ifx\protect\@typeset@protect
      \let\@currsize#1%
    \fi
    \fontsize{#2}{#3}\selectfont}
%    \end{macrocode}
%    For compatibility  we also define |\@setsize| the 209 command
%    \begin{macrocode}
%<*compat>
\def\@setsize#1#2#3#4{\@setfontsize#1{#4}{#2}}
%</compat>
%    \end{macrocode}
% \end{macro}
% \end{macro}
%
%
%  \begin{macro}{\oldstylenums}
%    This macro implements old style numerals but only works if we
%    assume that the standard math fonts are used. Thus it needs
%    changing in case other math encodings are used.
%    \begin{macrocode}
\def\oldstylenums#1{%
   \begingroup
%    \end{macrocode}
%    Provide spacing using the interword space of the current font.
%    \begin{macrocode}
    \spaceskip\fontdimen\tw@\font
%    \end{macrocode}
%    Then switch to the math italic font. We don't change the current
%    value of |\f@series| which means that you can use bold numerals
%    if |\bfseries| is in force (For this there must be a substitution
%    of |OML/cmm/bx/it| into |OML/cmm/b/it| in the corresponding |.fd|
%    files.
%    \begin{macrocode}
    \usefont{OML}{cmm}{\f@series}{it}%
    \mathgroup\symletters #1%
   \endgroup
}
%    \end{macrocode}
%  \end{macro}
%
% \begin{macro}{\hexnumber@}
%    To set up \LaTeX's special math character
%    definitions we first provide a macro to generate hexadecimal
%    numbers.  It is a rather simple |\ifcase|.
% \changes{v?}{1992/11/13}{Made expandable.}
% \changes{v?}{1992/12/03}{Make it accept counters.}
%    \begin{macrocode}
\def\hexnumber@#1{\ifcase\number#1 
 0\or 1\or 2\or 3\or 4\or 5\or 6\or 7\or 8\or
 9\or A\or B\or C\or D\or E\or F\fi}
%    \end{macrocode}
%  \end{macro}
%
%
%
% \begin{macro}{\nfss@text}
% \changes{v1.1e}{1990/01/25}{Macro added.}
%    In it simplest form |\nfss@text| is an |\mbox|.  This will
%    produce unbreakable text outside math and inside math you will
%    get text with the same fonts as outside. The only drawback is
%    that such item won't change sizes in subscripts. But this
%    behavior can be easily changed. With the \texttt{amstex} style
%    option one will get a sub style called \texttt{amstext} which will
%    redefine the |\nfss@text| macro to produce correct text in all
%    sizes.
%    
%    We have to use |\def| instead of the shorter |\let| since
%    |\mbox| is undefined when we reach this point.
% \changes{v1.1k}{1990/06/23}{Changed to `mbox.}
% \changes{v2.1n}{1994/05/17}{Added braces to allow use in subscripts}
%    \begin{macrocode}
\def\nfss@text#1{{\mbox{#1}}}
%    \end{macrocode}
% \end{macro}
%
% \begin{macro}{\copyright}
%    The definition of |\copyright| was changed so
%    that it works in other type styles,
%    and to make it robust. We leave the family untouched so that
%    the copyright notice will come out differently if a different
%    font family is in use.
%    This command is commented out, since it is now defined in
%    ltoutenc.dtx. 
% \changes{v1.1m}{1991/03/28}{Extra braces added.}
% \changes{v2.1n}{1994/05/17}{Really add extra braces}
% \changes{v2.2c}{1994/12/02}{\cs{copyright} is now in ltoutenc.
%    ASAJ} 
%    \begin{macrocode}
%\DeclareRobustCommand\copyright
%    {{\ooalign{\hfil
%     \raise.07ex\hbox{\mdseries\upshape c}\hfil\crcr
%     \mathhexbox20D}}}
%    \end{macrocode}
% \end{macro}
%
% \changes{v2.1a}{1993/11/24}{Removed `xpt stuff}
%
%
% \begin{macro}{\normalfont}
% \changes{v2.1a}{1993/11/11}{Macro added}
% \begin{macro}{\reset@font}
% \begin{macro}{\p@reset@font}
% \changes{v1.1n}{1991/08/26}{Macro introduced}
%    The macro |\reset@font| is used in \LaTeX{} to switch to a standard
%    font, in order to initialize the current font in situations where
%    typesetting is done in a new visual context (e.g.\ in a
%    footnote). We define it here to allow the test for the new
%    \LaTeX{} version above but nevertheless are able to run all kind
%    of mixtures.
% \changes{v1.1o}{1991/11/21}{Changed to protected version of macro.}
% \changes{v1.1o}{1991/11/21}{Added extra braces for robustness.}
%
%    The user interface name for |\reset@font| is |\normalfont|:
% \changes{v2.1k}{1994/05/14}{Remove surplus braces}
% \changes{v3.0f}{1995/10/16}{Added \cs{relax} after \cs{usefont},
%              as the latter eats up spaces.}
%    \begin{macrocode}
\DeclareRobustCommand\normalfont
                 {\usefont\encodingdefault
                          \familydefault
                          \seriesdefault
                          \shapedefault
                  \relax}
\let\reset@font\normalfont
%    \end{macrocode}
% \end{macro}
% \end{macro}
% \end{macro}
%
%
%
% We left out the special \LaTeX{} fonts which are not automatically
% included in the base version of the font selection since these fonts
% contain only a few characters which are also included in the AMS
% fonts so anybody who is using these fonts doesn't need them.
% But for compatibility reasons we will define these symbols.
%
% \changes{v2.1g}{1994/02/22}{Correct error mssage}
%    \begin{macrocode}
\def\not@base#1{\@latex@error
  {Command \noexpand#1not provided in base LaTeX2e}%
  {Load the latexsym or the amsfonts package to
   define this symbol}}
\def\mho{\not@base\mho}
\def\Join{\not@base\Join}
\def\Box{\not@base\Box}
\def\Diamond{\not@base\Diamond}
\def\leadsto{\not@base\leadsto}
\def\sqsubset{\not@base\sqsubset}
\def\sqsupset{\not@base\sqsupset}
\def\lhd{\not@base\lhd}
\def\unlhd{\not@base\unlhd}
\def\rhd{\not@base\rhd}
\def\unrhd{\not@base\unrhd}
%    \end{macrocode}
%
%
%
%    We now initialize all variables set by |\DeclareErrorFont|. These
%    values are not really important since they will be overwritten
%    later on by the definition in |fontdef.ltx|.
%
%    However, if \texttt{fontdef.cfg} is corrupted then at least a
%    hopefully suitable error font is present.
%
% \changes{v2.1k}{1994/05/14}{Init error font just before checking for
%                             fontdef.cfg}
%    \begin{macrocode}
\DeclareErrorFont{OT1}{cmr}{m}{n}{10}  %%% don't modify this setting
                                       %%% overwrite it in fontdef.cfg
                                       %%% if necessary
%    \end{macrocode}
%
%
%
%
% We now load the customizable parts of NFSS.
% \changes{v3.0d}{1995/07/19}
%      {(DPC) TeX2 support}
% \changes{v3.0e}{1995/09/15}
%      {(DPC) Modify TeX2 message}
% \changes{v3.0g}{1995/11/01}
%      {(DPC) Switch meaning of \cs{@addtofilelist} for cfg files}
% \changes{v3.0h}{1996/12/06}
%      {(DPC) Remove *** from messages internal/2338}
%    \begin{macrocode}
\ifnum\inputlineno=\m@ne
%    \end{macrocode}
% Still using \TeX2. need a configuration file to avoid setting the 8bit
% characters.
%    \begin{macrocode}
\InputIfFileExists{fonttext.cfg}
           {\typeout{====================================^^J%
                     ^^J%
                      Local config file fonttext.cfg used^^J%
                     ^^J%
                     ====================================}%
             \def\@addtofilelist##1{\xdef\@filelist{\@filelist,##1}}%
            }
           {\typeout{!!!!!!!!!!!!!!!!!!!!!!!!!!!!!!!!!!!!!^^J%
                     !^^J%
                     ! You MUST use a fonttext.cfg file!^^J%
                     ! As you are still using TeX2!!!!!^^J%
                     !^^J%
                     ! See the documentation file tex2.txt^^J%
                     !^^J%
                     !!!!!!!!!!!!!!!!!!!!!!!!!!!!!!!!!!!!!}%
                    \batchmode \@@end}
\else
%    \end{macrocode}
% With \TeX3 can use the standard |ltx| file if no configuration file
% exists.
%    \begin{macrocode}
\InputIfFileExists{fonttext.cfg}
           {\typeout{====================================^^J%
                     ^^J%
                      Local config file fonttext.cfg used^^J%
                     ^^J%
                     ====================================}%
             \def\@addtofilelist##1{\xdef\@filelist{\@filelist,##1}}%
            }
           {%%
%% This is file `fonttext.cfg',
%% generated with the docstrip utility.
%%
%% The original source files were:
%%
%% cslatex.dtx  (with options: `fonttext')
%% 
%% Tento soubor je soucasti baliku CsLaTeX a je vygenerovan z
%% distribucniho souboru cslatex.dtx.
%% 
%% Copyright (c) 1994--98, 2002
%% Jaroslav Snajdr, Zdenek Wagner, Jiri Zlatuska a The LaTeX3 Project
%% 
%% Tento soubor NENI soucasti systemu LaTeX2e
%% ------------------------------------------
%% 
%% Dalsi informace naleznete na www.cstug.cz/cslatex.
%% 
\ProvidesFile{fonttext.cfg}[1997/08/20 CSLaTeX]
\let\@hyphenation=\relax
\def\DeclareFontEncoding@#1#2#3{%
  \expandafter
  \ifx\csname T@#1\endcsname\relax
     \def\cdp@elt{\noexpand\cdp@elt}%
     \xdef\cdp@list{\cdp@list\cdp@elt{#1}%
                    {\default@family}{\default@series}%
                    {\default@shape}}%
     \expandafter\let\csname#1-cmd\endcsname\@changed@cmd
  \else
     \@font@info{Redeclaring font encoding #1}%
  \fi
  \global\@namedef{T@#1}{#2\csname @@#1Codes\endcsname\@hyphenation}%
  \global\@namedef{M@#1}{\default@M#3}%
}
%%
%% This is file `omlenc.def',
%% generated with the docstrip utility.
%%
%% The original source files were:
%%
%% ltoutenc.dtx  (with options: `OML')
%% 
%% This is a generated file.
%% 
%% Copyright 1993 1994 1995 1996 1997
%% The LaTeX3 Project and any individual authors listed elsewhere
%% in this file.
%% 
%% For further copyright information, and conditions for modification
%% and distribution, see the file legal.txt, and any other copyright
%% notices in this file.
%% 
%% This file is part of the LaTeX2e system.
%% ----------------------------------------
%%   This system is distributed in the hope that it will be useful,
%%   but WITHOUT ANY WARRANTY; without even the implied warranty of
%%   MERCHANTABILITY or FITNESS FOR A PARTICULAR PURPOSE.
%% 
%%   For error reports concerning UNCHANGED versions of this file no
%%   more than one year old, see bugs.txt.
%% 
%%   Please do not request updates from us directly.  Primary
%%   distribution is through the CTAN archives.
%% 
%% 
%% IMPORTANT COPYRIGHT NOTICE:
%% 
%% You are NOT ALLOWED to distribute this file alone.
%% 
%% You are allowed to distribute this file under the condition that it
%% is distributed together with all the files listed in manifest.txt.
%% 
%% If you receive only some of these files from someone, complain!
%% 
%% 
%% Permission is granted to copy this file to another file with a
%% clearly different name and to customize the declarations in that
%% copy to serve the needs of your installation, provided that you
%% comply with the conditions in the file legal.txt.
%% 
%% However, NO PERMISSION is granted to generate or to distribute a
%% modified version of this file under its original name.
%% 
%% You are NOT ALLOWED to change this file.
%% 
%% 
%% MODIFICATION ADVICE:
%% 
%% If you want to customize this file, it is best to make a copy of
%% the source file(s) from which it was produced.  Use a different
%% name for your copy(ies) and modify the copy(ies); this will ensure
%% that your modifications do not get overwritten when you install a
%% new release of the standard system.  You should also ensure that
%% your modified source file does not generate any modified file with
%% the same name as a standard file.
%% 
%% You can then easily distribute your modifications by distributing
%% the modified and renamed copy of the source file, taking care to
%% observe the conditions in legal.txt; this will ensure that other
%% users can safely use your modifications.
%% 
%% You will also need to produce your own, suitably named, .ins file to
%% control the generation of files from your source file; this file
%% should contain your own preambles for the files it generates, not
%% those in the standard .ins files.
%% 
%% The names of the source files used are shown above.
%% 
%% 
%% 
%%% From File: ltoutenc.dtx
\ProvidesFile{omlenc.def}
 [1998/06/12 v1.9p
         Standard LaTeX file]
\DeclareFontEncoding{OML}{}{}
\DeclareTextSymbol{\textless}{OML}{`\<}
\DeclareTextSymbol{\textgreater}{OML}{`\>}
\DeclareTextAccent{\t}{OML}{127}  % "7F
\endinput
%%
%% End of file `omlenc.def'.

%%
%% This is file `t1enc.def',
%% generated with the docstrip utility.
%%
%% The original source files were:
%%
%% ltoutenc.dtx  (with options: `T1')
%% 
%% This is a generated file.
%% 
%% Copyright 1993 1994 1995 1996 1997
%% The LaTeX3 Project and any individual authors listed elsewhere
%% in this file.
%% 
%% For further copyright information, and conditions for modification
%% and distribution, see the file legal.txt, and any other copyright
%% notices in this file.
%% 
%% This file is part of the LaTeX2e system.
%% ----------------------------------------
%%   This system is distributed in the hope that it will be useful,
%%   but WITHOUT ANY WARRANTY; without even the implied warranty of
%%   MERCHANTABILITY or FITNESS FOR A PARTICULAR PURPOSE.
%% 
%%   For error reports concerning UNCHANGED versions of this file no
%%   more than one year old, see bugs.txt.
%% 
%%   Please do not request updates from us directly.  Primary
%%   distribution is through the CTAN archives.
%% 
%% 
%% IMPORTANT COPYRIGHT NOTICE:
%% 
%% You are NOT ALLOWED to distribute this file alone.
%% 
%% You are allowed to distribute this file under the condition that it
%% is distributed together with all the files listed in manifest.txt.
%% 
%% If you receive only some of these files from someone, complain!
%% 
%% 
%% Permission is granted to copy this file to another file with a
%% clearly different name and to customize the declarations in that
%% copy to serve the needs of your installation, provided that you
%% comply with the conditions in the file legal.txt.
%% 
%% However, NO PERMISSION is granted to generate or to distribute a
%% modified version of this file under its original name.
%% 
%% You are NOT ALLOWED to change this file.
%% 
%% 
%% MODIFICATION ADVICE:
%% 
%% If you want to customize this file, it is best to make a copy of
%% the source file(s) from which it was produced.  Use a different
%% name for your copy(ies) and modify the copy(ies); this will ensure
%% that your modifications do not get overwritten when you install a
%% new release of the standard system.  You should also ensure that
%% your modified source file does not generate any modified file with
%% the same name as a standard file.
%% 
%% You can then easily distribute your modifications by distributing
%% the modified and renamed copy of the source file, taking care to
%% observe the conditions in legal.txt; this will ensure that other
%% users can safely use your modifications.
%% 
%% You will also need to produce your own, suitably named, .ins file to
%% control the generation of files from your source file; this file
%% should contain your own preambles for the files it generates, not
%% those in the standard .ins files.
%% 
%% The names of the source files used are shown above.
%% 
%% 
%% 
%%% From File: ltoutenc.dtx
\ProvidesFile{t1enc.def}
 [1998/06/12 v1.9p
         Standard LaTeX file]
\DeclareFontEncoding{T1}{}{}
\DeclareTextAccent{\`}{T1}{0}
\DeclareTextAccent{\'}{T1}{1}
\DeclareTextAccent{\^}{T1}{2}
\DeclareTextAccent{\~}{T1}{3}
\DeclareTextAccent{\"}{T1}{4}
\DeclareTextAccent{\H}{T1}{5}
\DeclareTextAccent{\r}{T1}{6}
\DeclareTextAccent{\v}{T1}{7}
\DeclareTextAccent{\u}{T1}{8}
\DeclareTextAccent{\=}{T1}{9}
\DeclareTextAccent{\.}{T1}{10}
\DeclareTextCommand{\b}{T1}[1]
   {{\o@lign{\relax#1\crcr\hidewidth\sh@ft{29}%
     \vbox to.2ex{\hbox{\char9}\vss}\hidewidth}}}
\DeclareTextCommand{\c}{T1}[1]
   {\leavevmode\setbox\z@\hbox{#1}\ifdim\ht\z@=1ex\accent11 #1%
     \else{\ooalign{\hidewidth\char11\hidewidth
        \crcr\unhbox\z@}}\fi}
\DeclareTextCommand{\d}{T1}[1]
   {{\o@lign{\relax#1\crcr\hidewidth\sh@ft{10}.\hidewidth}}}
\DeclareTextCommand{\k}{T1}[1]
   {\oalign{\null#1\crcr\hidewidth\char12}}
\DeclareTextCommand{\textperthousand}{T1}
   {\%\char 24 }          % space or `relax as delimiter?
\DeclareTextCommand{\textpertenthousand}{T1}
   {\%\char 24\char 24 }  % space or `relax as delimiter?
\DeclareTextSymbol{\AE}{T1}{198}
\DeclareTextSymbol{\DH}{T1}{208}
\DeclareTextSymbol{\DJ}{T1}{208}
\DeclareTextSymbol{\L}{T1}{138}
\DeclareTextSymbol{\NG}{T1}{141}
\DeclareTextSymbol{\OE}{T1}{215}
\DeclareTextSymbol{\O}{T1}{216}
\DeclareTextSymbol{\SS}{T1}{223}
\DeclareTextSymbol{\TH}{T1}{222}
\DeclareTextSymbol{\ae}{T1}{230}
\DeclareTextSymbol{\dh}{T1}{240}
\DeclareTextSymbol{\dj}{T1}{158}
\DeclareTextSymbol{\guillemotleft}{T1}{19}
\DeclareTextSymbol{\guillemotright}{T1}{20}
\DeclareTextSymbol{\guilsinglleft}{T1}{14}
\DeclareTextSymbol{\guilsinglright}{T1}{15}
\DeclareTextSymbol{\i}{T1}{25}
\DeclareTextSymbol{\j}{T1}{26}
\DeclareTextSymbol{\l}{T1}{170}
\DeclareTextSymbol{\ng}{T1}{173}
\DeclareTextSymbol{\oe}{T1}{247}
\DeclareTextSymbol{\o}{T1}{248}
\DeclareTextSymbol{\quotedblbase}{T1}{18}
\DeclareTextSymbol{\quotesinglbase}{T1}{13}
\DeclareTextSymbol{\ss}{T1}{255}
\DeclareTextSymbol{\textasciicircum}{T1}{`\^}
\DeclareTextSymbol{\textasciitilde}{T1}{`\~}
\DeclareTextSymbol{\textbackslash}{T1}{`\\}
\DeclareTextSymbol{\textbar}{T1}{`\|}
\DeclareTextSymbol{\textbraceleft}{T1}{`\{}
\DeclareTextSymbol{\textbraceright}{T1}{`\}}
\DeclareTextSymbol{\textcompwordmark}{T1}{23}
\DeclareTextSymbol{\textdollar}{T1}{`\$}
\DeclareTextSymbol{\textemdash}{T1}{22}
\DeclareTextSymbol{\textendash}{T1}{21}
\DeclareTextSymbol{\textexclamdown}{T1}{189}
\DeclareTextSymbol{\textgreater}{T1}{`\>}
\DeclareTextSymbol{\textless}{T1}{`\<}
\DeclareTextSymbol{\textquestiondown}{T1}{190}
\DeclareTextSymbol{\textquotedblleft}{T1}{16}
\DeclareTextSymbol{\textquotedblright}{T1}{17}
\DeclareTextSymbol{\textquotedbl}{T1}{`\"}
\DeclareTextSymbol{\textquoteleft}{T1}{`\`}
\DeclareTextSymbol{\textquoteright}{T1}{`\'}
\DeclareTextSymbol{\textsection}{T1}{159}
\DeclareTextSymbol{\textsterling}{T1}{191}
\DeclareTextSymbol{\textunderscore}{T1}{95}
\DeclareTextSymbol{\textvisiblespace}{T1}{32}
\DeclareTextSymbol{\th}{T1}{254}
\DeclareTextComposite{\.}{T1}{i}{`\i}
\DeclareTextComposite{\u}{T1}{A}{128}
\DeclareTextComposite{\k}{T1}{A}{129}
\DeclareTextComposite{\'}{T1}{C}{130}
\DeclareTextComposite{\v}{T1}{C}{131}
\DeclareTextComposite{\v}{T1}{D}{132}
\DeclareTextComposite{\v}{T1}{E}{133}
\DeclareTextComposite{\k}{T1}{E}{134}
\DeclareTextComposite{\u}{T1}{G}{135}
\DeclareTextComposite{\'}{T1}{L}{136}
\DeclareTextComposite{\v}{T1}{L}{137}
\DeclareTextComposite{\'}{T1}{N}{139}
\DeclareTextComposite{\v}{T1}{N}{140}
\DeclareTextComposite{\H}{T1}{O}{142}
\DeclareTextComposite{\'}{T1}{R}{143}
\DeclareTextComposite{\v}{T1}{R}{144}
\DeclareTextComposite{\'}{T1}{S}{145}
\DeclareTextComposite{\v}{T1}{S}{146}
\DeclareTextComposite{\c}{T1}{S}{147}
\DeclareTextComposite{\v}{T1}{T}{148}
\DeclareTextComposite{\c}{T1}{T}{149}
\DeclareTextComposite{\H}{T1}{U}{150}
\DeclareTextComposite{\r}{T1}{U}{151}
\DeclareTextComposite{\"}{T1}{Y}{152}
\DeclareTextComposite{\'}{T1}{Z}{153}
\DeclareTextComposite{\v}{T1}{Z}{154}
\DeclareTextComposite{\.}{T1}{Z}{155}
\DeclareTextComposite{\.}{T1}{I}{157}
\DeclareTextComposite{\u}{T1}{a}{160}
\DeclareTextComposite{\k}{T1}{a}{161}
\DeclareTextComposite{\'}{T1}{c}{162}
\DeclareTextComposite{\v}{T1}{c}{163}
\DeclareTextComposite{\v}{T1}{d}{164}
\DeclareTextComposite{\v}{T1}{e}{165}
\DeclareTextComposite{\k}{T1}{e}{166}
\DeclareTextComposite{\u}{T1}{g}{167}
\DeclareTextComposite{\'}{T1}{l}{168}
\DeclareTextComposite{\v}{T1}{l}{169}
\DeclareTextComposite{\'}{T1}{n}{171}
\DeclareTextComposite{\v}{T1}{n}{172}
\DeclareTextComposite{\H}{T1}{o}{174}
\DeclareTextComposite{\'}{T1}{r}{175}
\DeclareTextComposite{\v}{T1}{r}{176}
\DeclareTextComposite{\'}{T1}{s}{177}
\DeclareTextComposite{\v}{T1}{s}{178}
\DeclareTextComposite{\c}{T1}{s}{179}
\DeclareTextComposite{\v}{T1}{t}{180}
\DeclareTextComposite{\c}{T1}{t}{181}
\DeclareTextComposite{\H}{T1}{u}{182}
\DeclareTextComposite{\r}{T1}{u}{183}
\DeclareTextComposite{\"}{T1}{y}{184}
\DeclareTextComposite{\'}{T1}{z}{185}
\DeclareTextComposite{\v}{T1}{z}{186}
\DeclareTextComposite{\.}{T1}{z}{187}
\DeclareTextComposite{\`}{T1}{A}{192}
\DeclareTextComposite{\'}{T1}{A}{193}
\DeclareTextComposite{\^}{T1}{A}{194}
\DeclareTextComposite{\~}{T1}{A}{195}
\DeclareTextComposite{\"}{T1}{A}{196}
\DeclareTextComposite{\r}{T1}{A}{197}
\DeclareTextComposite{\c}{T1}{C}{199}
\DeclareTextComposite{\`}{T1}{E}{200}
\DeclareTextComposite{\'}{T1}{E}{201}
\DeclareTextComposite{\^}{T1}{E}{202}
\DeclareTextComposite{\"}{T1}{E}{203}
\DeclareTextComposite{\`}{T1}{I}{204}
\DeclareTextComposite{\'}{T1}{I}{205}
\DeclareTextComposite{\^}{T1}{I}{206}
\DeclareTextComposite{\"}{T1}{I}{207}
\DeclareTextComposite{\~}{T1}{N}{209}
\DeclareTextComposite{\`}{T1}{O}{210}
\DeclareTextComposite{\'}{T1}{O}{211}
\DeclareTextComposite{\^}{T1}{O}{212}
\DeclareTextComposite{\~}{T1}{O}{213}
\DeclareTextComposite{\"}{T1}{O}{214}
\DeclareTextComposite{\`}{T1}{U}{217}
\DeclareTextComposite{\'}{T1}{U}{218}
\DeclareTextComposite{\^}{T1}{U}{219}
\DeclareTextComposite{\"}{T1}{U}{220}
\DeclareTextComposite{\'}{T1}{Y}{221}
\DeclareTextComposite{\`}{T1}{a}{224}
\DeclareTextComposite{\'}{T1}{a}{225}
\DeclareTextComposite{\^}{T1}{a}{226}
\DeclareTextComposite{\~}{T1}{a}{227}
\DeclareTextComposite{\"}{T1}{a}{228}
\DeclareTextComposite{\r}{T1}{a}{229}
\DeclareTextComposite{\c}{T1}{c}{231}
\DeclareTextComposite{\`}{T1}{e}{232}
\DeclareTextComposite{\'}{T1}{e}{233}
\DeclareTextComposite{\^}{T1}{e}{234}
\DeclareTextComposite{\"}{T1}{e}{235}
\DeclareTextComposite{\`}{T1}{i}{236}
\DeclareTextComposite{\`}{T1}{\i}{236}
\DeclareTextComposite{\'}{T1}{i}{237}
\DeclareTextComposite{\'}{T1}{\i}{237}
\DeclareTextComposite{\^}{T1}{i}{238}
\DeclareTextComposite{\^}{T1}{\i}{238}
\DeclareTextComposite{\"}{T1}{i}{239}
\DeclareTextComposite{\"}{T1}{\i}{239}
\DeclareTextComposite{\~}{T1}{n}{241}
\DeclareTextComposite{\`}{T1}{o}{242}
\DeclareTextComposite{\'}{T1}{o}{243}
\DeclareTextComposite{\^}{T1}{o}{244}
\DeclareTextComposite{\~}{T1}{o}{245}
\DeclareTextComposite{\"}{T1}{o}{246}
\DeclareTextComposite{\`}{T1}{u}{249}
\DeclareTextComposite{\'}{T1}{u}{250}
\DeclareTextComposite{\^}{T1}{u}{251}
\DeclareTextComposite{\"}{T1}{u}{252}
\DeclareTextComposite{\'}{T1}{y}{253}
\endinput
%%
%% End of file `t1enc.def'.

%%
%% This is file `ot1enc.def',
%% generated with the docstrip utility.
%%
%% The original source files were:
%%
%% ltoutenc.dtx  (with options: `OT1')
%% 
%% This is a generated file.
%% 
%% Copyright 1993 1994 1995 1996 1997 1998 1999
%% The LaTeX3 Project and any individual authors listed elsewhere
%% in this file.
%% 
%% This file is part of the LaTeX2e system.
%% ----------------------------------------
%% 
%% It may be distributed under the terms of the LaTeX Project Public
%% License, as described in lppl.txt in the base LaTeX distribution.
%% Either version 1.0 or, at your option, any later version.
%%% From File: ltoutenc.dtx
\ProvidesFile{ot1enc.def}
 [1999/02/24 v1.9t
         Standard LaTeX file]
\DeclareFontEncoding{OT1}{}{}
\DeclareTextAccent{\"}{OT1}{127}
\DeclareTextAccent{\'}{OT1}{19}
\DeclareTextAccent{\.}{OT1}{95}
\DeclareTextAccent{\=}{OT1}{22}
\DeclareTextAccent{\^}{OT1}{94}
\DeclareTextAccent{\`}{OT1}{18}
\DeclareTextAccent{\~}{OT1}{126}
\DeclareTextAccent{\H}{OT1}{125}
\DeclareTextAccent{\u}{OT1}{21}
\DeclareTextAccent{\v}{OT1}{20}
\DeclareTextAccent{\r}{OT1}{23}
\DeclareTextCommand{\b}{OT1}[1]
   {{\o@lign{\relax#1\crcr\hidewidth\sh@ft{29}%
     \vbox to.2ex{\hbox{\char22}\vss}\hidewidth}}}
\DeclareTextCommand{\c}{OT1}[1]
   {\leavevmode\setbox\z@\hbox{#1}\ifdim\ht\z@=1ex\accent24 #1%
    \else{\ooalign{\unhbox\z@\crcr\hidewidth\char24\hidewidth}}\fi}
\DeclareTextCommand{\d}{OT1}[1]
   {{\o@lign{\relax#1\crcr\hidewidth\sh@ft{10}.\hidewidth}}}
\DeclareTextSymbol{\AE}{OT1}{29}
\DeclareTextSymbol{\OE}{OT1}{30}
\DeclareTextSymbol{\O}{OT1}{31}
\DeclareTextSymbol{\ae}{OT1}{26}
\DeclareTextSymbol{\i}{OT1}{16}
\DeclareTextSymbol{\j}{OT1}{17}
\DeclareTextSymbol{\oe}{OT1}{27}
\DeclareTextSymbol{\o}{OT1}{28}
\DeclareTextSymbol{\ss}{OT1}{25}
\DeclareTextSymbol{\textemdash}{OT1}{124}
\DeclareTextSymbol{\textendash}{OT1}{123}
\DeclareTextSymbol{\textexclamdown}{OT1}{60}
\DeclareTextSymbol{\textquestiondown}{OT1}{62}
\DeclareTextSymbol{\textquotedblleft}{OT1}{92}
\DeclareTextSymbol{\textquotedblright}{OT1}{`\"}
\DeclareTextSymbol{\textquoteleft}{OT1}{`\`}
\DeclareTextSymbol{\textquoteright}{OT1}{`\'}
\DeclareTextCommand{\L}{OT1}
   {\leavevmode\setbox\z@\hbox{L}\hb@xt@\wd\z@{\hss\@xxxii L}}
\DeclareTextCommand{\l}{OT1}
   {{\@xxxii l}}
\DeclareTextCompositeCommand{\r}{OT1}{A}
   {\leavevmode\setbox\z@\hbox{h}\dimen@\ht\z@\advance\dimen@-1ex%
    \rlap{\raise.67\dimen@\hbox{\char23}}A}
\DeclareTextCommand{\textdollar}{OT1}{{%
   \ifdim \fontdimen\@ne\font >\z@
      \slshape
   \else
      \upshape
   \fi
   \char`\$}}
\DeclareTextCommand{\textsterling}{OT1}{{%
   \ifdim \fontdimen\@ne\font >\z@
      \itshape
   \else
      \fontshape{ui}\selectfont
   \fi
   \char`\$}}
\endinput
%%
%% End of file `ot1enc.def'.
       % <- should come after T1 for speed
%%
%% This is file `il2enc.def',
%% generated with the docstrip utility.
%%
%% The original source files were:
%%
%% cslatex.dtx  (with options: `il2enc')
%% 
%% Tento soubor je soucasti baliku CsLaTeX a je vygenerovan z
%% distribucniho souboru cslatex.dtx.
%% 
%% Copyright (c) 1994--98, 2002
%% Jaroslav Snajdr, Zdenek Wagner, Jiri Zlatuska a The LaTeX3 Project
%% 
%% Tento soubor NENI soucasti systemu LaTeX2e
%% ------------------------------------------
%% 
%% Dalsi informace naleznete na www.cstug.cz/cslatex.
%% 
\ProvidesFile{il2enc.def}[1997/08/20 CSLaTeX]
\DeclareFontEncoding{IL2}{\uccode152=152 \lccode152=184
                          \uccode184=152 \lccode184=184
                          \uccode165=165 \lccode165=181
                          \uccode181=165 \lccode181=181
                          \uccode169=169 \lccode169=185
                          \uccode185=169 \lccode185=185
                          \uccode171=171 \lccode171=187
                          \uccode187=171 \lccode187=187
                          \uccode174=174 \lccode174=190
                          \uccode190=174 \lccode190=190
                          \sfcode254=0   \lccode254=0
                          \sfcode255=0   \lccode255=0
                          \sfcode158=0   \lccode158=0
                          \sfcode159=0   \lccode159=0 }{}
\DeclareTextAccent{\"}{IL2}{127}
\DeclareTextAccent{\'}{IL2}{19}
\DeclareTextAccent{\.}{IL2}{95}
\DeclareTextAccent{\=}{IL2}{22}
\DeclareTextAccent{\^}{IL2}{94}
\DeclareTextAccent{\`}{IL2}{18}
\DeclareTextAccent{\~}{IL2}{126}
\DeclareTextAccent{\H}{IL2}{125}
\DeclareTextAccent{\u}{IL2}{21}
\DeclareTextAccent{\v}{IL2}{20}
\DeclareTextAccent{\r}{IL2}{23}
\DeclareTextCommand{\b}{IL2}[1]
   {\oalign{\null#1\crcr\hidewidth\sh@ft{29}%
    \vbox to.2ex{\hbox{\char22}\vss}\hidewidth}}
\DeclareTextCommand{\c}{IL2}[1]
   {\setbox\z@\hbox{#1}\ifdim\ht\z@=1ex\accent24 #1%
    \else{\ooalign{\unhbox\z@\crcr\hidewidth\char24\hidewidth}}\fi}
\DeclareTextCommand{\d}{IL2}[1]
   {\oalign{\null#1\crcr\hidewidth\sh@ft{08}.\hidewidth}}
\DeclareTextSymbol{\AE}{IL2}{29}
\DeclareTextSymbol{\OE}{IL2}{30}
\DeclareTextSymbol{\O}{IL2}{31}
\DeclareTextSymbol{\ae}{IL2}{26}
\DeclareTextSymbol{\i}{IL2}{16}
\DeclareTextSymbol{\j}{IL2}{17}
\DeclareTextSymbol{\oe}{IL2}{27}
\DeclareTextSymbol{\o}{IL2}{28}
\DeclareTextSymbol{\ss}{IL2}{25}
\DeclareTextSymbol{\textemdash}{IL2}{124}
\DeclareTextSymbol{\textendash}{IL2}{123}
\DeclareTextSymbol{\textexclamdown}{IL2}{60}
\DeclareTextSymbol{\textquestiondown}{IL2}{62}
\DeclareTextSymbol{\textquotedblleft}{IL2}{92}
\DeclareTextSymbol{\textquotedblright}{IL2}{`\"}
\DeclareTextSymbol{\textquoteleft}{IL2}{`\`}
\DeclareTextSymbol{\textquoteright}{IL2}{`\'}
\DeclareTextCommand{\L}{IL2}
   {\leavevmode\setbox0\hbox{L}\hbox to\wd0{\hss\char32L}}
\DeclareTextCommand{\l}{IL2}{{\char32l}}
\DeclareTextCompositeCommand{\r}{IL2}{A}
   {\leavevmode\setbox0\hbox{h}\dimen@\ht0\advance\dimen@-1ex%
    \rlap{\raise.67\dimen@\hbox{\char'27}}A}
\DeclareTextCommand{\SS}{IL2}{SS}
\DeclareTextCommand{\textdollar}{IL2}{{%
    \ifdim \fontdimen\@ne\font >\z@
      \slshape
    \else
      \upshape
    \fi
    \char`\$}}
\DeclareTextCommand{\textsterling}{IL2}{{%
    \ifdim \fontdimen\@ne\font >\z@
      \itshape
    \else
      \fontshape{ui}\selectfont
    \fi
    \char`\$}}
\DeclareTextComposite{\'}{IL2}{l}{'345}%
\DeclareTextComposite{\'}{IL2}{r}{'340}%
\DeclareTextComposite{\'}{IL2}{a}{'341}%
\DeclareTextComposite{\'}{IL2}{e}{'351}%
\DeclareTextComposite{\'}{IL2}{\i}{'355}%
\DeclareTextComposite{\'}{IL2}{i}{'355}%
\DeclareTextComposite{\'}{IL2}{o}{'363}%
\DeclareTextComposite{\'}{IL2}{u}{'372}%
\DeclareTextComposite{\'}{IL2}{y}{'375}%
\DeclareTextComposite{\'}{IL2}{L}{'305}%
\DeclareTextComposite{\'}{IL2}{R}{'300}%
\DeclareTextComposite{\'}{IL2}{A}{'301}%
\DeclareTextComposite{\'}{IL2}{E}{'311}%
\DeclareTextComposite{\'}{IL2}{I}{'315}%
\DeclareTextComposite{\'}{IL2}{O}{'323}%
\DeclareTextComposite{\'}{IL2}{U}{'332}%
\DeclareTextComposite{\'}{IL2}{Y}{'335}%
\DeclareTextComposite{\`}{IL2}{a}{'270}%
\DeclareTextComposite{\`}{IL2}{A}{'230}%
\DeclareTextComposite{\v}{IL2}{c}{'350}%
\DeclareTextComposite{\v}{IL2}{d}{'357}%
\DeclareTextComposite{\v}{IL2}{e}{'354}%
\DeclareTextComposite{\v}{IL2}{n}{'362}%
\DeclareTextComposite{\v}{IL2}{r}{'370}%
\DeclareTextComposite{\v}{IL2}{s}{'271}%
\DeclareTextComposite{\v}{IL2}{z}{'276}%
\DeclareTextComposite{\v}{IL2}{t}{'273}%
\DeclareTextComposite{\v}{IL2}{l}{'265}%
\DeclareTextComposite{\v}{IL2}{u}{'371}%
\DeclareTextComposite{\v}{IL2}{C}{'310}%
\DeclareTextComposite{\v}{IL2}{D}{'317}%
\DeclareTextComposite{\v}{IL2}{E}{'314}%
\DeclareTextComposite{\v}{IL2}{N}{'322}%
\DeclareTextComposite{\v}{IL2}{R}{'330}%
\DeclareTextComposite{\v}{IL2}{S}{'251}%
\DeclareTextComposite{\v}{IL2}{T}{'253}%
\DeclareTextComposite{\v}{IL2}{Z}{'256}%
\DeclareTextComposite{\v}{IL2}{L}{'245}%
\DeclareTextComposite{\v}{IL2}{U}{'331}%
\DeclareTextComposite{\^}{IL2}{o}{'364}%
\DeclareTextComposite{\^}{IL2}{O}{'324}%
\DeclareTextComposite{\"}{IL2}{a}{'344}%
\DeclareTextComposite{\"}{IL2}{o}{'366}%
\DeclareTextComposite{\"}{IL2}{u}{'374}%
\DeclareTextComposite{\"}{IL2}{A}{'304}%
\DeclareTextComposite{\"}{IL2}{O}{'326}%
\DeclareTextComposite{\"}{IL2}{U}{'334}%
\DeclareTextComposite{\r}{IL2}{u}{'371}%
\DeclareTextComposite{\r}{IL2}{U}{'331}%
\DeclareTextSymbol{\flqq}{IL2}{158}%
\DeclareTextSymbol{\frqq}{IL2}{159}%
\DeclareTextSymbol{\clqq}{IL2}{254}%
\DeclareTextCommand{\crqq}{IL2}%
  {{\edef\@SF{\spacefactor\the\spacefactor}\char255 \@SF\relax}}%
\ifx\addlanguage\undefined\else
  \ifx\LdfInit\@undefined
    \def\LdfInit{%
      \chardef\atcatcode=\catcode`\@
      \catcode`\@=11\relax
      \input babel.def\relax
      \catcode`\@=\atcatcode \let\atcatcode\relax
      \LdfInit}
  \fi
\endinput\fi
\endinput
%%
%% End of file `il2enc.def'.

%%
%% This is file `omsenc.def',
%% generated with the docstrip utility.
%%
%% The original source files were:
%%
%% ltoutenc.dtx  (with options: `OMS')
%% 
%% This is a generated file.
%% 
%% Copyright 1993 1994 1995 1996 1997 1998 1999 2000 2001
%% The LaTeX3 Project and any individual authors listed elsewhere
%% in this file.
%% 
%% This file was generated from file(s) of the LaTeX base system.
%% --------------------------------------------------------------
%% 
%% It may be distributed and/or modified under the
%% conditions of the LaTeX Project Public License, either version 1.2
%% of this license or (at your option) any later version.
%% The latest version of this license is in
%%    http://www.latex-project.org/lppl.txt
%% and version 1.2 or later is part of all distributions of LaTeX
%% version 1999/12/01 or later.
%% 
%% This file may only be distributed together with a copy of the LaTeX
%% base system. You may however distribute the LaTeX base system without
%% such generated files.
%% 
%% The list of all files belonging to the LaTeX base distribution is
%% given in the file `manifest.txt'. See also `legal.txt' for additional
%% information.
%% 
%%% From File: ltoutenc.dtx
\ProvidesFile{omsenc.def}
 [2001/06/05 v1.94
         Standard LaTeX file]
\DeclareFontEncoding{OMS}{}{}
\DeclareTextSymbol{\textasteriskcentered}{OMS}{3}   % "03
\DeclareTextSymbol{\textbackslash}{OMS}{110}        % "6E
\DeclareTextSymbol{\textbar}{OMS}{106}              % "6A
\DeclareTextSymbol{\textbraceleft}{OMS}{102}        % "66
\DeclareTextSymbol{\textbraceright}{OMS}{103}       % "67
\DeclareTextSymbol{\textbullet}{OMS}{15}            % "0F
\DeclareTextSymbol{\textdaggerdbl}{OMS}{122}        % "7A
\DeclareTextSymbol{\textdagger}{OMS}{121}           % "79
\DeclareTextSymbol{\textparagraph}{OMS}{123}        % "7B
\DeclareTextSymbol{\textperiodcentered}{OMS}{1}     % "01
\DeclareTextSymbol{\textsection}{OMS}{120}          % "78
\DeclareTextCommand{\textcircled}{OMS}[1]{\hmode@bgroup
   \ooalign{%
      \hfil \raise .07ex\hbox {\upshape#1}\hfil \crcr
      \char 13 % "0D
   }%
 \egroup}
\endinput
%%
%% End of file `omsenc.def'.

\fontencoding{OT1}
\DeclareFontEncodingDefaults{}{}
\DeclareFontSubstitution{T1}{cmr}{m}{n}
\DeclareFontSubstitution{OT1}{cmr}{m}{n}
\DeclareFontSubstitution{IL2}{cmr}{m}{n}
\begingroup
\nfss@catcodes
%%
%% This is file `t1cmr.fd',
%% generated with the docstrip utility.
%%
%% The original source files were:
%%
%% cmfonts.fdd  (with options: `fd,T1cmr,ec')
%% 
%% This is a generated file.
%% 
%% Copyright 1993 1994 1995 1996 1997 1998 1999
%% The LaTeX3 Project and any individual authors listed elsewhere
%% in this file.
%% 
%% This file is part of the LaTeX2e system.
%% ----------------------------------------
%% 
%% It may be distributed under the terms of the LaTeX Project Public
%% License, as described in lppl.txt in the base LaTeX distribution.
%% Either version 1.0 or, at your option, any later version.
%% 
%% In particular, permission is granted to customize the declarations in
%% this file to serve the needs of your installation.
%% 
%% However, NO PERMISSION is granted to distribute a modified version
%% of this file under its original name.
%% 
\ProvidesFile{t1cmr.fd}
        [1998/03/27 v2.5g Standard LaTeX font definitions]
\providecommand{\EC@family}[5]{%
  \DeclareFontShape{#1}{#2}{#3}{#4}%
  {<5><6><7><8><9><10><10.95><12><14.4>%
   <17.28><20.74><24.88><29.86><35.83>genb*#5}{}}
\DeclareFontFamily{T1}{cmr}{}
\EC@family{T1}{cmr}{m}{n}{ecrm}
\EC@family{T1}{cmr}{m}{sl}{ecsl}
\EC@family{T1}{cmr}{m}{it}{ecti}
\EC@family{T1}{cmr}{m}{sc}{eccc}
\EC@family{T1}{cmr}{bx}{n}{ecbx}
\EC@family{T1}{cmr}{b}{n}{ecrb}
\EC@family{T1}{cmr}{bx}{it}{ecbi}
\EC@family{T1}{cmr}{bx}{sl}{ecbl}
\EC@family{T1}{cmr}{bx}{sc}{ecxc}
\EC@family{T1}{cmr}{m}{ui}{ecui}
\endinput
%%
%% End of file `t1cmr.fd'.

%%
%% This is file `ot1cmr.fd',
%% generated with the docstrip utility.
%%
%% The original source files were:
%%
%% cmfonts.fdd  (with options: `fd,OT1cmr')
%% 
%% This is a generated file.
%% 
%% Copyright 1993 1994 1995 1996 1997 1998 1999
%% The LaTeX3 Project and any individual authors listed elsewhere
%% in this file.
%% 
%% This file is part of the LaTeX2e system.
%% ----------------------------------------
%% 
%% It may be distributed under the terms of the LaTeX Project Public
%% License, as described in lppl.txt in the base LaTeX distribution.
%% Either version 1.0 or, at your option, any later version.
%% 
%% In particular, permission is granted to customize the declarations in
%% this file to serve the needs of your installation.
%% 
%% However, NO PERMISSION is granted to distribute a modified version
%% of this file under its original name.
%% 
\ProvidesFile{ot1cmr.fd}
        [1998/03/27 v2.5g Standard LaTeX font definitions]
\DeclareFontFamily{OT1}{cmr}{\hyphenchar\font45 }
\DeclareFontShape{OT1}{cmr}{m}{n}%
     {<5><6><7><8><9><10><12>gen*cmr%
      <10.95>cmr10%
      <14.4>cmr12%
      <17.28><20.74><24.88>cmr17}{}
\DeclareFontShape{OT1}{cmr}{m}{sl}%
     {%
      <5><6><7>cmsl8%
      <8><9>gen*cmsl%
      <10><10.95>cmsl10%
      <12><14.4><17.28><20.74><24.88>cmsl12%
      }{}
\DeclareFontShape{OT1}{cmr}{m}{it}%
     {%
      <5><6><7>cmti7%
      <8>cmti8%
      <9>cmti9%
      <10><10.95>cmti10%
      <12><14.4><17.28><20.74><24.88>cmti12%
      }{}
\DeclareFontShape{OT1}{cmr}{m}{sc}%
     {%
      <5><6><7><8><9><10><10.95><12>%
      <14.4><17.28><20.74><24.88>cmcsc10%
      }{}
% Warning: please note that the upright shape below is
%          used for the \pounds symbol of LaTeX. So this
%          font definition shouldn't be removed.
%
\DeclareFontShape{OT1}{cmr}{m}{ui}
   {
      <5><6><7><8><9><10><10.95><12>%
      <14.4><17.28><20.74><24.88>cmu10%
      }{}
%%%%%%% bold series
\DeclareFontShape{OT1}{cmr}{b}{n}
     {%
      <5><6><7><8><9><10><10.95><12>%
      <14.4><17.28><20.74><24.88>cmb10%
      }{}
%%%%%%%% bold extended series
\DeclareFontShape{OT1}{cmr}{bx}{n}
   {%
      <5><6><7><8><9>gen*cmbx%
      <10><10.95>cmbx10%
      <12><14.4><17.28><20.74><24.88>cmbx12%
      }{}
\DeclareFontShape{OT1}{cmr}{bx}{sl}
      {%
      <5><6><7><8><9>%
      <10><10.95><12><14.4><17.28><20.74><24.88>cmbxsl10%
      }{}
\DeclareFontShape{OT1}{cmr}{bx}{it}
      {%
      <5><6><7><8><9>%
      <10><10.95><12><14.4><17.28><20.74><24.88>cmbxti10%
      }{}
% Again this is necessary for a correct \pounds symbol in
% the cmr fonts Hopefully the dc/ec font layout will take
% over soon.
%
\DeclareFontShape{OT1}{cmr}{bx}{ui}
      {<->ssub*cmr/m/ui}{}
\endinput
%%
%% End of file `ot1cmr.fd'.

%%
%% This is file `il2cmr.fd',
%% generated with the docstrip utility.
%%
%% The original source files were:
%%
%% cslatex.dtx  (with options: `il2cmr')
%% 
%% Tento soubor je soucasti baliku CsLaTeX a je vygenerovan z
%% distribucniho souboru cslatex.dtx.
%% 
%% Copyright (c) 1994--98, 2002
%% Jaroslav Snajdr, Zdenek Wagner, Jiri Zlatuska a The LaTeX3 Project
%% 
%% Tento soubor NENI soucasti systemu LaTeX2e
%% ------------------------------------------
%% 
%% Dalsi informace naleznete na www.cstug.cz/cslatex.
%% 
\ProvidesFile{il2cmr.fd}[1997/02/06 CSLaTeX font definitions]
\DeclareFontFamily{IL2}{cmr}{\hyphenchar\font45 }
\DeclareFontShape{IL2}{cmr}{m}{n}
     {<5><6><7><8><9><10><12> gen*csr
      <10.95> csr10
      <14.4> csr12
      <17.28><20.74><24.88> csr17
     }{}
\DeclareFontShape{IL2}{cmr}{m}{sl}
     {<5><6><7> cssl8
      <8><9> gen*cssl
      <10><10.95> cssl10
      <12><14.4><17.28><20.74><24.88> cssl12
     }{}
\DeclareFontShape{IL2}{cmr}{m}{it}
     {<5><6><7> csti7
      <8> csti8
      <9> csti9
      <10><10.95> csti10
      <12><14.4><17.28><20.74><24.88> csti12
     }{}
\DeclareFontShape{IL2}{cmr}{m}{sc}
     {<5><6><7><8><9><10><10.95><12>
      <14.4><17.28><20.74><24.88> cscsc10
     }{}
\DeclareFontShape{IL2}{cmr}{m}{ui}
     {<5><6><7><8><9><10><10.95><12>
      <14.4><17.28><20.74><24.88> csu10
     }{}
\DeclareFontShape{IL2}{cmr}{b}{n}
     {<5><6><7><8><9><10><10.95><12>
      <14.4><17.28><20.74><24.88> csb10
     }{}
\DeclareFontShape{IL2}{cmr}{bx}{n}
     {<5><6><7><8><9> gen*csbx
      <10><10.95> csbx10
      <12><14.4><17.28><20.74><24.88> csbx12
     }{}
\DeclareFontShape{IL2}{cmr}{bx}{sl}
     {<5><6><7><8><9>
      <10><10.95><12><14.4><17.28><20.74><24.88> csbxsl10
     }{}
\DeclareFontShape{IL2}{cmr}{bx}{it}
     {<5><6><7><8><9>
      <10><10.95><12><14.4><17.28><20.74><24.88> csbxti10
     }{}
\DeclareFontShape{IL2}{cmr}{bx}{ui}
     {<->ssub * cmr/m/ui}{}
\ifx\addlanguage\undefined\else
  \ifx\LdfInit\@undefined
    \def\LdfInit{%
      \chardef\atcatcode=\catcode`\@
      \catcode`\@=11\relax
      \input babel.def\relax
      \catcode`\@=\atcatcode \let\atcatcode\relax
      \LdfInit}
  \fi
\endinput\fi
\endinput
%%
%% End of file `il2cmr.fd'.

\endgroup
\begingroup
\nfss@catcodes
%%
%% This is file `ot1cmss.fd',
%% generated with the docstrip utility.
%%
%% The original source files were:
%%
%% cmfonts.fdd  (with options: `OT1cmss')
%% 
%% This is a generated file.
%% 
%% Copyright 1993 1994 1995 1996 1997 1998 1999 2000 2001
%% The LaTeX3 Project and any individual authors listed elsewhere
%% in this file.
%% 
%% This file was generated from file(s) of the LaTeX base system.
%% --------------------------------------------------------------
%% 
%% It may be distributed and/or modified under the
%% conditions of the LaTeX Project Public License, either version 1.2
%% of this license or (at your option) any later version.
%% The latest version of this license is in
%%    http://www.latex-project.org/lppl.txt
%% and version 1.2 or later is part of all distributions of LaTeX
%% version 1999/12/01 or later.
%% 
%% This file may only be distributed together with a copy of the LaTeX
%% base system. You may however distribute the LaTeX base system without
%% such generated files.
%% 
%% The list of all files belonging to the LaTeX base distribution is
%% given in the file `manifest.txt'. See also `legal.txt' for additional
%% information.
%% 
%% In particular, permission is granted to customize the declarations in
%% this file to serve the needs of your installation.
%% 
%% However, NO PERMISSION is granted to distribute a modified version
%% of this file under its original name.
%% 
\ProvidesFile{ot1cmss.fd}
        [1999/05/25 v2.5h Standard LaTeX font definitions]
\DeclareFontFamily{OT1}{cmss}{\hyphenchar\font45 }
\DeclareFontShape{OT1}{cmss}{m}{n}
     {%
      <5><6><7><8>cmss8%
      <9>cmss9%
      <10><10.95>cmss10%
      <12><14.4>cmss12%
      <17.28><20.74><24.88>cmss17%
      }{}
% Font undefined, therefore substituted
\DeclareFontShape{OT1}{cmss}{m}{it}
      {<->sub*cmss/m/sl}{}
\DeclareFontShape{OT1}{cmss}{m}{sl}
    {%
      <5><6><7><8>cmssi8<9>cmssi9%
      <10><10.95>cmssi10%
      <12><14.4>cmssi12%
      <17.28><20.74><24.88>cmssi17%
      }{}
%%%%%%% Font/shape undefined, therefore substituted
\DeclareFontShape{OT1}{cmss}{m}{sc}
       {<->sub*cmr/m/sc}{}
%%%%%%% Font/shape undefined, therefore substituted
\DeclareFontShape{OT1}{cmss}{m}{ui}
       {<->sub*cmr/m/ui}{}
%%%%%%%% semibold condensed series
\DeclareFontShape{OT1}{cmss}{sbc}{n}
     {%
      <5><6><7><8><9>cmssdc10%
       <10><10.95><12><14.4><17.28><20.74><24.88>cmssdc10%
       }{}

%%%%%%%%% bold extended series
\DeclareFontShape{OT1}{cmss}{bx}{n}
     {%
      <5><6><7><8><9>cmssbx10%
      <10><10.95><12><14.4><17.28><20.74><24.88>cmssbx10%
      }{}
%%%%%%% Font/shape undefined, therefore substituted
\DeclareFontShape{OT1}{cmss}{bx}{ui}
       {<->sub*cmr/bx/ui}{}
\endinput
%%
%% End of file `ot1cmss.fd'.

%%
%% This is file `ot1cmtt.fd',
%% generated with the docstrip utility.
%%
%% The original source files were:
%%
%% cmfonts.fdd  (with options: `fd,OT1cmtt,nowarn')
%% 
%% This is a generated file.
%% 
%% Copyright 1993 1994 1995 1996 1997 1998 1999
%% The LaTeX3 Project and any individual authors listed elsewhere
%% in this file.
%% 
%% This file is part of the LaTeX2e system.
%% ----------------------------------------
%% 
%% It may be distributed under the terms of the LaTeX Project Public
%% License, as described in lppl.txt in the base LaTeX distribution.
%% Either version 1.0 or, at your option, any later version.
%% 
%% In particular, permission is granted to customize the declarations in
%% this file to serve the needs of your installation.
%% 
%% However, NO PERMISSION is granted to distribute a modified version
%% of this file under its original name.
%% 
\ProvidesFile{ot1cmtt.fd}
        [1998/03/27 v2.5g Standard LaTeX font definitions]
\DeclareFontFamily{OT1}{cmtt}{\hyphenchar \font\m@ne}
\DeclareFontShape{OT1}{cmtt}{m}{n}
     {%
      <5><6><7><8>cmtt8<9>cmtt9%
      <10><10.95>cmtt10%
      <12><14.4><17.28><20.74><24.88>cmtt12%
      }{}
%%%%%% make sure subst shapes are available
\DeclareFontShape{OT1}{cmtt}{m}{it}
     {%
      <5><6><7><8><9>%
      <10><10.95><12><14.4><17.28><20.74><24.88>cmitt10%
      }{}
\DeclareFontShape{OT1}{cmtt}{m}{sl}
     {%
      <5><6><7><8><9>%
      <10><10.95><12><14.4><17.28><20.74><24.88>cmsltt10%
      }{}
\DeclareFontShape{OT1}{cmtt}{m}{sc}
     {%
      <5><6><7><8><9>%
      <10><10.95><12><14.4><17.28><20.74><24.88>cmtcsc10%
      }{}
\DeclareFontShape{OT1}{cmtt}{m}{ui}
  {<->ssub*cmtt/m/it}{}
\DeclareFontShape{OT1}{cmtt}{bx}{n}
  {<->ssub*cmtt/m/n}{}
\DeclareFontShape{OT1}{cmtt}{bx}{it}
  {<->ssub*cmtt/m/it}{}
\DeclareFontShape{OT1}{cmtt}{bx}{ui}
  {<->ssub*cmtt/m/it}{}
\endinput
%%
%% End of file `ot1cmtt.fd'.

\endgroup
\DeclareErrorFont{OT1}{cmr}{m}{n}{10}
\newcommand\rmdefault{cmr}
\newcommand\sfdefault{cmss}
\newcommand\ttdefault{cmtt}
\newcommand\bfdefault{bx}
\newcommand\mddefault{m}
\newcommand\itdefault{it}
\newcommand\sldefault{sl}
\newcommand\scdefault{sc}
\newcommand\updefault{n}
\newcommand\encodingdefault{OT1}
\newcommand\familydefault{\rmdefault}
\newcommand\seriesdefault{\mddefault}
\newcommand\shapedefault{\updefault}
\ifx\addlanguage\undefined\else
  \ifx\LdfInit\@undefined
    \def\LdfInit{%
      \chardef\atcatcode=\catcode`\@
      \catcode`\@=11\relax
      \input babel.def\relax
      \catcode`\@=\atcatcode \let\atcatcode\relax
      \LdfInit}
  \fi
\endinput\fi
\endinput
%%
%% End of file `fonttext.cfg'.
}
\fi
\let\@addtofilelist\@gobble
%    \end{macrocode}
%
% Ditto for math although I don't that we will get a lot of
% customisation :-)
%    \begin{macrocode}
\InputIfFileExists{fontmath.cfg}
           {\typeout{====================================^^J%
                     ^^J%
                      Local config file fontmath.cfg used^^J%
                     ^^J%
                     ====================================}%
             \def\@addtofilelist##1{\xdef\@filelist{\@filelist,##1}}%
            }
           {\input fontmath.ltx
\endinput
}
\let\@addtofilelist\@gobble
%    \end{macrocode}
%
% Then we preload several fonts. This file might be customized
% \emph{without} changing the behavior of the format (i.e.\ necessary
% font definitions will be loaded at runtime if they are not
% preloaded).  This is done in the file \texttt{preload.ltx}.
%    \begin{macrocode}
\InputIfFileExists{preload.cfg}
           {\typeout{====================================^^J%
                     ^^J%
                      Local config file preload.cfg used^^J%
                     ^^J%
                     =====================================}%
             \def\@addtofilelist##1{\xdef\@filelist{\@filelist,##1}}%
            }
           {% \iffalse meta-comment
%
% Copyright 1993 1994 1995 1996 1997
% The LaTeX3 Project and any individual authors listed elsewhere
% in this file. 
% 
% For further copyright information, and conditions for modification
% and distribution, see the file legal.txt, and any other copyright
% notices in this file.
% 
% This file is part of the LaTeX2e system.
% ----------------------------------------
%   This system is distributed in the hope that it will be useful,
%   but WITHOUT ANY WARRANTY; without even the implied warranty of
%   MERCHANTABILITY or FITNESS FOR A PARTICULAR PURPOSE.
% 
%   For error reports concerning UNCHANGED versions of this file no
%   more than one year old, see bugs.txt.
% 
%   Please do not request updates from us directly.  Primary
%   distribution is through the CTAN archives.
% 
% 
% IMPORTANT COPYRIGHT NOTICE:
% 
% You are NOT ALLOWED to distribute this file alone.
% 
% You are allowed to distribute this file under the condition that it
% is distributed together with all the files listed in manifest.txt.
% 
% If you receive only some of these files from someone, complain!
% 
% 
% Permission is granted to copy this file to another file with a
% clearly different name and to customize the declarations in that
% copy to serve the needs of your installation, provided that you
% comply with the conditions in the file legal.txt.
% 
% However, NO PERMISSION is granted to produce or to distribute a
% modified version of this file under its original name.
%  
% You are NOT ALLOWED to change this file.
% 
% 
% 
% \fi
%
% \iffalse
%%% From File: preload.dtx
%<*dtx>
           \ProvidesFile{preload.dtx}
%</dtx>
%<*preload>
%<*!tex>
%<+cm>  \ProvidesFile{cmpreloa.%
%<+dc>  \ProvidesFile{dcpreloa.%
%<+xpt>                         xpt}
%<+xipt>                        xip}
%<+xiipt>                       xii}
%<+min> \ProvidesFile{preload.min}
%<+ori> \ProvidesFile{preload.ori}
%</!tex>
%<+tex> \ProvidesFile{preload.ltx}
% \fi
%       \ProvidesFile{preload.dtx}
         [1995/12/04 v2.1f LaTeX Kernel (Font Preloading)]
%
% \CheckSum{41}
%
%% \CharacterTable
%%  {Upper-case    \A\B\C\D\E\F\G\H\I\J\K\L\M\N\O\P\Q\R\S\T\U\V\W\X\Y\Z
%%   Lower-case    \a\b\c\d\e\f\g\h\i\j\k\l\m\n\o\p\q\r\s\t\u\v\w\x\y\z
%%   Digits        \0\1\2\3\4\5\6\7\8\9
%%   Exclamation   \!     Double quote  \"     Hash (number) \#
%%   Dollar        \$     Percent       \%     Ampersand     \&
%%   Acute accent  \'     Left paren    \(     Right paren   \)
%%   Asterisk      \*     Plus          \+     Comma         \,
%%   Minus         \-     Point         \.     Solidus       \/
%%   Colon         \:     Semicolon     \;     Less than     \<
%%   Equals        \=     Greater than  \>     Question mark \?
%%   Commercial at \@     Left bracket  \[     Backslash     \\
%%   Right bracket \]     Circumflex    \^     Underscore    \_
%%   Grave accent  \`     Left brace    \{     Vertical bar  \|
%%   Right brace   \}     Tilde         \~}
%
%\iffalse       This is a META comment
%
% File `preload.dtx'.
% Copyright (C) 1989-1994 Frank Mittelbach and Rainer Sch\"opf,
% all rights reserved.
%
% \fi
%
% \GetFileInfo{preload.dtx}
% \title{The \texttt{preload.dtx} file\thanks {This file has version
%    number \fileversion, dated \filedate}\\ for use with \LaTeXe}
% \date{\filedate}
% \author{Frank Mittelbach \and Rainer Sch\"opf}
%
% \changes{v2.0b}{1993/03/08}{Added 12pt preloads}
% \changes{v2.1e}{1994/11/07}{(DPC) Updated to use \cs{ProvidesFile}}
%
% \def\dst{\expandafter{\normalfont\scshape docstrip}}
%
% \setcounter{StandardModuleDepth}{1}
%
% \maketitle
%
% \section{Overview}
%
%   This file contains an number of possible settings for preloading
%   fonts during installation of NFSS2 (which is used by \LaTeXe).  It
%   will be used to generate the following files:
%   \begin{center}
%   \begin{tabular}{ll}
%   preload.min   &  minimal subset of fonts necessary to run NFSS2 \\
%   preload.ori   &  preload of CM fonts similar to the old 
%                        \texttt{lfonts.tex}                       \\
%   preload.ltx    &  The standard selection of preloads \\
%   cmpreloa.xpt   &  preload of CM fonts for 10pt document size\\
%   cmpreloa.xip   &  preload of CM fonts for 11pt document size\\
%   cmpreloa.xii   &  preload of CM fonts for 12pt document size\\
%   dcpreloa.xpt   &  preload of DC fonts for 10pt size \\
%   dcpreloa.xip   &  preload of DC fonts for 11pt size \\
%   dcpreloa.xii   &  preload of DC fonts for 12pt size \\
%   \end{tabular}
%   \end{center}
%
%    These files are for installations that make use of Computer
%    Modern fonts either old encoding (OT1) or Cork encoding (T1). The
%    Computer Modern fonts with Cork encoding are known as DC-fonts.
%
%    Most important is \texttt{preload.ltx} which is used during
%    format generation. You are \emph{not} allowed to change this file.
%
% \section{Customization}
%
%    You can customize the preloaded fonts in your \LaTeXe{} system by
%    installing a file with the name \texttt{preload.cfg}. If this
%    file exists it will be used in place of the system file
%    \texttt{preload.ltx}.  You can, for example, copy one of the
%    files mentioned above (that can be generated from this source) to
%    \texttt{preload.cfg}.
%
%    Or you can define completely other preloads. In that case start
%    from \texttt{preload.min} since that contains the fonts that have
%    to be preloaded by *all* \LaTeXe{} sytems.
%
%    Avoid using \texttt{preload.ori}, it will load so many fonts that
%    on most installations it is nearly impossible to load other font
%    families afterwards. This file is only generated to show what
%    fonts have been preloaded by \LaTeX~2.09.
%
%    If you normally use other fonts than Computer Modern
%    \texttt{preload.min} might be best.
%
%    \begin{quote} \textbf{Warning:} If you preload fonts with
%    encodings other than the normally supported encodings you have to
%    declare that encoding in a \texttt{fontdef.cfg} configuration
%    file (see the documentation in the file \texttt{fontdef.dtx}).
%    Adding an extra encoding to the format might produce non-portable
%    documents, thus this should be avoided if possible.
%    \end{quote}
%
%    
% \StopEventually{}
%
% \section{Module switches for the \dst{} program}
%
%  The \dst{} will generate the above file from this source using the
%  following module directives:
% \begin{center}
% \begin{tabular}{ll}
%   driver & produce a documentation driver file \\
%   preload& produce a preload\ldots file \\[2pt]
%   cm     & for OT1 encoded Computer Modern \\
%   dc     & for T1 encoded Computer Modern \\[2pt]
%   min    & produce minimal subset \\
%   xpt    & produce 10pt preloads \\
%   xipt   & produce 11pt preloads \\
%   xiipt  & produce 12pt preloads \\
%   ori    & produce preloads similar to old \texttt{lfonts.tex}\\
%   tex    & produce preload.ltx\\
% \end{tabular}
% \end{center}
% A typical \dst{} command file would then have entries like:
% \begin{verbatim}
%\generateFile{preload.min}{t}{\from{preload.dtx}{preload,min}}
%\end{verbatim}
% for generating preload files.
%
% \section{A driver for this document}
%
%    The next bit of code contains the documentation driver file for
%    \TeX{}, i.e., the file that will produce the documentation you
%    are currently reading. It will be extracted from this file by the
%    \dst{} program.
%    \begin{macrocode}
%<*driver>
\documentclass{ltxdoc}
%\OnlyDescription  % comment out for implementation details
\begin{document}
   \DocInput{preload.dtx}
\end{document}
%</driver>
%    \end{macrocode}
%
%
% \section{The code}
%
%    We begin by loading the math extension font (cmex10)
%    and the \LaTeX{} line and circle fonts.
%    It is necessary to do this explicitly since these are
%    used by \texttt{lplain.tex} and \texttt{latex.tex}.
%    Since the internal font name contains |/| characters
%    and digits we construct the name via |\csname|.
%    These are the only fonts (!) that must be loaded in this file.
%
%    All |\DeclarePreloadSizes| can be removed or others can be added,
%    they only influence the processing speed.
% \changes{v2.0c}{1993/08/13}{Added `relax at end of font names.}
%    \begin{macrocode}
\expandafter\font\csname OMX/cmex/m/n/10\endcsname=cmex10\relax
\font\tenln  =line10   \font\tenlnw  =linew10\relax
\font\tencirc=lcircle10 \font\tencircw=lcirclew10\relax
%    \end{macrocode}
%    The above fonts should not be touched but anything below this
%    point here in the preload suggestions can be modified without any
%    problems.
%    \begin{macrocode}
%<-tex>%*******************************************
%<-tex>% Start any modification below this point **
%<-tex>%*******************************************
%<-tex>
%%
%% Computer Modern Roman:
%%-----------------------
%<*ori>
\DeclarePreloadSizes{OT1}{cmr}{m}{n}
        {5,6,7,8,9,10,10.95,12,14.4,17.28,20.74,24.88}
\DeclarePreloadSizes{OT1}{cmr}{bx}{n}{9,10,10.95,12,14.4,17.28}
\DeclarePreloadSizes{OT1}{cmr}{m}{sl}{10,10.95,12}
\DeclarePreloadSizes{OT1}{cmr}{m}{it}{7,8,9,10,10.95,12}
%</ori>
%<+xpt&cm> \DeclarePreloadSizes{OT1}{cmr}{m}{n}{5,7,10}
%<+xpt&dc> \DeclarePreloadSizes{T1}{cmr}{m}{n}{5,7,10}
%<+xipt&cm> \DeclarePreloadSizes{OT1}{cmr}{m}{n}{6,8,10.95}
%<+xipt&dc> \DeclarePreloadSizes{T1}{cmr}{m}{n}{6,8,10.95}
%<+xiipt&cm> \DeclarePreloadSizes{OT1}{cmr}{m}{n}{6,8,12}
%<+xiipt&dc> \DeclarePreloadSizes{T1}{cmr}{m}{n}{6,8,12}
%%
%% Computer Modern Sans:
%%----------------------
%<+ori> \DeclarePreloadSizes{OT1}{cmss}{m}{n}{10,10.95,12}
%%
%% Computer Modern Typewriter:
%%----------------------------
%<+ori> \DeclarePreloadSizes{OT1}{cmtt}{m}{n}{9,10,10.95,12}
%%
%% Computer Modern Math:
%%----------------------
%<*ori>
\DeclarePreloadSizes{OML}{cmm}{m}{it}
         {5,6,7,8,9,10,10.95,12,14.4,17.28,20.74}
\DeclarePreloadSizes{OMS}{cmsy}{m}{n}
         {5,6,7,8,9,10,10.95,12,14.4,17.28,20.74}
%</ori>
%    \end{macrocode}
%    
%    The math fonts are the same for both DC and CM fonts. So far
%    there isn't an agreed on standard.
% \changes{v2.4e}{1995/12/04}
%      {Ulrik Vieth. added 12pt OMS and OML preloads  /1989}
%    \begin{macrocode}
%<*xpt> 
\DeclarePreloadSizes{OML}{cmm}{m}{it}{5,7,10}
\DeclarePreloadSizes{OMS}{cmsy}{m}{n}{5,7,10}
%</xpt> 
%<*xipt> 
\DeclarePreloadSizes{OML}{cmm}{m}{it}{6,8,10.95}
\DeclarePreloadSizes{OMS}{cmsy}{m}{n}{6,8,10.95}
%</xipt>
%<*xiipt> 
\DeclarePreloadSizes{OML}{cmm}{m}{it}{6,8,12}
\DeclarePreloadSizes{OMS}{cmsy}{m}{n}{6,8,12}
%</xiipt> 
%%
%% LaTeX symbol fonts: 
%%--------------------
%<*ori>
\DeclarePreloadSizes{U}{lasy}{m}{n}
         {5,6,7,8,9,10,10.95,12,14.4,17.28,20.74}
%</ori>
%</preload>
%    \end{macrocode}
%
%
%
% \Finale
%
\endinput
}
\let\@addtofilelist\@gobble
%    \end{macrocode}
%
%
% \begin{macro}{\@acci}
% \begin{macro}{\@accii}
% \begin{macro}{\@acciii}
% \changes{v2.1m}{1994/05/16}{Define saved versions of accents}
%    We also save the values of some accents in |\@acci|, |\@accii|
%    and |\@acciii| so they can be restored by a |minipage| inside a
%    |tabbing| environment.
%    \begin{macrocode}
\let\@acci\' \let\@accii\` \let\@acciii\=
%    \end{macrocode}
% \end{macro}
% \end{macro}
% \end{macro}
%
%  
% \begin{macro}{\cal}
% \changes{v3.0a}{1995/05/24}
%      {(DPC) Remove definition}
% \begin{macro}{\mit}
% \changes{v3.0a}{1995/05/24}
%      {(DPC) Remove definition}
%    Here were the two old \meta{alphabet identifiers}.
% \end{macro}
% \end{macro}
%
%
% \iffalse
%<+checkmem>\CHECKMEM
% \fi
%
% \Finale
%
