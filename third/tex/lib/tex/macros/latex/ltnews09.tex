% \iffalse meta-comment
%
% Copyright 1993 1994 1995 1996 1997
% The LaTeX3 Project and any individual authors listed elsewhere
% in this file. 
% 
% For further copyright information, and conditions for modification
% and distribution, see the file legal.txt, and any other copyright
% notices in this file.
% 
% This file is part of the LaTeX2e system.
% ----------------------------------------
%   This system is distributed in the hope that it will be useful,
%   but WITHOUT ANY WARRANTY; without even the implied warranty of
%   MERCHANTABILITY or FITNESS FOR A PARTICULAR PURPOSE.
% 
%   For error reports concerning UNCHANGED versions of this file no
%   more than one year old, see bugs.txt.
% 
%   Please do not request updates from us directly.  Primary
%   distribution is through the CTAN archives.
% 
% 
% IMPORTANT COPYRIGHT NOTICE:
% 
% You are NOT ALLOWED to distribute this file alone.
% 
% You are allowed to distribute this file under the condition that it
% is distributed together with all the files listed in manifest.txt.
% 
% If you receive only some of these files from someone, complain!
% 
% 
% Permission is granted to copy this file to another file with a
% clearly different name and to customize the declarations in that
% copy to serve the needs of your installation, provided that you
% comply with the conditions in the file legal.txt.
% 
% However, NO PERMISSION is granted to produce or to distribute a
% modified version of this file under its original name.
%  
% You are NOT ALLOWED to change this file.
% 
% 
% 
% \fi
% Filename: ltnews09.tex 08/06/1998 CAR
% Filename: ltnews09.tex 27/06/1998 FMi
% Filename: ltnews09.tex 27/06/1998 bb

% This is issue 9 of LaTeX News.

\documentclass
%   [lw35fonts]
   {ltnews}

\newcommand\cs[1]{\texttt{\textbackslash#1}}

%\usepackage[T1]{fontenc}

\publicationmonth{June}
\publicationyear{1998}
\publicationissue{9}

\begin{document}

\maketitle


\section{New math font encodings}

A joint working group of the \TeX{} Users Group and the \LaTeX3
Project is developing a new 8-bit math font encoding for \TeX{}.  
It is designed to overcome several limitations and implementation
problems of the old math font encodings and to simplify switching
between different sets of math fonts, much as the \LaTeX{} font
selection interface has simplified switching between text fonts.

%%% Add something about additional symbols and STIX glyphs???

Since the work on this project relies entirely on volunteer work, we
cannot give a specific release date yet.  However, a prototype
implementation already exists.  This contains several sets of virtual
fonts, some \LaTeX{} packages and a kernel module; we hope to
integrate it into the main \LaTeX{} distribution for the next
release.

Documents using only standard \LaTeX{} commands for math symbols
should not be affected by switching to the new math font encodings  
However, documents, classes or packages making specific assumptions
about the encoding of math symbol fonts are likely to break.

Further information about the Math Font Group may be found on
the World Wide Web at \texttt{http://www.tug.org/twg/mfg/}.


\section{A new math accent}

A new math accent, \verb|\mathring|, has been added. This is a math mode
version of the ring accent (\r{}) which is available in text 
mode with the command \verb|\r|.

\section{Extended \cs{DeclareMathDelimiter}}

The command \verb|\DeclareMathDelimiter| has been extended.
Normally this command takes six arguments. Previously, when being used
to declare a character (such as \texttt{[}) as a delimiter, a variant
form was used with only five arguments. The argument specifying the
default `math class' was omitted. Now the full six-argument form may be
used in this case. The extra information is used to implicitly declare
the character via \verb|\DeclareMathSymbol| for use when the symbol is
not used with \verb|\left| or \verb|\right|.

The old five-argument form is detected and will work as before.

\newpage

\section{Tools distribution}

The \texttt{multicol} package now supports the production of multiple
columns without balancing the last page. To get this effect use the
\texttt{multicols*} environment.

The \texttt{layout} package was partly recoded by Hideo Umeki to
display page layout effects in a better way.

As suggested by Donald Arseneau, the \texttt{calc} package was extended
to support the new commands \verb|\widthof{<text>}|,
\verb|\heightof{<text>}|, and \verb|\depthof{<text>}| within a
\texttt{calc}-expression.  At the same time we modified a few kernel
commands so that \texttt{calc}-expressions can now be used in various
useful places such as the dimension arguments to the \texttt{tabular}
environment and the \verb|\rule| command. For many other standard
\LaTeX{} commands this was already possible.


\section{Support for Cyrillic encodings}
 
We are very pleased that, after a lengthy period of development, a set
of fonts, encodings and support files for using \LaTeX\ with Cyrillic
characters will soon be available.
 
Test versions of the `LH' fonts for these Cyrillic encodings, based on
the Computer Modern design, are available from CTAN archives in the
directory \texttt{fonts/cyrillic/lh-test}.  The \LaTeX\ support files
(by Werner Lemberg and Vladimir Volovich) are also available from CTAN
archives in\\
 \texttt{macros/latex/contrib/supported/t2}
 
 
\section{Default docstrip header}

Many \LaTeX\ users now distribute packages in documented source form
using the \textsf{docstrip} system. Docstrip allows a header to be
placed on generated package files, suitable for giving copyright
information, or distribution conditions.

We have changed the default version of this header so that it allows
stripped files to be distributed in ready-to-run installations such as
the \TeX{}Live CD\@.  If you use the default header for distributing
your files you should check that the new copyright text is acceptable
to you.  The file \texttt{docstrip.dtx} explains how to produce your own
header if you wish to do so.


\end{document}
