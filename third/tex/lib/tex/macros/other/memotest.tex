\documentstyle[12pt]{memo}  % type size of 12 points

%\documentstyle[11pt]{memo} % type size of 11 points
%\documentstyle{memo}       % type size of 10 points
%
% This is a sample file to show you typical input for printing a
% memo at the Los Alamos National Laboratory using LaTeX.
% 
% Notice that the '%' is the comment character.  The '%' and the
% rest of the line are ignored by LaTeX.  All of the LaTeX memo
% commands will appear in this file.  The ones not used by this
% file will be commented out.  To look at the body of the memo,
% skip down to the \begin{document} command.

% When you print out the DVI file that is generated from this
% source file, you will notice that the header doesn't look like
% pre-printed memo paper.  That is because \headerfonts{texfonts}
% is used here.  
% 
% If you are printing on a PostScript printer, use
% \headerfonts{postscript} to get a good looking header.  The
% \headerfonts{lafonts} command will also produce a nice header,
% but you will need to have installed the LANL header fonts.  If
% you use \headerfonts{memopaper}, put pre-printed memo paper
% in your printer or copying machine.
%
  \headerfonts{texfonts}
%  \headerfonts{postscript}
%  \headerfonts{lafonts}
%  \headerfonts{memopaper}

  \bodyfonts{texfonts}
%  \bodyfonts{postscript}
  \typeface{tt}
%  \typeface{rm}

% With no date command, today's date will be printed.
%
  \date{March 28, 1990}

  \symbol{C-2}
%  \serialnumber{5-213}
  \mailstop{B253}
  \telephone{5-0859}
  \to{LaTeX Users}
  \from{Steve Sydoriak, C-2, MS B253}
  \subject{typical memo}
%  \reference{Office Procedures Manual}
%  \thru{Alicia J. Lujan, ABC-3, MS Q123 \\ 
%        L. S. Steele, AB-5, MS R456}

  \originator{SS}
  \typist {SS}
  \encas
  \cy{T.J. Benton, BB-5, MS G999 \\ A.L. Salazar, XX-7, MS B888}

%  \signature {Steve Sydoriak}
%  \signer {SS}
%  \approval{R. J. O'Conner}
%  \attachments{Graph, Gravitational Pull vs. Image Time, TP-3, MS B881}
%  \attachmentas
%  \attachmentsas
%  \distribution{K. C. Jordan, C-5, MS B111\\
%               T. S. Solomon, TP-1, MS B222}
%  \enc{Memo, Smith to Jones, June 25, 1986\\
%       Letter, Landau to Gresham, March 1, 1987}
%  \attachmentspagebreak
%  \cypagebreak
%  \distributionpagebreak
%  \encpagebreak

%  \shortmemostyle % For memos of ten lines or less.

%  \classlabel{u}  %  UNCLASSIFIED
%  \classlabel{c}  %  CONFIDENTIAL
%  \classlabel{s}  %  SECRET

%  \memopaperhcorr{4pt}
%  \memopapervcorr{-6pt}

\begin{document}
\opening

This example shows what a typical memo might look like.  It uses
the texfonts option for the header and the body of the memo.
The texfonts option calls for use of TeX's Computer Modern fonts.
The body has the typewriter typeface with a point size of 12.

Remember that LaTeX interprets a blank line as the start of a new
paragraph, and that any of the special characters \#,\$,\%, \&,
\{, and \} must be preceded by a backslash.  To produce double
quotes in typewriter typeface, use the " key on your keyboard.  To
produce double quotes in Roman typeface, use `` and '' instead. 

The file named memotest.tex that was used to print this memo can
be used as a template to write your own memos.  All of the
preamble commands that can be used by LaTeX memos are shown in
memotest.tex.  Many of the commands are commented out.  They can be
activated by removing the \% at the beginning of the line.

The spacing and indentation of the preamble commands make the file
easier to read; the outcome of your file is not affected.  The
preamble begins with the $\backslash$documentstyle command and
continues to the $\backslash$begin\{document\} command where the
document section starts.  Your file must have a
$\backslash$end\{document\} command to indicate the end of the
memo.

The texfonts option was chosen for the header of this memo
because the Computer Modern fonts are available on any
installation of TeX\@.  Use the postscript, lafonts, or
memopaper options in the $\backslash$headerfonts command to
obtain a good looking header.

Get a copy of {\em LaTeX Memo Reference}, CIC \#919, from the
Computing Information Center to learn more about the LaTeX memo
commands.  The document is also available online as an stexted CFS
file in two places: /c2doc/unix/slatexmemo and
/c2doc/vms/slatexmemo.  See {\em LaTeX, A Document Preparation
System} by Leslie Lamport to learn all about the LaTeX commands
that you can put in the body of your memo.

See {\em LaTeX, A Document Preparation System} by Leslie
Lamport to learn all about the LaTeX commands that you can put
in the body of your memo.

\closing
\end{document}
