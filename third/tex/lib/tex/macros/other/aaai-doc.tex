%NAME: aaai-doc.tex
\documentstyle[aaai]{article}

\title{Instructions to Authors\thanks{Schlumberger Palo Alto Research
supported the development of these instructions.}}
\author{Tom Mitchell \and Reid G. Smith  \\
	Program Co-Chairs \\ AAAI-88 Conference
	\And
	Shirley Jowell \\
	Morgan Kaufmann Publishers \\
	San Mateo, California}

\begin{document}

\maketitle

\begin{abstract}
The {\em Proceedings of AAAI-88}, the published presentations of
    the National Conference on Artificial Intelligence, will be
    printed using the photo-offset print process directly from 
    camera-ready copy furnished by the authors.
    To ensure that all papers in the {\em Proceedings} have a uniform
    appearance, authors are asked to adhere to the following
    instructions.
\end{abstract}


\section{Introduction}

This year AAAI papers will not be reduced by the printers,
but instead will be printed from $8.5 \times 11''$ masters.
As a result, preparation of AAAI papers will be easier at most
institutions.

AAAI papers should be typeset using a system such as \LaTeX{} or Scribe
and printed using a good laser or other letter-quality printer.
{\em Do not use line printer or dot matrix printer output.}
Papers with poor quality output, e.g., light or gray type, 
and papers which significantly deviate from these instructions,
e.g., by using 8-point fonts,
will not be included in the proceedings,
as such papers would be rendered
unreadable by the printing process.

Output can either be printed directly on the model forms, 
or printed on heavy bond paper and pasted onto
the model forms using a glue stick, spray adhesive, or rubber cement.
The pasted area should be flat and free from creases, 
and there should be no excess glue anywhere on the forms.

\section{Style and Format}

\subsection{General Instructions}

Manuscripts must be printed two columns to a page on the master forms
supplied.
The light blue guidelines on the forms (which will not reproduce in
the printing process) outline the printing area.
Do not print anything outside of these lines.
Use only one side of the master forms.

Use 10-point type in a clear, readable font with 1-point leading (10 on 11).
``Write-white'' laser printers must use fonts specifically designed for
them---not fonts for ``write-black'' printers.

Start all pages (except the first) directly under the top margin, as
indicated.
See next section regarding how to format the first page.

Indent when starting a new paragraph, except after major headings.

\subsection{Title Page}

\subsubsection{Title and Author Information}
The title appears near the top of the first page, centered on the entire
width of the page, in a 14-point bold font.
The name(s) of the author(s) appear in the designated areas below the title
of the paper, along with affiliation(s) and complete address(es) in a
12-point font.

Any credit to a sponsoring agency should appear in a footnote at the
bottom of the left column of the first page.
See the example in these instructions.

\subsubsection{Abstract}
The Abstract appears at the beginning of the first column, using a
slightly smaller width than the body of the paper.
The title ``Abstract'' appears above the body of the abstract in
a bold font.
The Abstract should be no longer than 200 words.

\subsubsection{Text}
The main body of the text follows the Abstract, observing the blue
guidelines for the two-column format.


\subsection{Sections}

Section, subsection, and subsubsection headings should be printed in the
style shown in these instructions.
Leave a blank space of approximately 10 points above and 8 points below
section headings, 9 points above and 4 points below subsection headings,
and 8 points above and 3 points below subsubsection headings.

Special sections should be arranged and headed as follows:

\begin{description}
\item[Acknowledgments:] The Acknowledgments section, if included,
		       follows the main body of the text and is headed
		       ``Acknowledgments,'' printed in the same style
		       as a section heading, but without a number.
\item[Appendices:] Any Appendices follow the Acknowledgments 
		  (or directly follow the text) and look like
		  sections, except that they are numbered with capital
		  letters instead of arabic numerals.
\item[References:]
The References section is headed ``References,'' printed in the same
style as a section heading, but without a number.
A sample list of references is given at the end of these
instructions.
Use a consistent format for references, such as provided by
Bib\TeX{}.

\end{description}

\subsection{Citations}

Citations within the text should include the author's last name and
year, for example \cite{cheeseman:probability}.
Append lower case letters to the year in cases of ambiguity.
Multiple authors should be treated as in the following examples:
\cite{abelson-et-al:scheme} and
\cite{brachman-schmolze:kl-one}.
If the author portion of a citation is obvious, it should be omitted,
e.g., Levesque \shortcite{levesque:belief}.
Multiple citations should be collapsed as follows:
\cite{levesque:functional-foundations,levesque:belief}.

\nocite{abelson-et-al:scheme}
\nocite{brachman-schmolze:kl-one}
\nocite{cheeseman:probability}
\nocite{haugeland:mind-design}
\nocite{lenat:heuristics}
\nocite{levesque:functional-foundations}
\nocite{levesque:belief}

\subsection{Footnotes}

Footnotes should appear at the bottom of the page.
They are referred to by superscript numbers.\footnote{This is how your
footnotes should appear.}
Footnotes should be separated from the text by a short line.\footnote{Note
the line separating these footnotes from the text.}

\section{Figures, Drawings, Tables, and Photographs}

\subsection{General Instructions}

Figures, drawings, tables, and photographs should be placed throughout
the paper at the places where they are first discussed, rather than at
the end of the paper.
If placed at the bottom or top of a page, illustrations may run across
both columns.
Securely attach them to the master form with glue stick, spray adhesive,
rubber cement, or write tape.
Do not use transparent tape as the printing process causes it to
obscure copy.
Number them sequentially.
The references should be in the following form:
Figure 1, Table 1, etc.

The illustration number and caption should appear under the
illustration. 
Leave a margin of 1/4-inch around the area covered by the
figure and caption.

Captions, labels, and other text in illustrations
be at least 10-point type.

{\em Line printer printouts should not be used.}

\subsection{Drawings}

Original line drawings should be drawn in {\em black\/} ink, not pencil.
Do not color in drawings.
Lines should be heavy enough to reproduce well.

\subsection{Photographs}

Photographs should be black and white glossies.
Do not pre-halftone photographs.
Color photographs do not produce well.
(Red will reproduce as black, for example.)
Photographs incur extra expense, so
please use them judiciously.

\section{Length of Papers}

Papers should not exceed {\em five\/} pages, including illustrations and
tables.
One additional page may be included by accompanying the paper with a
check for \$250, payable to the American Association for Artificial
Intelligence.
Papers over six pages will not be accepted for publication.

\section{Identification}

Make certain that your name is typed or written on the {\em back\/} of
every page of the masters, and number the pages sequentially.
Failure to do this may result in some pages of your paper being
misplaced or, worse, inserted in another paper.
This information is for identification only (final page numbers will
be assigned by the Publisher).

\section{Mailing}

{\em Make a photocopy of your final paper.}
Keep the photocopy in your files for reference or in case the original
is lost in the mail.

Your paper should be {\em received\/} by June 1, 1988.
{\em Papers received later than June 1, 1988 will not be
included in the proceedings.}

Do not fold the master forms for mailing.
Use the enclosed cardboard backing and envelope for mailing.
If you use a separate envelope, clearly mark on the envelope:
{\bf Do Not Fold or Bend}.

Send to:
\begin{quote}
Shirley Jowell \\
Attn: AAAI-88 Conference \\
Morgan Kaufmann Publishers \\
2929 Campus Drive \\
San Mateo, CA \hspace{1em} 94403\\
U. S. A.
\end{quote}

\section{Inquiries}

If you have any questions about the preparation or submission of your
paper as instructed in this package, please contact:
\begin{quote}
Shirley Jowell \\
Morgan Kaufmann Publishers \\
(415) 578-9911
\end{quote}

\appendix

\section{Using \LaTeX{}}

A \LaTeX{} style option (for version 2.09 of \LaTeX{}) that implements these
instructions has been prepared, 
as well as
two Bib\TeX{} styles
(one for version 0.98i and one for version 0.99c of Bib\TeX{})
that implement the citation and reference style in
these instructions.

The relevant files have been placed in the \LaTeX{} style collection at
Rochester.
To retrieve these files use one of the following methods:
\begin{enumerate}
\item  For Internet users - how to ftp:

Here is an example session.  Ftp syntax varies from host to
host; your syntax may be different.

\begin{verbatim}
% ftp cayuga.cs.rochester.edu
 ... (general blurb)
ftp> login anonymous
Password: <any non-null string>
 ... (more blurb)
ftp> cd public/latex-style
 ... (more blurb)
ftp> get aaai-instructions.tex
 ... (more blurb)
ftp> get aaai.sty
 ... (more blurb)
ftp> get aaai-named-0.99.bst
 ... (more blurb)
ftp> get aaai-named-0.98.bst
 ... (more blurb)
ftp> quit
\end{verbatim}

The name ``rochester.arpa'' or the address 
``192.\discretionary{}{}{}5.\discretionary{}{}{}53.\discretionary{}{}{}209''
can be used instead of ``cayuga.cs.rochester.edu''.

\item  Non-Internet users - how to retrieve by mail:

Send a piece of mail to LaTeX-Style@cs.rochester.edu
in the following format:
Subject line should contain the phrase ``@file request''.  Body of the
mail should start with a line containing only an @ (at) sign.  
The first line following should be a mail address {\em from}
rochester {\em to} you.  Follow with the names of the files you want
separated by spaces or new lines.
End with a line containing only an @ sign.  Case is not significant.

For example, if you are {\tt user} at {\tt dept.site.edu}, this is what you
should send:  

\begin{verbatim}
To: latex-style@cs.rochester.edu
Subject: @file request

@
user@dept.site.edu
aaai-instructions.tex aaai.sty 
aaai-named-0.99.bst aaai-named-0.98.bst 
@
\end{verbatim}

It is best to fully qualify your mail address unless it is fully
registered.  For example, use user\%site.bitnet@cunyvm.cuny.edu instead of
user@site.bitnet.

\end{enumerate}

Further information on using these styles can be obtained by writing:
\begin{quote}
Peter F. Patel-Schneider \\
Schlumberger Palo Alto Research \\
3340 Hillview Avenue \\
Palo Alto, California \hspace{1em} 94304 \\
pfps@spar.slb.com
\end{quote}

%% This section was initially prepared using BibTeX.  The .bbl file was
%% placed here later
%\bibliography{publications}
%\bibliographystyle{aaai-named-0.99}
\begin{thebibliography}{}

\bibitem[\protect\citename{Abelson \bgroup \em et al.\egroup ,
  }1985]{abelson-et-al:scheme}
Harold Abelson, Gerald~Jay Sussman, and Julie Sussman.
\newblock {\em Structure and Interpretation of Computer Programs}.
\newblock MIT Press, Cambridge, Massachusetts, 1985.

\bibitem[\protect\citename{Brachman and Schmolze,
  }1985]{brachman-schmolze:kl-one}
Ronald~J. Brachman and James~G. Schmolze.
\newblock An overview of the {KL-ONE} knowledge representation system.
\newblock {\em Cognitive Science}, 9(2):171--216, April--June 1985.

\bibitem[\protect\citename{Cheeseman, }1985]{cheeseman:probability}
Peter Cheeseman.
\newblock In defense of probability.
\newblock In {\em Proceedings IJCAI-85}, pages 1002--1009. International Joint
  Committee for Artificial Intelligence, August 1985.

\bibitem[\protect\citename{Haugeland, }1981]{haugeland:mind-design}
John Haugeland, editor.
\newblock {\em Mind Design}.
\newblock Bradford Books, Montgomery, Vermont, 1981.

\bibitem[\protect\citename{Lenat, }1981]{lenat:heuristics}
Douglas~B. Lenat.
\newblock The nature of heuristics.
\newblock Technical Report CIS-12 (SSL-81-1), Xerox Palo Alto Research Centers,
  April 1981.

\bibitem[\protect\citename{Levesque, }1984a]{levesque:functional-foundations}
Hector~J. Levesque.
\newblock Foundations of a functional approach to knowledge representation.
\newblock {\em Artificial Intelligence}, 23(2):155--212, July 1984.

\bibitem[\protect\citename{Levesque, }1984b]{levesque:belief}
Hector~J. Levesque.
\newblock A logic of implicit and explicit belief.
\newblock In {\em Proceedings AAAI-84}, pages 198--202. American Association
  for Artificial Intelligence, August 1984.

\end{thebibliography}

\end{document}


