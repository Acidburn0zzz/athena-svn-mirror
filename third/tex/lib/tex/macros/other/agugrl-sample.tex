%NAME: agugrl-sample.tex
% an example of how to use the agugrl macro package
% By Stephen Gildea <mit-erl!gildea> 01 Jul 87
% This text is from ``Sample for typewritten GRL papers''
\documentstyle[agugrl]{article}
\begin{document}

\title{Investigation of Two High-Stress Drop Earthquakes in the
  Shumagin Seismic Gap, Alaska}
\author{Leigh House and John Boatwright}
\address{Lamont-Doherty Geological Observatory and Department of
  Geological Sciences, Columbia University
  Palisades, New York}
\author{Keith Preistley\thanks{Also at U.S. Geological Survey, Reston}}
\address{Seismological Laboratory, Mackay School of Mines, University of
  Nevada at Reno}
\maketitle

\begin{abstract}
Two moderate size earthquakes occurred within a local network of
short-period seismograph stations in the Shumagin Islands, Alaska, on
April 6, 1974.
\end{abstract}

\section{Introduction}
Two moderate size ($m_b =5.8$, $6.0$) earthquakes and their
aftershocks that occurred within the Shumagin Islands seismic network
in Alaska have produced a unique data set for a detailed study of
convergent tectonics at depth in an area that has been identified as a
seismic gap \cite{kelleher}.

Lamont-Doherty Geological Observatory has operated a network of
vertical short-period, radio-telemetered station in the Shumagin
Islands region of Alaska since July 1973 \cite{davies}.  As originally
installed, the network consisted of eight high-gain remote stations that
telemeter their data to a central recording site at Sand Point (Figure 1).
\marginboxed{Fig. 1}

\section{SMA 1 Waveform Analysis}
Since the Sand Point station was at an {\em SH\/} node, the vertical
and horizontal components were combined to obtain the incident {\em
SV\/} pulse shape by using the free surface transformation
\begin{equation}
u_{sv}(t) = \frac{\cos 2j}{2 \cos j} u_x(t) + \sin ju_z(t)
\end{equation}
Here $j$ is the angle of incidence of the {\em S\/} save, and $u_x(t)$
is the horizontal component (positive downward), shown in Figure 2.
\marginboxed{Fig. 2}

\subsection{Focal Mechanism of the Main Shocks}
The focal mechanism of the second main shock ($m_b = 6.0$), shown in
Figure 3, was determined from long-period arrivals at World-Wide
Standard Seismographic network stations.  Bost {\em S\/} and {\em P\/}
wave first motions were used; however, the solution is more strongly
constrained by the {\em S\/} wave polarizations.  The results are
shown in Table 1.\marginboxed{Table 1}

\subsubsection{Magnitudes and b value}
Magnitudes of most of the earthquakes in this sequence were estimated
from coda duration measurements similar to the techniques used by R.
Lee et al.\ (unpublished manuscript, 1984).

\begin{acknowledgments}
The authors are grateful to L. Sykes and K. Jacob for critical reviews
of this paper.
\end{acknowledgments}

% You can also use BibTeX to generate the bibliography automatically.
% With BibTeX, use the natsci bibliography format.
\begin{thebibliography}{}

\bibitem[Belt, 1968]{belt}
Belt, E.S., Post-Acadian rifts and related facies, eastern Canada, in 
{\em Studies on Appalachian Geology,} edited by E. Zen et al.,
pp.~95--113, John Wiley, New York, 1968.

\bibitem[Davies and House, 1979]{davies}
Davies, J.R., and House, M.O, Another random paper.

\bibitem[Kelleher et al., 1970]{kelleher}
Kelleher et al., Can anyone find these two referenced papers for me.

\bibitem[Orringer, 1974]{mit}
Orringer, O., Frontal analysis program, {\em Rep.\ ASRL TR 1023,}
Aeroelastic and Struct.\ Lab., Mass.\ Inst.\ of Technol., Cambridge, 1974.
\end{thebibliography}

\begin{addresses}
J. Boatwright and L. House, Lamont-Doherty Geological Observatory and
Department of Geological Sciences, Columbia University, Palisades NY
10964.

K. Priestley, Seismological Laboratory, Mackay School of Mines,
University of Nevada at Reno, Reno NV 89557.
\end{addresses}

\begin{received}
(Received September 17, 1983; \\
revised June 9, 1984; \\
accepted June 22, 1984)
\end{received}

\copyrightnotice{Copyright 1986 by the American Geophysical Union.}

\papernumber{4B1073. \\ 0148--0227/86/004B--1073\$05.00}

\runninghead{House et al.: High-Stress Drop Earthquakes}

\begin{captions}
Fig.~1. Detail of short-period WWSSN analysis.  The lowermost trace
is the seismogram as digitized with the band-passed seismogram above it.
\end{captions}

%% Every table goes in its own table environment.
\begin{table}
  \begin{center}
  TABLE 1. Preparation of CRC for GRL

  \end{center}
  %% The expandedtabular environment is like tabular, but the table
  %% is expanded to the current line width.
  \begin{expandedtabular}{cc}
  \hline
  Problem & Response \\ \hline
  Need general information & Read AGU Style: A \\
   & Guide for Contributors \\
   & \\
  Specific Questions & Call AGU Office \\
   & (202) 462--6903 \\
   & \\
  Format for references & See General Instructions \\ \hline
  \end{expandedtabular}
\end{table}

\end{document}

