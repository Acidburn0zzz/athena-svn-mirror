%format latex
\documentstyle[11pt,supertabular]{article}
%  4-FEB-91     JLB     Adapted this file to the additions made to the
%                       style file.
%  3-JUN-91     RmS     Changed into a proper LaTeX input file
\setlength{\topmargin}{0cm}
\setlength{\oddsidemargin}{0cm}
\setlength{\textheight}{23cm}
\setlength{\textwidth}{15cm}
\newcommand{\tbsp}{\rule{0pt}{18pt}}
                  % I use \tbsp to get a vertical distance after \hline
\begin{document}
\section{The Supertabular}
Here is the description of a new style file called \verb|supertab|.
This new style offers a new environment, the \verb|supertabular|
environment. As the name says it is an extension of the normal
\verb|tabular| environment.

With the original \verb|tabular| environment a tabular must always fit
on {\it one} page. If the tabular becomes too large the text overwrites
the page's bottom margin and you get an \verb|Overfull vbox| message.

The \verb|supertabular| environment uses the \verb|tabular| environment
internally, but it evaluates the used space every time it gets a
\verb|\\| command. If the tabular reaches the textheight,
it automatically inserts an \verb|\end{tabular}| command, starts a new
page and inserts the tablehead on the new page continuing the tabular.

New commands to use with \verb|supertabular| are:
\begin{center}
\begin{tabular}{l p{8cm}}
\verb|\tablefirsthead{...}| & defines the contents of the first occurence of
                            the tabular head.
                            The use of this command is optional.
                         Don't forget to close the head by a \verb|\\|
                         or \verb|\newline|.                        \\
\verb|\tablehead{...}| & defines the contents of all subsequent ocurrences of
                         the tabular head.
                         Don't forget to close the head by a \verb|\\|
                         or \verb|\newline|.                        \\
\verb|\tabletail{...}| & defines something which should be inserted
                         before each \verb|\end{tabular}|, except the last.\\
\verb|\tablelasttail{...}| & defines something which should be inserted
                             before the last \verb|\end{tabular}|.
                             The use of this command is optional.\\
\verb|\topcaption{...}|     &\\
\verb|\bottomcaption{...}|  &\\
\verb|\tablecaption{...}|   & Provide a caption for the super-table, either
                              at the top or at the bottom of the table. When
                              \verb|\tablecaption| is used the caption will
                              be placed at the default location, which is at
                              the top.\\
\end{tabular}
\end{center}
Note that you define new lines inside the supertabular as normal by
\verb|\\| or \verb|\\[...pt]|. All column definition commands can be
used, including \verb|@{...}| and \verb|p{...}|. You can {\it not} use
the optional positioning argument and the width argument which are
possible with the \verb|\begin{tabular}| command.

Here is an example of a supertabular. You will find the definitions at
the end of the supertabular.
\begin{center}
\tablefirsthead{\hline  \multicolumn{1}{|c}{\tbsp Number}
                       & \multicolumn{1}{c}{Number$^2$}
                       & Number$^4$
                       & \multicolumn{1}{c|}{Number!} \\ \hline\tbsp  }
\tablehead{\hline \multicolumn{4}{|l|}{\small\sl continued from previous page}\\
           \hline \multicolumn{1}{|c}{\tbsp Number}
                       & \multicolumn{1}{c}{Number$^2$}
                       & Number$^4$
                       & \multicolumn{1}{c|}{Number!} \\ \hline\tbsp  }
\tabletail{\hline\multicolumn{4}{|r|}{\small\sl continued on next page}\\\hline}
\tablelasttail{\hline}
\bottomcaption{This table is split across pages}
\par
\begin{supertabular}{| r@{\hspace{6.5mm}}| r@{\hspace{5.5mm}}| r | r|}
1   &     1  &        1  &           1    \\
2   &     4  &       16  &           2    \\
3   &     9  &       81  &           6    \\
4   &    16  &      256  &          24    \\[5mm]
5   &    25  &      625  &         120    \\
6   &    36  &     1296  &         720    \\
7   &    49  &     2401  &        5040    \\
8   &    64  &     4096  &       40320    \\
9   &    81  &     6561  &      362880    \\
10  &   100  &    10000  &     3628800    \\
11  &   121  &    14641  &    39916800    \\
12  &   144  &    20736  &   479001600    \\[.5cm]
\hline & & & \\
13  &   169  &    28561  &  6.22702080E+9 \\[1cm]
14  &   196  &    38416  &  8.71782912E+10\\
15  &   225  &    50625  &  1.30767437E+12\\
16  &   256  &    65536  &  2.09227899E+13\\
17  &   289  &    83521  &  3.55687428E+14\\[5mm]
18  &   324  &   104976  &  6.40237370E+15\\
19  &   361  &   130321  &  1.21645100E+17\\
20  &   400  &   160000  &  2.43290200E+18\\
\end{supertabular}
\end{center}
And here are the definitions:
\begin{verbatim}
\newcommand{\tbsp}{\rule{0pt}{18pt}}
                  % I use \tbsp to get a vertical distance after \hline
\tablefirsthead{\hline  \multicolumn{1}{|c}{\tbsp Number}
                       & \multicolumn{1}{c}{Number$^2$}
                       & Number$^4$
                       & \multicolumn{1}{c|}{Number!} \\ \hline\tbsp  }
\tablehead{\hline \multicolumn{4}{|l|}%
                {\small\sl continued from previous page}\\
           \hline \multicolumn{1}{|c}{\tbsp Number}
                       & \multicolumn{1}{c}{Number$^2$}
                       & Number$^4$
                       & \multicolumn{1}{c|}{Number!} \\ \hline\tbsp  }
\tabletail{\hline\multicolumn{4}{|r|}%
                {\small\sl continued on next page}\\\hline}
\tablelasttail{\hline}
\bottomcaption{This table is split across pages}
\begin{supertabular}{| r@{\hspace{6.5mm}}| r@{\hspace{5.5mm}}| r | r|}
1   &     1  &        1  &           1    \\
2   &     4  &       16  &           2    \\
3   &     9  &       81  &           6    \\
4   &    16  &      256  &          24    \\[5mm]
...
19  &   361  &   130321  &  1.21645100E+17\\
20  &   400  &   160000  &  2.43290200E+18\\
\end{supertabular}
\end{verbatim}

Here is another example whith a {\tt p} column-definition. The tablehead
is the same as above. The tabletail is a double \verb|\hline|
\verb|\arraystretch| is set to {\tt 1.5} and font size is \verb|\small|.
\begin{center}
\tablelasttail{\hline\hline}
\topcaption{This table should also be split accross pages.}
\renewcommand{\arraystretch}{1.5}
\small
\begin{supertabular}{| r@{\hspace{6.5mm}}| r@{\hspace{5.5mm}}| r p{5cm}|}
1  &  1  &  1  &  here is a relative short entry \\
2  &  1  &  1  &  and here is a long entry, where line breaks and line
                  breaks and line breaks have to occur \\
3  &  1  &  1  &  here is also a long entry, where also a line break
                  should occur\\
4  &  1  &  1  &  here is also a long entry, where also a line break
                  should occur\\
5  &  1  &  1  &  here is a relative short entry \\
6  &  1  &  1  &  here is also a long entry, where also a line break
                  should occur\\
7  &  1  &  1  &  here is also a long entry, where also a line break
                  should occur\\
8  &  1  &  1  &  and here is a long entry, where line breaks and line
                  breaks and line breaks have to occur \\
9  &  1  &  1  &  and here is a long entry, where line breaks and line
                  breaks and line breaks have to occur \\
10 &  1  &  1  &  here is also a long entry, where also a line break
                  should occur\\
11 &  1  &  1  &  here is also a long entry, where also a line break
                  should occur\\
12 &  1  &  1  &  here is a relative short entry \\
13 &  1  &  1  &  here is also a long entry, where also a line break
                  should occur\\
14 &  1  &  1  &  and here is a long entry, where line breaks and line
                  breaks and line breaks have to occur \\
15 &  1  &  1  &  and here is a long entry, where line breaks and line
                  breaks and line breaks have to occur \\
16 &  1  &  1  &  here is also a long entry, where also a line break
                  should occur\\
17 &  1  &  1  &  and here is a long entry, where line breaks and line
                  breaks and line breaks have to occur \\
18 &  1  &  1  &  and here is a long entry, where line breaks and line
                  breaks and line breaks have to occur \\
\end{supertabular}
\end{center}

Here is the same table again, but this time using the supertabular*
environment and stretching the table to the full width of the text.

\begin{center}
\tablelasttail{\hline\hline}
\renewcommand{\arraystretch}{1.5}
\small
\begin{supertabular*}{\textwidth}%
                     {| r@{\extracolsep{6.5mm plus 1fil}}|
                        r@{\extracolsep{5.5mm plus 0.5fil}}|
                        r@{\extracolsep{0mm plus 0.1fil}} p{5cm}|}
1  &  1  &  1  &  here is a relative short entry \\
2  &  1  &  1  &  and here is a long entry, where line breaks and line
                  breaks and line breaks have to occur \\
3  &  1  &  1  &  here is also a long entry, where also a line break
                  should occur\\
4  &  1  &  1  &  here is also a long entry, where also a line break
                  should occur\\
5  &  1  &  1  &  here is a relative short entry \\
6  &  1  &  1  &  here is also a long entry, where also a line break
                  should occur\\
7  &  1  &  1  &  here is also a long entry, where also a line break
                  should occur\\
8  &  1  &  1  &  and here is a long entry, where line breaks and line
                  breaks and line breaks have to occur \\
9  &  1  &  1  &  and here is a long entry, where line breaks and line
                  breaks and line breaks have to occur \\
10 &  1  &  1  &  here is also a long entry, where also a line break
                  should occur\\
11 &  1  &  1  &  here is also a long entry, where also a line break
                  should occur\\
12 &  1  &  1  &  here is a relative short entry \\
13 &  1  &  1  &  here is also a long entry, where also a line break
                  should occur\\
14 &  1  &  1  &  and here is a long entry, where line breaks and line
                  breaks and line breaks have to occur \\
15 &  1  &  1  &  and here is a long entry, where line breaks and line
                  breaks and line breaks have to occur \\
\end{supertabular*}
\end{center}

\section{Known problems}

\begin{itemize}
\item When a float occurs on the same page as the start of a supertabular you
can expect unexpected results.

When the float was defined on the same page you might end up with the first
part of the supertabular on a page by its own.
When the list of unprocessed floats is not empty it will be made empty by the
first part of the supertabular (if it is split across pages that is) because it
calls \verb=\clearpage=.
\item You should not use the supertabular {\em inside\/} a floating-environment
such as {\tt table} as this will result in \TeX\ trying to put the whole
supertabular on {\em one\/} page.
\item In some instances you might still end up with overfull \verb=\vbox=
messages. These instances are described in the (documented version of) the
style file itself.
\end{itemize}

\end{document}
