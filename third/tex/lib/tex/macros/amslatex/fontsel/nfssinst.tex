
% This file is part of the NFSS (New Font Selection Scheme) package.
% ------------------------------------------------------------------

\def\fileversion{v1.0a}
\def\filedate{91/03/29}


\documentstyle{article}

 \title{Installation of the new font selection scheme, 
        $\beta$-Test release}

 \author{Frank Mittelbach \and
         Rainer Sch\"opf}

\date{\filedate, \fileversion}


\newcommand{\AmS}{{\the\textfont2 A}\kern-.1667em\lower.5ex\hbox
        {\the\textfont2 M}\kern-.125em{\the\textfont2 S}}
\newcommand{\AmSTeX}{\AmS-\TeX}
\newcommand{\AmSLaTeX}{\AmS-\LaTeX}
\newcommand{\meta}[1]{$\langle${\it#1\/$\rangle$}}


\begin{document}

\maketitle
\begin{abstract}
  This file gives some basic information which are necessary to
  generate a format with the new font selection scheme. It is
  essentially the last part of {\tt fontsel.tex} the article that
  appeared in TUGboat. However, it was modified and updated to reflect
  changes within the font selection scheme, wheras {\tt fontsel.tex}
  was left unchanged.
\end{abstract}

 When setting up a new format one has to start Ini\TeX{} with {\tt
 lplain.tex} as the input file. After displaying some progress report
 on the terminal, {\tt lplain.tex} will try to \verb+\input+ the files
 {\tt hyphen.tex} and {\tt lfonts.tex}.

 It seems a good idea to rename these files because, when \TeX{}
 complains that it cannot find them and asks you to type in another
 file name, you get the chance to substitute your favourate
 hyphenation patterns without changing {\tt lplain.tex} or copying
 something to {\tt hyphen.tex}.  The transcript file will show the
 name of the file used which is very useful to debug weird errors
 (later).

 When the point is reached where \TeX{} wants to read in {\tt
 lfonts.tex}, you now have to specify `{\tt lfonts.new}'.  After
 processing this file (which will take some time), Ini\TeX{} stops
 once more since it cannot find the file {\tt xxxlfont.sty}.  This is
 intentional;  in this way you may now specify the desired
 default by entering one of the following file names:
\begin{description}
 \item[{\tt oldlfont.sty}]
    If you choose this file, your format will be identical to the
     standard \LaTeX{} version 2.09 except that a few additional
     commands (like \verb+\normalshape+) are available.  Of course,
     documents or style options which explicitly refer to things like
     \verb+\tentt+ will produce error messages since such internal
     commands are no longer defined.\footnote{By the way, such
     documents were at no time portable since Leslie Lamport stated
     that it was always permissible to customize {\tt lfonts.tex}
     according to the local needs. Therefore this is {\em not\/} an
     incompatible change.} Nevertheless it is easy to fix the
     problem in such a case: if we know that \verb+\tentt+ referred to
     {\tt cmtt10}, i.e.\ Computer modern typewriter normal at 10pt,
     we can define it as
    \begin{verbatim}
 \newcommand{\tentt}{\fontfamily{cmtt}%
     \fontseries{m}\fontshape{n}\fontsize{10}{12pt}%
     \selectfont}
\end{verbatim}
    Since we assume the `{\tt oldlfont}' option as default, where
     \verb+\tt+ resets series and shape, the definition could be
     shortened to
    \begin{verbatim}
\newcommand{\tentt}{\fontsize{10}{12pt}\tt}
\end{verbatim}
    To get the new way of font selection as described in {\tt
 fontsel.tex} (e.g.\ where \verb+\tt+ simply means to switch to
 another family) you only have to add the `{\tt newlfont}' style
 option to the \verb+\documentstyle+ command in your document.

  \item[\tt newlfont.sty]
    This is just the counterpart to {\tt oldlfont.sty}: it will make
     the new mechanism the default and you have to add `{\tt oldlfont}'
     as a style option if you want to process older documents which
     depend on the old mechanism.

  \item[\tt basefont.tex]
    This file is similar to {\tt newlfont.sty} but does not define the
     \LaTeX{} symbol fonts. These fonts contain only a few characters
     which are also included in the AMS symbol fonts.  Therefore we
     provided the possibility of generating a format which doesn't
     unnecessarily occupy one of the (only) sixteen math groups within
     one math version. Using this file you can easily switch to the
     old scheme (adding `{\tt oldlfont}' as an option), to the new
     scheme with \LaTeX{} symbol fonts (using `{\tt newlfont}') or to
     the new scheme with additional AMS fonts by using either the
     style option `{\tt amsfonts}' (fonts only) or the style option `{\tt
     amstex}' (defining the whole set of \AmSTeX{} macros in a \LaTeX{}
     like syntax).
 \end{description}
We suggest using the {\tt basefont.tex} file since the new font
 selection scheme will be incorporated into \LaTeX{} version 2.10, but
 on installations with many users it might be better to switch
 smoothly to the new font
 selection scheme by first using `{\tt oldlfont}' as a default.

Anyway, after reading the file chosen, \TeX{} will continue by
 processing {\tt latex.tex} and finally displaying the message ``Input
 any local modifications here''.  If you don't dare to do so, use
 \verb+\dump+ to finish the run.  This will leave you with a new {\tt
 .fmt} file (to be put into \TeX's format area) and the corresponding
 transcript file.  It isn't a very good idea to delete this one
 because you might need it later to find out what you did when you
 dumped the format!


 \section{Remarks on the development of this interface}

 We started designing the new font selection scheme around April 1989.
 A first implementation was available after one month's work and
 thereafter the prototype version ran successfully for some months at
 a few sites in Germany and the UK\null. Frank's visit to Stanford as
 well as our work on the `{\tt amstex}' style option brought new
 aspects to our view. The result was a more or less complete
 redefinition of the \LaTeX{} interface for this font mechanism.  It
 was a long way from the first sketch (which was about five pages in
 Frank's notebook) to the current implementation of the interface
 which now consists of nearly 2000 lines of code and about 4000 lines
 of internal documentation. The \AmSTeX{} project itself, which
 triggered this reimplemenation, has about the same dimensions. Surely
 in such a huge software package one will find typos and bugs. But we
 hope that most of the bugs in the code are found by now. It is
 planned that the new font selection scheme will replace the old one
 in \LaTeX{} version 2.10.  We therefore hope that this pre-release
 which runs in version 2.09 will help to find all remaining problems
 so that the switch to version 2.10 will be without discomfort to the
 user.

 \section{Acknowledgements}

During this project we got help from many people. A big `thank you'
to all of them, especially to Michael Downes from the AMS for his
cooperation and help, to Stefan Lindner for his help with the
Atari \TeX{} and to Sebastian Rahtz for playing a willing
guinea-pig. Finally we also want to thank Ron Whitney who did a
marvelous job on all our articles so far. This time we posed some
extra problems because he had to first make a new format in
order to read how to make a new format.

 \section{List of distributed files}

 \begin{description}
  \item[\tt lfonts.new]
   The new version of {\tt lfonts.tex}, to be copied to a file of
   this name after the old {\tt lfonts.tex} has been renamed.
  \item[\tt fontdef.ori]
   The font definitions for the computer modern fonts in the
   distribution by
   Donald~E. Knuth.  To be copied to {\tt fontdef.tex} if this
   selection is to be used.
  \item[\tt fontdef.max]
   Complete font definitions for the computer modern fonts and the
   AMSFonts collections.  To be copied to {\tt fontdef.tex} if this
   selection is to be used.
  \item[\tt preload.ori]
   Preloads the same fonts as the old {\tt
   lfonts.tex} does.  To be copied
   to {\tt preload.tex} if this is desired.
  \item[\tt preload.min]
   Preloads only the absolute minimum of fonts.  To be copied
   to {\tt preload.tex} if this is desired.
  \item[\tt newlfont.sty]
   Selects new version of font selection for \LaTeX.
  \item[\tt oldlfont.sty]
   Selects old version of font selection for \LaTeX.
  \item[\tt basefont.tex]
   Like {\tt newlfont.sty}, but does not define the \LaTeX{} symbol
   fonts.
  \item[\tt margid.sty]
   Style file that
   defines all \meta{math alphabet identifiers} to have one argument.
   This is the default that is built in into the new font selection
   scheme.  Therefore this style file is only necessary if the
   installation decided to load `{\tt nomargid.sty}' at dump time.
  \item[\tt nomargid.sty]
   In contrast to {\tt margid.sty}, defines all
   \meta{math alphabet identifiers} to switch to the
   alphabet.  This style option is necessary if you want to be
   compatible to the old \LaTeX{} syntax {\em in math mode only}.
  \item[\tt tracefnt.sty]
   Style file that allows the tracing of font usage.
   Use \verb=\tracingfonts= with values 1 to 3 and watch
   what happens.
  \item[\tt syntonly.sty]
   Defines the \verb+\syntaxonly+ declaration.  This can be used
   in the preamble of a document to suppress all output.
  \item[\tt euscript.sty]
   Contains the definitions to use the Euler script fonts.
 \end{description}
There are some more files included in the distribution. For a complete
 set see the {\tt readme} files. 

Together with the \AmSLaTeX{} package (that uses the the new font
 selection scheme) there are some additional file that might be of
 interest of many users even even if they don't intend to use the
 mathematical features of the \AmSLaTeX{} package. This includes the
 files:
\begin{description}
  \item[\tt amsfonts.sty]
   Defines the commands to select symbols from the AmSFonts
   collection.
  \item[\tt amsbsy.sty]
   Defines the \verb+\boldsymbol+ command.
  \item[\tt amssymb.sty]
   Defines additional \AmSTeX{} symbols.
  \item[\tt amstext.sty]
   Defines the \AmSTeX{} \verb+\text+ command.
\end{description}
Note, that these files come with the \AmSLaTeX{} distribution and not
 with the standard fontsel package.


\end{document}
