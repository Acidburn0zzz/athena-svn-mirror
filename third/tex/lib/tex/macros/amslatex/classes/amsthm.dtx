%%% ====================================================================
%%%  @LaTeX-file{
%%%     filename  = "amsthm.dtx",
%%%     version   = "1.2d",
%%%     date      = "1996/10/29",
%%%     time      = "09:31:26 EST",
%%%     checksum  = "56577 784 3833 30650",
%%%     author    = "American Mathematical Society",
%%%     copyright = "Copyright (C) 1996 American Mathematical Society,
%%%                  all rights reserved.  Copying of this file is
%%%                  authorized only if either:
%%%                  (1) you make absolutely no changes to your copy,
%%%                  including name; OR
%%%                  (2) if you do make changes, you first rename it
%%%                  to some other name.",
%%%     address   = "American Mathematical Society,
%%%                  Technical Support,
%%%                  Electronic Products and Services,
%%%                  P. O. Box 6248,
%%%                  Providence, RI 02940,
%%%                  USA",
%%%     telephone = "401-455-4080 or (in the USA and Canada)
%%%                  800-321-4AMS (321-4267)",
%%%     FAX       = "401-331-3842",
%%%     email     = "tech-support@ams.org (Internet)",
%%%     supported = "yes",
%%%     keywords  = "latex, amslatex, ams-latex, theorem",
%%%     abstract  = "This is part of the AMS-\LaTeX{} distribution.
%%%                  It is a package which provides multiple theorem
%%%                  styles, unnumbered theorem types, and a proof
%%%                  environment with automatic ending Q.E.D. symbol.
%%%                  Loosely derived from Mittelbach's theorem.sty.",
%%%     docstring = "The checksum field contains: CRC-16 checksum,
%%%                  word count, line count, and character count, as
%%%                  produced by Robert Solovay's checksum utility.",
%%%  }
%%% ====================================================================
% \iffalse
%<*driver>
\NeedsTeXFormat{LaTeX2e}
\documentclass{amsdtx}
\begin{document}
\title{The \pkg{amsthm} package}
\author{American Mathematical Society\\Michael Downes}
\date{29 October 1996 \\
Version 1.02}
\hDocInput{amsthm.dtx}
\end{document}
%</driver>
% \fi
%
% \MakeShortVerb{\|}
%
% \maketitle
% \section{Introduction}
%    The \pkg{amsthm} package is loosely derived from \fn{theorem.sty}
%    version 2.1c; it adds \cn{newtheorem*} for unnumbered environments
%    and changes the extensibility support for loading extra theorem
%    styles from external files: now it is done through the package
%    option mechanism.
%
%    Here are some examples showing the kinds of theorem environment
%    declarations that are possible with the \pkg{amsthm} package.
% \begin{verbatim}
% \newtheorem{prop}{Proposition}
% \newtheorem{thm}{Theorem}[section]
% \newtheorem{lem}[thm]{Lemma}
% \newtheorem*{Zorn}{Zorn's Lemma}
%
% \theoremstyle{definition}
% \newtheorem{dfn}{Definition}
%
% \theoremstyle{remark}
% \newtheorem*{rmk}{Remark}
% \end{verbatim}
%
%    The first four statements all define environments using the default
%    theorem style (`plain'), since there is no prefatory
%    \cn{theoremstyle} declaration. The first statement defines an
%    automatically numbered \env{prop} environment whose headings will
%    look like this: Proposition 1, Proposition 2, and so forth. The
%    second statement defines an environment \env{thm} with numbers
%    subordinate to section numbers, so the headings will look like
%    this: Theorem 1.1, Theorem 1.2, Theorem 1.3, \dots, (in section 2:)
%    Theorem 2.1, Theorem 2.2, and so forth. The third statement defines
%    a \env{lem} environment whose numbers will interleave in sequence
%    with the theorem numbers: Theorem 1.3, Lemma 1.4, Lemma 1.5,
%    Theorem 1.6, and so forth. The fourth statement defines a special
%    unnumbered lemma named `Zorn's Lemma'. The remaining two
%    \cn{newtheorem} statements have no special features except for the
%    \cn{theoremstyle} declarations that cause the \env{dfn} and
%    \env{rmk} environments to have some differences in appearance.
%
%    There are three basic styles provided: The `plain' style produces
%    bold headings and italic body text; the `definition' style produces
%    bold headings and normal body text; the `remark' style produces
%    italic headings and normal body text.
%
%    A \cn{swapnumbers} command allows theorem numbers to be swapped to
%    the front of the theorem headings. Putting \cn{swapnumbers} in your
%    document preamble will cause \emph{all following} \cn{newtheorem}
%    statements to produce number-first headings. (To provide maximum
%    control, \cn{swapnumbers} is designed so that it can be used more
%    than once; each time it is used, theorem numbers will be swapped to
%    the opposite side for all following \cn{newtheorem} statements. But
%    rarely will it need to be invoked more than once per document.)
%
%    There is a \cn{newtheoremstyle} command provided to make the
%    creation of custom theoremstyles fairly easy.
%
%    Usage:
% \begin{verbatim}
%                   #1
% \newtheoremstyle{NAME}%
%     #2          #3          #4
%   {ABOVESPACE}{BELOWSPACE}{BODYFONT}%
%     #5      #6        #7         #8
%   {INDENT}{HEADFONT}{HEADPUNCT}{HEADSPACE}%
%     #9
%   {CUSTOM-HEAD-SPEC}
% \end{verbatim}
%    Leaving the `indent' argument empty is equivalent to entering
%    |0pt|. The `headpunct' and `headspace' arguments are for the
%    punctuation and horizontal space between the theorem head and the
%    following text. There are two special values that may be used for
%    `headspace': a single space means that a normal interword space
%    should be used; ``\cn{newline}'' means that there should be a line
%    break after the head instead of horizontal space. The
%    `custom-head-spec' argument follows a special convention: it is
%    interpreted as the replacement text for an internal three-argument
%    function \cn{thmhead}, i.e., as if you were defining
% \begin{verbatim}
% \renewcommand{\thmhead}[3]{...#1...#2...#3...}
% \end{verbatim}
%    but omitting the initial |\renewcommand{\thmhead}[3]|. The three
%    arguments that will be supplied to \cn{thmhead} are the name,
%    number, and optional note components. Within the replacement text
%    you can (and normally will want to) use other special functions
%    \cn{thmname}, \cn{thmnumber}, and \cn{thmnote}. These will print
%    their argument if and only if the corresponding argument of
%    \cn{thmhead} is nonempty. For example
% \begin{verbatim}
% {\thmname{#1}\thmnumber{ #2}\thmnote{ (#3)}}
% \end{verbatim}
%    This would cause the theorem note \arg{3} to be printed with a
%    preceding space and enclosing parentheses, if it is present, and if
%    it is absent, the space and parentheses will be omitted because
%    they are inside the argument of \cn{thmnote}.
%
%    Finally, if you have an extra bit of arbitrary code that you want
%    to slip in somewhere, the best place to do it is in the `body font'
%    argument.
%
%    The \cn{newtheoremstyle} command is designed to provide, through a
%    relatively simple interface, control over the style aspects that
%    are most commonly changed. Clearly it cannot serve for all possible
%    theorem styles. Therefore there is a second interface provided to
%    allow arbitrary theorem styles by reading suitable definitions from
%    a separate file whose name ends with \fn{.thm}. If the desired
%    style is far from any of the basic styles provided by the
%    \pkg{amsthm} package, writing the definitions may require some
%    expertise in \latex/'s macro language.
%
%    Suppose that you wanted to make a theorem style `exercise' for
%    exercises. Create a file called \fn{exercise.thm} and in it define
%    a command \cs{th@exercise} following the form of the commands
%    \cs{th@plain}, \cs{th@definition}, \cs{th@remark} given below. Then
%    to use the new style, write
% \begin{verbatim}
% \usepackage[exercise]{amsthm}
% ...
% \theoremstyle{exercise}
% \end{verbatim}
%    Similarly, it's possible to place a group of related
%    \cn{newtheoremstyle} statements in a \fn{.thm} file, let's say
%    \fn{stygroup.thm}, so that they could be loaded on demand in
%    various documents by
% \begin{verbatim}
% \usepackage[stygroup]{amsthm}
% \end{verbatim}
%
%    This strategy fails if you want to load a \fn{.thm} file in a
%    document preamble and the \pkg{amsthm} package has already been
%    loaded in the documentclass (e.g., \cls{amsart}). Then you need to
%    use a statement such as
% \begin{verbatim}
% \PassOptionsToPackage{stygroup}{amsthm}
% \end{verbatim}
%    \emph{before} the \cn{documentclass} command.
%
% \StopEventually{}
%
% \section{Implementation}
%    Standard declaration of package name and date.
%    \begin{macrocode}
\NeedsTeXFormat{LaTeX2e}% LaTeX 2.09 can't be used (nor non-LaTeX)
[1994/12/01]% LaTeX date must December 1994 or later
\ProvidesPackage{amsthm}[1996/10/24 v1.2d]
%    \end{macrocode}
%
%    Load some utility functions from \pkg{amsgen} if it is not already
%    loaded. Can't use \cn{RequirePackage} because of the later
%    \cn{ProcessOptions} command, and can't put this after the
%    \cn{ProcessOptions} command because we might need some bits for
%    processing a \fn{.thm} file.
%    \begin{macrocode}
\@ifpackageloaded{amsgen}{}{%%
%% This is file `amsgen.sty',
%% generated with the docstrip utility.
%%
%% The original source files were:
%%
%% amsgen.dtx 
%% 
%%% ====================================================================
%%%  @LaTeX-file{
%%%     filename  = "amsgen.dtx",
%%%     version   = "2.0",
%%%     date      = "1999/11/30",
%%%     time      = "12:33:33 EST",
%%%     author    = "American Mathematical Society",
%%%     copyright = "Copyright 1995, 1999 American Mathematical Society,
%%%                  all rights reserved.  Copying of this file is
%%%                  authorized only if either:
%%%                  (1) you make absolutely no changes to your copy,
%%%                  including name; OR
%%%                  (2) if you do make changes, you first rename it
%%%                  to some other name.",
%%%     address   = "American Mathematical Society,
%%%                  Technical Support,
%%%                  Electronic Products and Services,
%%%                  P. O. Box 6248,
%%%                  Providence, RI 02940,
%%%                  USA",
%%%     telephone = "401-455-4080 or (in the USA and Canada)
%%%                  800-321-4AMS (321-4267)",
%%%     FAX       = "401-331-3842",
%%%     checksum  = "07569 396 1613 14368",
%%%     email     = "tech-support@ams.org (Internet)",
%%%     codetable = "ISO/ASCII",
%%%     keywords  = "latex, amslatex, ams-latex",
%%%     supported = "yes",
%%%     abstract  = "This is part of the AMS-\LaTeX{} distribution.
%%%                  It contains some general internal macros shared
%%%                  by several different files in AMS-\LaTeX{}.",
%%%     docstring = "The checksum field above contains a CRC-16
%%%                  checksum as the first value, followed by the
%%%                  equivalent of the standard UNIX wc (word
%%%                  count) utility output of lines, words, and
%%%                  characters.  This is produced by Robert
%%%                  Solovay's checksum utility.",
%%%  }
%%% ====================================================================
\NeedsTeXFormat{LaTeX2e}% LaTeX 2.09 can't be used (nor non-LaTeX)
[1994/12/01]% LaTeX date must December 1994 or later
\ProvidesFile{amsgen.sty}[1999/11/30 v2.0]
\providecommand{\@saveprimitive}[2]{\begingroup\escapechar`\\\relax
  \edef\@tempa{\string#1}\edef\@tempb{\meaning#1}%
  \ifx\@tempa\@tempb \global\let#2#1%
  \else
    \edef\@tempb{\meaning#2}%
    \ifx\@tempa\@tempb
    \else
      \@latex@error{Unable to properly define \string#2; primitive
      \noexpand#1no longer primitive}\@eha
    \fi
  \fi
  \endgroup}
\let\@xp=\expandafter
\let\@nx=\noexpand
\newtoks\@emptytoks
\def\@oparg#1[#2]{\@ifnextchar[{#1}{#1[#2]}}
\long\def\@ifempty#1{\@xifempty#1@@..\@nil}
\long\def\@xifempty#1#2@#3#4#5\@nil{%
  \ifx#3#4\@xp\@firstoftwo\else\@xp\@secondoftwo\fi}
\long\def\@ifnotempty#1{\@ifempty{#1}{}}
\def\FN@{\futurelet\@let@token}
\def\DN@{\def\next@}
\def\RIfM@{\relax\ifmmode}
\def\setboxz@h{\setbox\z@\hbox}
\def\wdz@{\wd\z@}
\def\boxz@{\box\z@}
\def\relaxnext@{\let\@let@token\relax}
\long\def\new@ifnextchar#1#2#3{%
  \let\reserved@d= #1%
  \def\reserved@a{#2}\def\reserved@b{#3}%
  \futurelet\@let@token\new@ifnch
}
\def\new@ifnch{%
  \ifx\@let@token\reserved@d \let\reserved@b\reserved@a \fi
  \reserved@b
}
\def\@ifstar#1#2{\new@ifnextchar *{\def\reserved@a*{#1}\reserved@a}{#2}}
\@ifundefined{every@math@size}{%
\let\every@math@size=\every@size
\def\glb@settings{%
     \expandafter\ifx\csname S@\f@size\endcsname\relax
       \calculate@math@sizes
     \fi
     \csname S@\f@size\endcsname
      \ifmath@fonts
        \begingroup
          \escapechar\m@ne
          \csname mv@\math@version \endcsname
          \globaldefs\@ne
          \let \glb@currsize \f@size
          \math@fonts
        \endgroup
        \the\every@math@size
      \else
      \fi
}
\def\set@fontsize#1#2#3{%
    \@defaultunits\@tempdimb#2pt\relax\@nnil
    \edef\f@size{\strip@pt\@tempdimb}%
    \@defaultunits\@tempskipa#3pt\relax\@nnil
    \edef\f@baselineskip{\the\@tempskipa}%
    \edef\f@linespread{#1}%
    \let\baselinestretch\f@linespread
      \def\size@update{%
        \baselineskip\f@baselineskip\relax
        \baselineskip\f@linespread\baselineskip
        \normalbaselineskip\baselineskip
        \setbox\strutbox\hbox{%
          \vrule\@height.7\baselineskip
                \@depth.3\baselineskip
                \@width\z@}%
%%%     \the\every@size
        \let\size@update\relax}%
  }
}{}% end \@ifundefined test
\newdimen\ex@
\addto@hook\every@math@size{\compute@ex@}
\def\compute@ex@{%
  \begingroup
  \dimen@-\f@size\p@
  \ifdim\dimen@<-20\p@
    \global\ex@ 1.5\p@
  \else
    \advance\dimen@10\p@ \multiply\dimen@\tw@
    \edef\@tempa{\ifdim\dimen@>\z@ -\fi}%
    \dimen@ \ifdim\dimen@<\z@ -\fi \dimen@
    \advance\dimen@-\@m sp % fudge factor
    \vfuzz\p@
    \def\do{\ifdim\dimen@>\z@
      \vfuzz=.97\vfuzz
      \advance\dimen@ -\p@
      \@xp\do \fi}%
    \do
    \dimen@\p@ \advance\dimen@-\vfuzz
    \global\ex@\p@
    \global\advance\ex@ \@tempa\dimen@
  \fi
  \endgroup
}
\def\@addpunct#1{\ifnum\spacefactor>\@m \else#1\fi}
\def\frenchspacing{\sfcode`\.1006\sfcode`\?1005\sfcode`\!1004%
  \sfcode`\:1003\sfcode`\;1002\sfcode`\,1001 }
\def\nomath@env{\@amsmath@err{%
  \string\begin{\@currenvir} allowed only in paragraph mode%
}\@ehb% "You've lost some text"
}
\def\Invalid@@{Invalid use of \string}
\endinput
%%
%% End of file `amsgen.sty'.
}
%    \end{macrocode}
%
%    The \cn{theoremstyle} command is very simple except for the need to
%    warn about an unknown theoremstyle.
%    \begin{macrocode}
\newcommand{\theoremstyle}[1]{%
  \@ifundefined{th@#1}{%
    \PackageWarning{amsthm}{Unknown theoremstyle `#1'}%
    \thm@style{plain}%
  }{%
    \thm@style{#1}%
  }%
}
%    \end{macrocode}
%
%    \begin{macrocode}
\newtoks\thm@style
\thm@style{plain}
%    \end{macrocode}
%
%    This code for handling theorem body and header font is a
%    simplification of the code in Mittelbach's \fn{theorem} package.
%    For consistency we make the header punctuation a token register as
%    well. And we add a separate font specification for the optional
%    note (since in AMS publications the note usually takes a different
%    font). The \cs{normalfont} that in Mittelbach's code resided here
%    is transferred to \cs{@thm}.
%
%    Note: \cs{thm@bodyfont} is an artifact of borrowing from |theorem.dtx|
%    and requires more consideration of possible repercussions before it
%    is adopted or removed.  For now, comment it out, and advise anyone who
%    asks that the recommended way to make such changes is to use the
%    \cs{newtheoremstyle} facility.  [bnb, 1996/09/24]
%
%    Note: Similarly for \cs{thm@notefont}; current methods inadequate,
%    more work needed.  [mjd, 1995/08/07]
%    What's really needed is a full-fledged systematic approach for
%    specifying the desired order and formatting of the three identified
%    parts of a theorem head (name, number, note). In the absence of
%    such a solution, the use of \cs{thm@notefont} (but not
%    \cs{theoremnumberfont}) seems to be more harm than good: if you need
%    to change it, as likely as not you'll need to change something else
%    that's not as accessible, so you might as well go to
%    \cs{newtheoremstyle} or direct surgery on \cs{thmhead@plain} or whatever.
%    \begin{macrocode}
\newtoks\thm@bodyfont
\thm@bodyfont{\itshape}
\newtoks\thm@headfont
\thm@headfont{\bfseries}
\newtoks\thm@notefont
\thm@notefont{}
\newtoks\thm@headpunct
\thm@headpunct{.}
%    \end{macrocode}
%    Vertical spacing: initialize to current value of \cs{topsep}.
%    If a document class loads the \pkg{amsthm} package it
%    should take care to set these variables explicitly, if current
%    \cs{topsep} is not the desired value.
%    \begin{macrocode}
\newskip\thm@preskip \thm@preskip\topsep
\newskip\thm@postskip \thm@postskip\topsep
%    \end{macrocode}
%    Modify \cn{newtheorem} to add |*| option. If a |*| is found, pass
%    it on to \cs{@xnthm} as the first argument. (This information
%    enables us to handle two different possibilities in a single
%    function \cs{@xnthm} instead of needing two separate functions.)
%    \begin{macrocode}
\renewcommand{\newtheorem}{\@ifstar{\@xnthm *}{\@xnthm \relax}}
%    \end{macrocode}
%
%    Check to see if an optional arg is present after the first
%    mandatory arg (which is \arg{2} at the moment since the |*| or
%    non-|*| is \arg{1}).
%    \begin{macrocode}
\def\@xnthm#1#2{%
  \let\@tempa\relax
  \@xp\@ifdefinable\csname #2\endcsname{%
    \global\@xp\let\csname end#2\endcsname\@endtheorem
    \ifx *#1% unnumbered, need to get one more mandatory arg
      \edef\@tempa##1{%
        \gdef\@xp\@nx\csname#2\endcsname{%
          \@nx\@thm{\@xp\@nx\csname th@\the\thm@style\endcsname}%
            {}{##1}}}%
    \else % numbered theorem, need to check for optional arg
      \def\@tempa{\@oparg{\@ynthm{#2}}[]}%
    \fi
  }%
  \@tempa
}
%    \end{macrocode}
%
%    Environment numbered together with a previously defined
%    environment.
%
%    Arg1: env name, e.g. `thm'\par
%    Arg2: optional sibling counter\par
%    Arg3: heading text e.g. `Theorem'
%    \begin{macrocode}
\def\@ynthm#1[#2]#3{%
%    \end{macrocode}
%    If optional arg \arg{2} is empty, call \cs{@xthm} to look for a
%    possible optional arg in terminal position. Note that
%    the two optional args are mutually exclusive. As \arg{2} is a
%    counter name and must be processed by \cs{csname} anyway,
%    we can use a simpler test instead of \cs{@ifempty}.
%    \begin{macrocode}
  \ifx\relax#2\relax
    \def\@tempa{\@oparg{\@xthm{#1}{#3}}[]}%
  \else
    \@ifundefined{c@#2}{%
      \def\@tempa{\@nocounterr{#2}}%
    }{%
      \@xp\xdef\csname the#1\endcsname{\@xp\@nx\csname the#2\endcsname}%
      \toks@{#3}%
      \@xp\xdef\csname#1\endcsname{%
        \@nx\@thm{%
          \let\@nx\thm@swap
            \if S\thm@swap\@nx\@firstoftwo\else\@nx\@gobble\fi
          \@xp\@nx\csname th@\the\thm@style\endcsname}%
            {#2}{\the\toks@}}%
      \let\@tempa\relax
    }%
  \fi
  \@tempa
}
%    \end{macrocode}
%
%    Environment numbered relative to the counter given as \arg{3}. This
%    function should really be named \cs{@znthm} but we're trying to
%    save a bit of hash table and string pool by reusing one of the
%    command names rendered obsolete by the amsthm option.
%
%    Arg1: env name e.g. `thm';
%    Arg2: heading text e.g. `Theorem';
%    Arg3: parent counter e.g. section.
%    \begin{macrocode}
\def\@xthm#1#2[#3]{%
%    \end{macrocode}
%    Set up the counter \verb'c@#1' and optionally add it to the reset
%    list of counter \arg{3}. As \arg{3} is a
%    counter name and must be processed by \cs{csname} anyway,
%    we can use a simpler test instead of \cs{@ifempty}.
%    \begin{macrocode}
  \ifx\relax#3\relax
    \newcounter{#1}%
  \else
    \newcounter{#1}[#3]%
%    \end{macrocode}
%    Define \cn{thexxx} to be \verb'\theyyy.\arabic{xxx}' (assuming
%    default values of punctuation and numbering style). The use of
%    \cs{xdef} here is inherited from the old \LaTeX{} code, I'm not
%    sure it's a good idea in general, but there should not be any
%    problems unless someone changes the value of \cs{@thmcounter} or
%    \cs{@thmcounter}.
%    \begin{macrocode}
    \@xp\xdef\csname the#1\endcsname{\@xp\@nx\csname the#3\endcsname
      \@thmcountersep\@thmcounter{#1}}%
  \fi
  \toks@{#2}%
  \@xp\xdef\csname#1\endcsname{%
    \@nx\@thm{%
      \let\@nx\thm@swap
        \if S\thm@swap\@nx\@firstoftwo\else\@nx\@gobble\fi
      \@xp\@nx\csname th@\the\thm@style\endcsname}%
      {#1}{\the\toks@}}%
}
%    \end{macrocode}
%
%    If arg \arg{2} is empty, this is an unnumbered environment;
%    otherwise \arg{2} is the name of a counter. \arg{3} is descriptive
%    name such as ``Theorem'' or ``Lemma''. Arg \arg{1} is the style
%    function, for example \cs{th@plain}.
%    \begin{macrocode}
\def\@thm#1#2#3{\normalfont
  \trivlist
%    \end{macrocode}
%    Explicitly set plain style here, then override parts as necessary
%    in the function provided as \arg{1}. As far as I can tell the
%    standard article/book documentclasses don't handle nonstandard
%    values of \cn{labelsep} well: if you start any kind of list or
%    trivlist and modify the value of labelsep, then the same value will
%    be used for embedded enumerate's or itemize's because the normal
%    value is not restored by \cs{@listi} (nor by \texttt{ii,iii,...}).
%    So let's save current labelsep by hand in order to restore it later.
%    \begin{macrocode}
  \edef\@restorelabelsep{\labelsep\the\labelsep}%
  \labelsep.5em\relax \let\thmheadnl\relax
  \let\thm@indent\noindent % no indent
  \let\thm@swap\@gobble
  \thm@headfont{\bfseries}% heading font bold
  \thm@headpunct{.}% add period after heading
  \thm@preskip\topsep
  \thm@postskip\thm@preskip
  #1% style overrides
  \@topsep \thm@preskip               % used by first \item
  \@topsepadd \thm@postskip           % used by \@endparenv
  \def\@tempa{#2}\ifx\@empty\@tempa
    \def\@tempa{\@oparg{\@begintheorem{#3}{}}[]}%
  \else
    \refstepcounter{#2}%
    \def\@tempa{\@oparg{\@begintheorem{#3}{\csname the#2\endcsname}}[]}%
  \fi
  \@tempa
}
%    \end{macrocode}
%
%    This variation of the \cs{@thm} command is no longer needed. The
%    variation \cs{@xthm} was commandeered for \cn{newtheorem} use.
%    \begin{macrocode}
\let\@ythm\relax
%    \end{macrocode}
%
%    Init \cn{thmname} etc.
%    \begin{macrocode}
\let\thmname\@iden \let\thmnote\@iden \let\thmnumber\@iden
%    \end{macrocode}
%
%  \begin{macro}{\@upn}
%    If a suitable italic font with upright numbers and punctuation is
%    available, this function should be redefined to be a no-op.
%    \begin{macrocode}
\providecommand\@upn{\textup}
%    \end{macrocode}
%  \end{macro}
%
%    Definitions for theorem heads.
%    \begin{macrocode}
\def\thmhead@plain#1#2#3{%
%    \end{macrocode}
%    To allow for the case where the thmname part is empty and the
%    heading consists only of a number (don't laugh, we have
%    examples from real mathematical manuscripts), we don't add the
%    space at the beginning of thmnumber unless \arg{1} is nonempty.
%    \begin{macrocode}
  \thmname{#1}\thmnumber{\@ifnotempty{#1}{ }#2}%
%    \end{macrocode}
%    In thmnote we always add a leading space, i.e., assuming that
%    at least one of the preceding parts will always be present.
%    \begin{macrocode}
  \thmnote{ {\the\thm@notefont(#3)}}}
\let\thmhead\thmhead@plain
%    \end{macrocode}
%    Swappedhead is for the case where the number precedes the
%    word "Theorem".
%    \begin{macrocode}
\def\swappedhead#1#2#3{%
  \thmnumber{#2}\thmname{\@ifnotempty{#2}{. }#1}%
  \thmnote{ {\the\thm@notefont(#3)}}}
%    \end{macrocode}
%
%    In \cs{@begintheorem} \cn{thmheadnl} is called after the theorem
%    head: maybe a newline, otherwise a no-op.
%    \begin{macrocode}
\let\thmheadnl\relax
%    \end{macrocode}
%
%    If argument \arg{2} is empty, then this is an unnumbered
%    environment. Otherwise \arg{2} is a numbering command such as
%    \cn{thexyz}.
%    \begin{macrocode}
\def\@begintheorem#1#2[#3]{%
  \item[\normalfont % reset in case body font is abnormal
%    \end{macrocode}
%    The standard weird compensation for labelsep space inside a
%    \cs{trivlist} \cn{item}:
%    \begin{macrocode}
  \hskip\labelsep
  \the\thm@headfont
  \thm@indent
%    \end{macrocode}
%    Changes to \cs{thmnumber} and \cs{thmnote} are local to this group.
%    \begin{macrocode}
  \@ifempty{#1}{\let\thmname\@gobble}{\let\thmname\@iden}%
  \@ifempty{#2}{\let\thmnumber\@gobble}{\let\thmnumber\@iden}%
  \@ifempty{#3}{\let\thmnote\@gobble}{\let\thmnote\@iden}%
%    \end{macrocode}
%    The \cs{thm@swap} function selects either \cs{swappedhead} or
%    \cs{thmhead}.
%    \begin{macrocode}
  \thm@swap\swappedhead\thmhead{#1}{#2}{#3}%
%    \end{macrocode}
%    I can't think of any example where the after-head punctuation
%    should be omitted so it seems correct not to use \cs{@addpunct}
%    here.
%    \begin{macrocode}
  \the\thm@headpunct]%
  \@restorelabelsep
  \thmheadnl % possibly a newline.
  \ignorespaces}
%    \end{macrocode}
%
%  \begin{macro}{\nonslanted}
%    The \cn{nonslanted} command changes the current font to
%    \cn{upshape} if it is \cn{itshape} or \cn{slshape}. This is used
%    for document structure numbers that should be consistently upright
%    in all contexts.
%    \begin{macrocode}
\def\nonslanted{\relax
%    \end{macrocode}
%    Can't do a direct \cs{ifx} between \cs{f@shape} and \cs{itdefault}
%    because the latter is \cs{long} (grumble grumble).
%    \begin{macrocode}
  \@xp\let\@xp\@tempa\csname\f@shape shape\endcsname
  \ifx\@tempa\itshape\upshape
  \else\ifx\@tempa\slshape\upshape\fi\fi}
%    \end{macrocode}
%  \end{macro}
%
%  \begin{macro}{\swapnumbers}
%    The \cn{swapnumbers} command sets a switch \cs{thm@swap} that is
%    used by \cn{newtheorem}. To conserve hash table we load
%    \cs{thm@swap} with two uses; the first one is needed only in
%    \cn{newtheorem} declarations and the second one is needed only in
%    typesetting theorem environments.
%    \begin{macrocode}
\def\swapnumbers{\edef\thm@swap{\if S\thm@swap N\else S\fi}}
\def\thm@swap{N}%
%    \end{macrocode}
%  \end{macro}
%
%    \cs{@opargbegintheorem} not needed, \cs{@oparg} utility serves
%    instead.
%    \begin{macrocode}
\let\@opargbegintheorem\relax
%    \end{macrocode}
%
%    Except for the body font, default values are built into \cs{@thm}.
%    \begin{macrocode}
\def\th@plain{%
%%  \let\thm@indent\noindent % no indent
%%  \thm@headfont{\bfseries}% heading font is bold
%%  \thm@notefont{}% same as heading font
%%  \thm@headpunct{.}% add period after heading
%%  \let\thm@swap\@gobble
%%  \thm@preskip\topsep
%%  \thm@postskip\theorempreskipamount
  \itshape % body font
}
%    \end{macrocode}
%
%    Theorem style `definition' is the same as `plain' except for the
%    body font.
%    \begin{macrocode}
\def\th@definition{%
  \normalfont % body font
}
%    \end{macrocode}
%
%    Theorem style `remark' differs from `plain' in head font and  body
%    font. Also smaller spacing above and below.
%    \begin{macrocode}
\def\th@remark{%
  \thm@headfont{\itshape}%
  \normalfont % body font
  \thm@preskip\topsep
  \divide\thm@preskip\tw@
  \thm@postskip\thm@preskip
}
%    \end{macrocode}
%
%    The standard definition of \cs{@endtheorem} is just
%    \cs{endtrivlist}, but that doesn't automatically start a new
%    paragraph, so we add \cs{@endpefalse} in order to ensure a new
%    paragraph.
%    \begin{macrocode}
\def\@endtheorem{\endtrivlist\@endpefalse }
%    \end{macrocode}
%
%  \begin{macro}{\newtheoremstyle}
%    An easy way to make a not too complicated variant theorem style.
%    Usage:
% \begin{verbatim}
%                   #1
% \newtheoremstyle{NAME}%
%     #2          #3          #4
%   {ABOVESPACE}{BELOWSPACE}{BODYFONT}%
%     #5      #6        #7         #8
%   {INDENT}{HEADFONT}{HEADPUNCT}{HEADSPACE}%
%     #9
%   {CUSTOM-HEAD-SPEC}
% \end{verbatim}
%    \begin{macrocode}
\newcommand{\newtheoremstyle}[9]{%
%    \end{macrocode}
%    Empty or 0pt for \arg{5} is translated to \cs{noindent}.
%    \begin{macrocode}
  \@ifempty{#5}{\dimen@\z@skip}{\dimen@#5\relax}%
  \ifdim\dimen@=\z@
%    \end{macrocode}
%    \arg{4} is body font. Extra code could be included there if
%    necessary.
%    \begin{macrocode}
    \toks@{#4\let\thm@indent\noindent}%
  \else
    \toks@{#4\def\thm@indent{\noindent\hbox to#5{}}}%
  \fi
%    \end{macrocode}
%    Arg \arg{8} is a glue spec for the space after the head. As
%    a proper glue spec for `normal interword space' is rather hard to
%    write, we recognize an argument of |{ }| as a special case and
%    translate internally to the necessary fontdimen equivalent.
%    Furthermore, if \arg{8} consists entirely of \cn{newline}, then we
%    will perform a line break after the theorem head instead of adding
%    horizontal space. At the moment [1995/01/23] this is not perfectly
%    well implemented because of complications with the way \latex/'s
%    \cn{item} adds a heading to the vertical list; for best results
%    there should not be anything (not even a blank line) after the
%    |\begin{xxx}| command.
%    \begin{macrocode}
  \def\@tempa{#8}\ifx\space\@tempa
%    \end{macrocode}
%    Notice that we disregard stretch and shrink for labelsep =
%    interwordspace.
%    \begin{macrocode}
    \toks@\@xp{\the\toks@ \labelsep\fontdimen\tw@\font\relax}%
  \else
    \def\@tempb{\newline}%
    \ifx\@tempb\@tempa
      \toks@\@xp{\the\toks@ \labelsep\z@skip
        \def\thmheadnl{%
%    \end{macrocode}
%    In the following line, the `noskipsec' switch prevents a following
%    list item from running in on the same line as the theorem head; the
%    `nobreak' switch helps get the vertical spacing right.
%    \begin{macrocode}
          \@noskipsectrue \global\@nobreaktrue
%    \end{macrocode}
%    If the first thing in the theorem is a list, this definition of
%    \cs{everypar} will be overridden with something suitable.
%    \begin{macrocode}
          \everypar{\global\@minipagefalse \global\@newlistfalse
            \global\@inlabelfalse \global\@nobreakfalse
%    \end{macrocode}
%    Remove the parindent box and put down the theorem head in its
%    place.
%    \begin{macrocode}
            {\setbox\z@\lastbox}\box\@labels\par
%    \end{macrocode}
%    For the `newline' style of theorem head, there's no easy way to
%    adjust the vertical space between the head and the following text.
%    A user syntax decision, basically. Don't want to deal with it at
%    the moment. [mjd,1995/07/27]
%    \begin{macrocode}
            \nobreak\vskip-\parskip
            \everypar{}\noindent}}%
      }%
    \else
      \toks@\@xp{\the\toks@ \labelsep#8\relax}%
    \fi
  \fi
  \begingroup \th@plain % to set \thm@preskip and postskip
  \@defaultunits\@tempskipa#2\thm@preskip\relax\@nnil
  \@defaultunits\@tempskipb#3\thm@postskip\relax\@nnil
  \xdef\@gtempa{\thm@preskip\the\@tempskipa
    \thm@postskip\the\@tempskipb\relax}%
  \endgroup
  \@temptokena\@xp{\@gtempa
    \thm@headfont{#6}\thm@headpunct{#7}%
  }%
  \@ifempty{#9}{%
    \let\thmhead\thmhead@plain
  }{%
    \@namedef{thmhead@#1}##1##2##3{#9}%
    \@temptokena\@xp{\the\@temptokena
      \@xp\let\@xp\thmhead\csname thmhead@#1\endcsname}%
  }%
  \@xp\xdef\csname th@#1\endcsname{\the\toks@ \the\@temptokena}%
}
%    \end{macrocode}
%  \end{macro}
%
%
%  \begin{macro}{\qed}
%    Define \cn{qed} for end of proof. This command might
%    occur in math mode, in a displayed equation, but it should never
%    occur in inner math mode in ordinary paragraph text.
%    \begin{macrocode}
\DeclareRobustCommand{\qed}{%
  \ifmmode % if math mode, assume display: omit penalty etc.
  \else \leavevmode\unskip\penalty9999 \hbox{}\nobreak\hfill
  \fi
%    \end{macrocode}
%    The hbox is to prevent a line break within the \cn{qedsymbol} if it
%    is defined to be something composite--- e.g., things like
%    \verb"(Corollary 1.2) \openbox" as are occasionally done.
%    \begin{macrocode}
  \quad\hbox{\qedsymbol}}
%    \end{macrocode}
%  \end{macro}
%
%    The reason that we do not simply use the \cn{square} symbol from
%    msam for the open-box qed symbol is that we want to avoid requiring
%    users to have the AMSFonts font package. And the \fn{lasy} \cn{Box}
%    is too large.
%    \begin{macrocode}
\newcommand{\openbox}{\leavevmode
%    \end{macrocode}
%    I think I got these numbers from measuring \fn{msam}'s \cn{square}
%    but I forgot to make notes at the time. [mjd,1995/01/25]
%    \begin{macrocode}
  \hbox to.77778em{%
  \hfil\vrule
  \vbox to.675em{\hrule width.6em\vfil\hrule}%
  \vrule\hfil}}
\newcommand{\qedsymbol}{\openbox}
%    \end{macrocode}
%
%    The proof environment is never numbered, and has a \cn{qed} at the
%    end, which makes it inconvenient to use \cn{newtheorem} for
%    defining it. Also authors frequently need to substitute an
%    alternative heading text (e.g. `Proof of Lemma 4.3')
%    instead of the default `Proof'. For all these reasons we define the
%    proof environment here instead of leaving it for authors to define.
%    \begin{macrocode}
\newenvironment{proof}[1][\proofname]{\par
  \normalfont
  \topsep6\p@\@plus6\p@ \trivlist
  \item[\hskip\labelsep\itshape
    #1\@addpunct{.}]\ignorespaces
}{%
  \qed\endtrivlist
}
%    \end{macrocode}
%    Default for \cn{proofname}:
%    \begin{macrocode}
\newcommand{\proofname}{Proof}
%    \end{macrocode}
%
%    Any option given in the \cn{usepackage} statement will be treated
%    as the name of a file containing additional theorem style
%    definitions.
%    \begin{macrocode}
\DeclareOption*{\input{\CurrentOption .thm}}
\ProcessOptions
%    \end{macrocode}
%
%    For reference:
% \begin{verbatim}
% From: tycchow@math.mit.edu (Timothy Y. Chow)
% Subject: Suppressing theorem numbering in LaTeX
% Message-ID: <1994Aug11.234754.22523@galois.mit.edu>
% Date: Thu, 11 Aug 94 23:47:54 GMT
% To: tex-news@SHSU.EDU
%
% A friend of mine wants numbering of theorems, conjectures, and so on
% suppressed if there is only one of them in his article.  In other words
% he wants "Conjecture 1" to appear as simply "Conjecture" if there is no
% Conjecture 2.  What is the best way to go about doing this?
% ...
% \end{verbatim}
%    Maybe something clever can be done to make the desired behavior
%    happen automatically. Note that this would seem to be a general
%    numbering problem rather than a theorem-specific one, because
%    similar behavior would be desirable for appendixes: according to
%    standard publishing practice, if there's only one it is titled just
%    `Appendix', and if there are more than one they are titled
%    `Appendix A', `Appendix B', and so on.
%
%    The usual \cs{endinput} to ensure that random garbage at the end of
%    the file doesn't get copied by \fn{docstrip}.
%    \begin{macrocode}
\endinput
%    \end{macrocode}
%
% \changes{v1.2a}{1995/02/01}{Added missing calls to theoremnotefont}
% \changes{v1.2b}{1995/08/07}{Changed some names to avoid confusion}
%    Changed some names from e.g., \cs{theoremheadfont} to
%    \cs{thm@headfont} to avoid confusion with variables of the same
%    name (but used differently) from the `thm' package.
% \changes{v1.2b}{1995/08/07}{Added restorelabelsep}
% \changes{v1.2b}{1995/08/07}{Removed \cs{theoremnotefont}, current methods
%   inadequate/more work needed}
% \changes{v1.2d}{1996/10/15}{Upgraded to correspond to in-house version;
%   commented out \cs{theorembodyfont} pending further study}
%
% \CheckSum{560}
% \Finale
