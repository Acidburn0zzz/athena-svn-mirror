\documentstyle{article}
\font\cirth=cirth
\input num
\def\nj{n$\!\!$\j}
\def\C#1{\c{#1}$_{_{#1}}$}
\def\bs{$\backslash$}
\baselineskip=18pt
\begin{document}
\begin{center}
THE CIRTH
\end{center}

\noindent
The {\sl Certhas Daeron} was originally devised to represent the sounds
of Sindarin only. The oldest {\sl cirth} were \C{1}, \C{2}, \C{5}, \C{6};
\C{8}, \C{9}, \C{12}; \C{18}, \C{19}, \C{22}; \C{29}, \C{31}; \C{35},
\C{36}; \C{39}, \C{42}, \C{46}, \C{50}; and a {\sl certh} varying
between \C{13} and \C{15}. The assignment of values was unsystematic.
\C{39}, \C{42}, \C{46} and \C{50} were vowels and remained so in all later
developments. \C{13} and \C{15} were used for {\sl h} or {\sl s}, according
as \C{35} was used for {\sl s} or {\sl h}. This tendency to hesitate
in the assignment of values for {\sl s} and {\sl h} continued in
later arangements. In those characters that consisted of a
`stem' and a `branch', \C{1}--\C{31}, the attachment of the branch was,
if on one side only, usually made on the right side.
The reverse was not infrequent, but had no phonetic significance.

The extension and elaboration of this {\sl certhas} was called in its older
form the {\sl Angerthas Daeron}, since the additions to the old {\sl cirth}
and their re-organization was attributed to Daeron. The principal additions,
however, the introductions of two new series, \C{13}--\C{17}, and 
\C{23}--\C{28}, were actually most probably inventions of the Noldor of 
Eregion, since they were used for the representation of sounds not found
in Sindarin.

In the rearrangement of the {\sl Angerthas} the following principles
are observable (evidently inspired by the F\"eanorian system):
(1) adding a stroke to a brance added a `voice';
(2) reversing the {\sl certh} indicated opening to a `spirant';
(3) placing the branch on both sides of the stem added voice and
nasality.
These principles were regularly carried out, except in one point.
For (archaic) Sindarin a sign for a spirant {\sl m} (or nasal {\sl v})
was required, and since this could best be provided by a reversal of the sign
for {\sl m}, the reversible \C{6} was given the value {\sl m}, but \C{5}
was given the value {\sl hw}.

\C{36}, the theoretic value of which was {\sl z} was used, in spelling
Sindarin or Quenya, for {\sl ss}: cf.\ F\"eanorian 31. \C{39} was used for
either {\sl i} or {\sl y} consonant); \C{34}, \C{35} were used indifferently
for {\sl s}; and \C{38} was used for the frequent sequence {\sl nd}, although
it was not clearly related in shape to the dentals.

\bigskip

In the Table of Values those on the left are, when seperated by --, the
values of the older {\sl Angerthas}. Those on the right are the
value of the Dwarvish {\sl Angerthas Moria}\footnote{Those in (\ ) are
values only found in Elvish use: $\star$ marks}.
The Dwarves of Moria, as can be seen, introduced a number of unsystematic
changes in value, as well as certain new {\sl cirth}:
\C{37}, \C{40}, \C{41}, \C{53}, \C{55}, \C{56}. The dislocation in
values was due to mainly two causes:
(1) the alteration in the values of \C{34}, \C{35}, \C{54} respectively to
{\sl h}, (the clear or glottal beginning of a work with an initial vowel
that appeared in Khuzdul), and {\sl s};
(2) the abandonment of the \C{14}, \C{16} for which the Dwarves substitutde
\C{29}, \C{30}. The consequent use of 12 for {\sl r},
the invention of \C{53} for {\sl n} (and its confusion with \C{22});
the use of \C{17} as {\sl z}, to go with \C{54} in its value {\sl s}, and
the consequent use of \C{36} as \nj\ and the new {\sl certh} \C{37} for {\sl ng}
may also be observed. The new \C{55}, \C{56} were in origin a halved form
of \C{46}, and were used for vowels like those heard in English {\sl butter},
which were frequent in Dwarvish and in the Westron. When weak or evanescent
they were often reduced to a mere stroke without a stem. This {\sl Angerthas
Moria} is represented in the tomb-inscription.

The Dwarves of Erebor used a further modification of
this system, known as the mode or Erebor, and exmplified in the Book or
Mazarbul. Its chief characteristics were: the use of \C{43} as {\sl z};
of \C{17} as {\sl ks} ({\sl x}); and the invention of two new {\sl cirth},
\C{57}, \C{58} for {\sl ps} and {\sl ts}. They also reintroduced \C{14},
\C{16} for the values {\sl j}, {\sl zh}; but used \C{29}, \C{30} for
{\sl g}, {\sl gh}, or as mere variants of \C{19}, \C{21}. These peculiarities
are not included in the table, except for the special Ereborian {\sl cirth}, 
\C{57}, \C{58}.

\begin{center}
{\sc the angerthas}\\
{\sc table of values}\\
\begin{tabular}{|lc|lc|lc|lc|}
\hline
\small  1&\cirth p&\small 16&\cirth zh&\small 31&\cirth l&\small 46&\cirth e\\
\small  2&\cirth b&\small 17&\cirth nj&\small 32&\cirth lh&\small 47&\cirth E\\
\small  3&\cirth f&\small 18&\cirth k&\small 33&\cirth \char18&\small 48&\cirth a\\
\small  4&\cirth v&\small 19&\cirth g&\small 34&\cirth s&\small 49&\cirth A\\
\small  5&\cirth hw&\small 20&\cirth kh&\small 35&\cirth S&\small 50&\cirth o\\
\small  6&\cirth m&\small 21&\cirth gh&\small 36&\cirth z&\small 51&\cirth O \char25\\
\small  7&\cirth mb&\small 22&\cirth N&\small 37&\cirth \char19&\small 52&\cirth \char26 \char27\\
\small  8&\cirth t&\small 23&\cirth kw&\small 38&\cirth nd \char21&\small 53&\cirth\char32\\
\small  9&\cirth d&\small 24&\cirth gw&\small 39&\cirth i&\small 54&\cirth h\\
\small 10&\cirth th&\small 25&\cirth \char12&\small 40&\cirth y&\small 55&\cirth \char28\\
\small 11&\cirth dh&\small 26&\cirth \char13&\small 41&\cirth hy&\small 56&\cirth \char29\\
\small 12&\cirth n&\small 27&\cirth ngw&\small 42&\cirth u&\small 57&\cirth ps\\
\small 13&\cirth ch&\small 28&\cirth nw&\small 43&\cirth U&\small 58&\cirth ts\\
\small 14&\cirth j&\small 29&\cirth r&\small 44&\cirth w&         &\cirth c\\
\small 15&\cirth sh&\small 30&\cirth rh&\small 45&\cirth \char23 \char24&\&&\cirth \&\\
\hline
\end{tabular}\\
\bigskip
\begin{tabular}{|lc|lc|lc|lc|}
\hline
\small  1&\rm p&\small 16&\rm zh&\small 31&\rm l&\small 46&\rm e\\
\small  2&\rm b&\small 17&\rm nj--z&\small 32&\rm lh&\small 47&\rm \=e\\
\small  3&\rm f&\small 18&\rm k&\small 33&\rm ng--nd&\small 48&\rm a\\
\small  4&\rm v&\small 19&\rm g&\small 34&\rm s--h&\small 49&\rm \=a\\
\small  5&\rm hw&\small 20&\rm kh&\small 35&\rm s--'&\small 50&\rm o\\
\small  6&\rm m&\small 21&\rm gh&\small 36&\rm z--\nj&\small 51&\rm \=o\\
\small  7&\rm (mh)mb&\small 22&\rm \nj--n&\small 37&\rm ng$^\star$&\small 52&\rm \"o\\
\small  8&\rm t&\small 23&\rm kw&\small 38&\rm nd--nj&\small 53&\rm n$^\star$\\
\small  9&\rm d&\small 24&\rm gw&\small 39&\rm i(y)&\small 54&\rm h--s\\
\small 10&\rm th&\small 25&\rm khw&\small 40&\rm y$^\star$&\small 55&\rm $\star$\\
\small 11&\rm dh&\small 26&\rm ghw,w&\small 41&\rm hy$^\star$&\small 56&\rm $\star$\\
\small 12&\rm n--r&\small 27&\rm ngw&\small 42&\rm u&\small 57&\rm ps$^\star$\\
\small 13&\rm ch&\small 28&\rm nw&\small 43&\rm \=u&\small 58&\rm ts$^\star$\\
\small 14&\rm j&\small 29&\rm r--j&\small 44&\rm w&         &\rm $+$h\\
\small 15&\rm sh&\small 30&\rm rh--zh&\small 45&\rm \"u&         &\rm \&\\
\hline
\end{tabular}
\end{center}

\noindent
{\bf USING THE CIRTH FROM \TeX}

\noindent
The name of the font, as distributed, is {\tt cirth} and can be simply
accessed by {\tt \bs font\bs cirth=cirth}.
The normal letters are mapped according to the older {\sl Angerthas}
where possible. The letter values and ligatures are indicated on the
table below.
Additionally the file {\tt num.tex} is provided that allows the
characters to be accesses by referenced to their numeric entry point
in Tolkien's Table of Values.
There are two interfaces for this.
Firstly the macro {\tt \bs c} takes a single paramer which should
be a number and coresponds to the table entry. (for example {\tt \bs c\{24\}}
produces \c{24}.)
Alternatively you can type the english for the number preceded by a
`c' (e.g. {\tt \bs ctwentyfour} produces \ctwentyfour). In the case
where there are two {\sl cirth} for a single entry {\tt \bs c} takes
the first and the english macros are suffixed by either an
`a' or a `b'. For example {\tt \bs c\{38\}} produces \c{38} and
you must use {\tt \bs cthirtyeighta} for \cthirtyeighta and
{\tt \bs cthirtyeightb} for \cthirtyeightb.

\begin{center}
{\sc the angerthas}\\
\begin{tabular}{|cc|cc|cc|cc|}
\hline
\cirth  p&\tt  p&\cirth  zh&\tt  zh&\cirth  l&\tt  l&\cirth  e&\tt  e\\
\cirth  b&\tt  b&\cirth  nj&\tt  nj&\cirth lh&\tt lh&\cirth  E&\tt  E\\
\cirth  f&\tt  f&\cirth   k&\tt   k&\c{33}   &\tt 33&\cirth  a&\tt  a\\
\cirth  v&\tt  v&\cirth   g&\tt   g&\cirth  s&\tt  s&\cirth  A&\tt  A\\
\cirth hw&\tt hw&\cirth  kh&\tt  kh&\cirth  S&\tt  S&\cirth  o&\tt  o\\
\cirth  m&\tt  m&\cirth  gh&\tt  gh&\cirth  z&\tt  z&\cirth  O&\tt  O\\
\cirth mb&\tt mb&\cirth   N&\tt   N&\c{37}   &\tt 37&\c{52}   &\tt 52\\
\cirth  t&\tt  t&\cirth  kw&\tt  kw&\cirth nd&\tt nd&\c{53}   &\tt 53\\
\cirth  d&\tt  d&\cirth  gw&\tt  gw&\cirth  i&\tt  i&\cirth  h&\tt  h\\
\cirth th&\tt th&\cirth khw&\tt khw&\cirth  y&\tt  y&\c{55}   &\tt 55\\
\cirth dh&\tt dh&\cirth ghw&\tt ghw&\cirth hy&\tt hy&\c{56}   &\tt 56\\
\cirth  n&\tt  n&\cirth ngw&\tt ngw&\cirth  u&\tt  u&\cirth ps&\tt ps\\
\cirth ch&\tt ch&\cirth  nw&\tt  nw&\cirth  U&\tt  U&\cirth ts&\tt ts\\
\cirth  j&\tt  j&\cirth   r&\tt   r&\cirth  w&\tt  w&\cirth  c&\tt  c\\
\cirth sh&\tt sh&\cirth  rh&\tt  rh&\c{38}   &\tt 38&\cirth \&&\tt \&\\
\hline
\end{tabular}\\
\end{center}

The files {\tt cirbf.mf}, {\tt cirsl.mf}, and {\tt cirss.mf} have
also been provided that produce (respectively) boldface, slanted,
and sans-serif Cirth fonts. The boldface fonts stands out well
and can be used when inserting Cirth into normal text. For Cirth
on its own I reccomend the normal font at 12 point. The Sans-serif
font eliminates the serifs (as expected) and makes the pen round,
giving very clear characters. (With the normal slanted pen the
slanted strokes to the left are darker than to the right.) I can
think of no use for the slanted font (but it was easy to do!).
MetaFont hacks can easily combine the options and produce a
slanted bold font without serifs if they so chose.

Please send all comments, criticisms or improvements to
me. e-mail: jaymin@maths.tcd.ie, or by mail to
Jo Jaquinta, 44 Bancroft Avenue, Tallaght, Dublin 24, Ireland.
\end{document}
